%
%  This document contains chapter 2 of the thesis.
%

\chapter{EXPERIMENT DESIGN}\label{ch:experiment-design}
% TODO: What is the purpose of these experiments?

% TODO: Maybe make a system design section/chapter? This would go over how
%  the actualy system works as opposed to how the experiments are set up.


\section{Experiments}\label{sec:experiments}
- Infinite (a la \cite{Cohensius2017})
- Finite
- Expert proxies, untrained inactive
- Untrained proxies, expert inactive
- Finite with agent costs

\subsection{Criteria}\label{subsec:criteria}
- Time (squared?)
- Accuracy/Precision \& Recall
- Cost

\subsection{Parameters}\label{subsec:parameters}
- \# of Proxies
- \# of Non-Proxies
- Distribution of votes
    - Uniform
    - Gaussian
    - Bimodal about center
    - Skewed?
- Dimensions?
- Cost per proxy
- Cost per non-proxy
- Seed
    - Metaparameter--allows reproduction of an iteration. Each iteration
      will have a unique seed.

\section{Mechanisms}\label{sec:mechanisms}
- Explain what mechanisms are, as well as my expansion using weight mechanisms.

\subsection{Voting Mechanisms}\label{subsec:voting-mechanisms}
- Explain what voting mechanisms are

\subsubsection{Candidate Mechanisms}\label{subsubsec:candidate-mechanisms}
- Explain what candidate mechanisms are

- Median
- Borda
- Ranked Choice
- Plurality? Likely not going to work well
- . . .

\subsubsection{Average Mechanisms}\label{subsubsec:average-mechanisms}
- Explain what average mechanisms are

- Mean
- Weight by Borda
- Weight by Ranked Choice
- . . .

\subsection{Weight Mechanisms}\label{subsec:weight-mechanisms}
- Explain what weight mechanisms are

- Vote for closest (implemented by \cite{Cohensius2017})
- Borda
- Ranked Choice
- . . .


\section{Experiment Design}\label{sec:experiment-design}
% TODO: I don't currently have anything in here about agents with different
%  costs. I might not get to that until after running the cost-less
%  experiments, but I ought to look for opportunities to implement it.
%
%  Costs can potentially be added post-experiment, since the outcome of the
%  experiment does not depend on the costs of the agents. This would
%  primarily allow for identifying a cost-ratio and ideal number of agents.
%
%  It might also be interesting to develop a system that uses a budget
%  without a predefined number of proxies and inactives.

The experiments will be performed in two phases.
First, all data will be collected.
This is done primarily to make all desired data available during the analysis
process.
% FIXME: Is there really nothing else I could say here?
Afterwards, the data will be analysed in an attempt to identify any patterns,
strengths, or weaknesses proxy voting has as a measurement verification tool.

\subsection{Data Collection}\label{subsec:data-collection}
% Steps
In order to ensure consistency, each experiment will follow the same general
steps.
% TODO: Fill this out more

\begin{enumerate}[label=\textbf{\arabic*}., leftmargin=2\parindent]
    \titleditem{Generate proxies}
    Given the space in which an experiment is to take place, the system will
    generate a number of proxies using the given distribution.
    The number and distribution of these proxies will depend on the
    parameters of the experiment.

    \titleditem{Generate inactive voters}
    The system will then generate a set of inactive voters.
    Again, the number and distribution these voters will depend on the
    experiment's parameters and will often not be the same as the proxies'
    parameters.
    In particular, the distribution and range used will generally be
    different from the proxies' since the proxies serve as `experts' and so
    should be more accurate.

    \titleditem{Assign weights to proxies}
    Using the given \hyperref[subsec:weight-mechanisms]{weighting mechanism},
    each inactive voter will apply weights to one or more proxies.
    The system will sum these weights to get a total weight for each proxy.

    \titleditem{Estimate truth}
    Once all the weights have been calculated, the system will coalesce
    the weights into an estimated value using the provided
    \hyperref[subsec:voting-mechanisms]{voting mechanism}.
    This value is the system's estimation of the true value.

    \titleditem{Calculate error}
    Finally, the error of the system can be calculated by using the true
    value and the system's estimation of the truth.
    This error can be used to determine how well the system worked and
    ultimately determine if proxy voting is useful under the given parameters.
\end{enumerate}

% FIXME: I might want to specify how many times a combination is ran
Each combination of parameters will be run a number of times in order to obtain
a fair spread of error for that set of parameters.
After each iteration, the combination of parameters will be recorded as well
as the estimated truth, the error, as well as the rest of the
\hyperref[subsec:criteria]{criteria}.

% TODO: Caveats

\subsection{Analysis}\label{subsec:analysis}
% TODO: What is done after an experiment?


\section{Further Work and Improvements}\label{sec:further-work-and-improvements}
