%
%  This document contains chapter 2 of the thesis.
%

% TODO: Maybe make a system design section/chapter? This would go over how
%  the actualy system works as opposed to how the experiments are set up.
\chapter{EXPERIMENT DESIGN}\label{ch:experiment-design}
In order to identify the ability of proxy voting to serve as a correctional
method, a number of experiments have been devised.
The primary goal of these analyses is to identify the strengths and
weaknesses of proxy voting under certain conditions and attempt to minimize
the costs of ensuring accurate measurements.


\section{Preliminaries}\label{sec:preliminaries}

\subsection{System Design and Agent Space}\label{subsec:system-design-and
-agent-space}
Each system \system\ will output some estimation \systemtruth\ of the truth
\truth.
This will be done by allowing the set of agents belonging to \system,
known as \systemagents, to observe \truth\ and provide their
own estimation \agenttruth, where \agent\ is an individual agent.
These estimations will then be compiled into \systemtruth\ using
the system's \hyperref[subsec:voting-mechanisms]{voting mechanism}.
An agent need not provide the same \agenttruth\ for the same \truth,
and by extension \systemtruth\ may not be the same each time a
system is run.

All agents will operate within a defined space~\systemspace, where
$\systemspace \subseteq \real^k$ and $k \geq 1$ dimensions, as described
by~\cite[para.~2.1]{Cohensius2017}.
Specifically, this paper will focus on interval spaces of
$\systemspace = [a, b]^1$.
Spaces of $\systemspace = [a, b]^2$ or higher could also be used, but since
measurements are typically performed one at a time in a singular dimension (one
doesn't measure both the length and width with a single measurement--it takes
two) and would ultimately only increase the complexity of the system, this
exercise is left for further work.
Using the space \systemspace\ allows each agent's preference, which for this
paper's purposes represents \agenttruth, to be represented as a position within
this space.
This likewise means \systemtruth\ will be a position within \systemspace.

% TODO: Talk about \agentweight and \systemweight
% TODO: Talk about \proxies vs inactive

\subsection{Agent Types}\label{subsec:agent-types}
This study explores the idea of using finite voters, as opposed to the
infinite voters described in~\cite{Cohensius2017}.
These voters' positions will be randomly selected using a distribution, which
will vary from experiment to experiment.
By design, there are two primary types of voters: expert and untrained.

\textit{Expert voters} are agents that give a more accurate estimate of the
true value.
These voters have tighter extremes for estimated values or are chosen from
a more accurate distribution (such as Gaussian about \truth) allowing
them to be more certain in their estimation.
These voters represent better, though likely more expensive, methods for
taking a measurement.  % TODO: such as. . .

\textit{Untrained voters}, on the other hand, are agents with a worse
distribution or more extreme minimums and maximums.
These agents have a greater degree on error, and while they might be correct
about the true value they are unable to be as certain about their estimations
as the expert voters due to their possibility for error.
Measurement methods represented by these voters are less accurate, though
likely cheaper.  % TODO: such as. . .

%  - Finite with agent costs


\section{Experiment Criteria and Parameters}\label{sec:experiment-criteria
-and-parameters}
Both untrained and expert voters have the same goal: to enhance the accuracy of
the system.
They will work in tandem in order to accomplish this goal.
There are two immediately identifiable ways to use both types of voters which
will be explored in this paper.
First, the experts can be used as proxies and the untrained as inactive voters.
This exploits the idea discussed in~\cite{Miller1969, Mueller1972} of
allowing experts to guide the system while still pulling information from the
other agents.

The second obvious way to employ these agents is using the untrained voters
and the proxies and allowing the experts to transfer their voting power to them.
This is the direct inverse of the previous setup.
The intuition behind this strategy is since untrained voters are likely
cheaper and so can be more numerous, they have a better chance of being closer
to the true value due to the law of large numbers.
Naturally, this technique would be most beneficial when untrained agents use a
distribution with a decent likelihood of the true value being included, such
as Gaussian or uniform.
While this study does not plan to exploit the law of large numbers anywhere
near its full potential, this setup may be beneficial where many measurements
can be taken or when untrained agents are considerably less expensive than
experts.

Both setups will be examined in this study.
Fortunately, the process of how to create the setups is identical.
The only difference is which type of agent serves as proxies.

\subsection{Criteria}\label{subsec:criteria}
A system is only as good as its ability to perform.
In order to gauge the ability or goodness of proxy voting as a form of
measurement correction, three primary criteria have been identified.

First, the most important part of the system is the accuracy.
If a system using proxy voting performs worse than simply averaging all votes
it would have been better to skip the additional complexities of using
proxies.
Accuracy will be measured using squared~loss~\loss.
Fortunately, since agents operate within \systemspace, calculating the
squared loss for a system~\systemloss, is as simple as calculating the
squared distance
\begin{align}
    \systemloss &= {
        \left(\sqrt{
            (\systemtruth_1 - \truth_1)^2 +
            \mathellipsis +
            (\systemtruth_k - \truth_k)^2
        }\right)
    }^2 \nonumber \\
    &= (\systemtruth_1 - \truth_1)^2 +
    \mathellipsis +
    (\systemtruth_k - \truth_k)^2
    \text{,}\label{eq:loss}
\end{align}
where $\systemtruth_k$ and $\truth_k$ are the $k$th
dimension of the estimated and real truth respectively.

The next criteria used to judge systems is the cost of the system.
The total cost of a system \systemcost\ is calculated as
\begin{equation}
    \systemcost =
    \sum_{\agent \in \agents}{
        \agentcost
    }
    \text{,}\label{eq:cost}
\end{equation}
where \agentcost\ is the cost of an agent.
This cost is an arbitrary value of an arbitrary unit which accounts for time,
monetary cost, or any other cost the measuring technique requires.
While in reality there may be some fluctuation in cost for the same
measurement, for simplicity this study considers all agents of the same type
to have the same cost.
The goal of system cost as a criteria is to determine if a less-costly system
could perform just as well if not better than one with a higher cost.
% TODO: Would ratios between expert an untrained agents are beneficial? Maybe
% useful for knowing if adding an agent is useful?

% - Number of measurements?
%   - Expert
%   - Untrained
% - Time (squared?)/Calculation complexity
%   - Useful to know if humans could use it.

\subsection{Parameters}\label{subsec:parameters}

\subsubsection{System Parameters}\label{subsubsec:system-parameters}
Each system will have a different set of parameters.
First, each system will need a set of proxy voters.
These proxies can be either experts or untrained, depending on which setup is
being used.
Similarly, each system will also need a set of inactive voters which can also
be either experts or untrained.

Finally, each system will receive two mechanisms: a
\hyperref[subsec:voting-mechanisms]{voting mechanism} and a
\hyperref[subsec:weighting-mechanisms]{weighting mechanism}.
The weighting mechanism will be used to apply weights to each proxy, and the
voting mechanism will be used to condense all~\agenttruth\ to
produce~\systemtruth.

\subsubsection{Agent Parameters}\label{subsubsec:agent-parameters}
Similar to systems, each type of agent will have a different set of parameters.
There are two parameters that make creates the most distinction between
agents: their vote distribution and their extents.

The extents determine the maximum and minimum value away from \truth the
agent is able to vote.
For example, an agent with an extent of 10 could vote anywhere in the
interval $[\truth - 10, \truth + 10]$.
While the extent determines the general boundaries, some distributions allow
numbers slightly outside these ranges.
For these distributions, the percent of votes inside the extents are
specified in \autoref{tab:distributions-percent-inside-extents}.
These extents are not meant to be known by the system, but are rather an
estimation of the range of error about \truth\ a measurement in the real world
might have.

The vote distribution determines which distribution is used by the agent when
attempting to measure \truth.
The distributions used in this study are found
in~\autoref{tab:distributions}.
These distributions, combined with the extent, have a large influence on the
agent's ability to measure \truth.

Finally, each agent has a cost \agentcost\ associated with it.
This cost is an arbitrary unit used to represent the monetary cost, time
required, and/or any other value required by the agent.
The cost can be assigned or reassigned at any time before calculating the
system cost since the system itself does not need to know the cost of the
agents it's using to estimate~\truth.

% - Dimensions?
% - Seed
%     - Metaparameter--allows reproduction of an iteration. Each iteration
%       will have a unique seed.
%     - Might be best to not discuss this in the paper.


\section{Mechanisms}\label{sec:mechanisms}
\textit{Mechanisms} are a group of strategies used to accomplish some goal.
These are different from the mechanisms described by~\cite{Cohensius2017} in
that they do not exclusively include voting rules.
Instead, these have been relabeled to `voting mechanisms' and the definition
of mechanisms has been expanded to also include rules used by agents to
assign weights to proxies, which rules are named `weighting mechanisms.'

\subsection{Voting Mechanisms}\label{subsec:voting-mechanisms}
\textit{Voting mechanisms} are a concept borrowed from~\cite{Cohensius2017},
there simply labeled `mechanisms.'
These are algorithms used to condense proxies and their weights, which are
required to be positive, into \systemtruth.
These algorithms can be broken down into two primary categories: candidate
mechanisms and average mechanisms.

\subsubsection{Candidate Mechanisms}\label{subsubsec:candidate-mechanisms}
\textit{Candidate mechanisms} are mechanisms that select a proxy to serve as
\systemtruth.
% TODO: Fill out strengths and weaknesses
This study experiments with a number of candidate mechanisms, which are
described below.

\mechanismheader{Median}
The median mechanism is one of the mechanisms used in~\cite{Cohensius2017}.
This study build off their by using finite voters instead of infinite, as
well as applying weighting mechanisms which will result in different results.
The formula for the median mechanism is the same as the one used
in~\cite[para.~2.4]{Cohensius2017}, which was described as, with some
symbolic changes,
\begin{equation*}
    \textbf{md}(\systemproxies) =
    \min\left\{\agent_i \in \systemproxies \text{ s.t. }
    |\{j: \agent_i \geq \agent_j\}| \geq
    |\{j: \agent_i < \agent_j\}|
    \right\}
    \text{,}
\end{equation*}
meaning the mechanisms selects the minimum agent that has an equal or greater
number of agents below it as it has above it in terms of \agenttruth.

For weighted scenarios, the median agent will be the minimum agent whose sum of
weights below plus itself is greater than or equal to $\frac{\systemweight}{2}$,
as described with
\begin{equation}
    \textbf{md}(\systemagents) = \min\left\{
    \agent_i \in \systemproxies \text{ s.t. }
    \weightof{\agent_i} +
    \sum_{\agent_j \in \systemproxies}^{\agent_{i - 1}} \weightof{\agent_j}
    \geq \frac{\systemweight}{2}
    \right\}
    \text{,}\label{eq:median-mechanism}
\end{equation}
where \systemproxies\ is ordered by \agenttruth.

\mechanismheader{Plurality}
The plurality mechanism simply selects the proxy with the highest weight.
It is described by
\begin{equation}
    \textbf{pl}(\systemproxies) =
    \argmax \weightof{\systemproxies}
    \text{,}\label{eq:plurality-mechanism}
\end{equation}
meaning the proxy with the maximum weight.
While a seemingly simply mechanism, this voting mechanism combined with some
weighting mechanism allows for common voting systems.
For example, plurality used with \hyperref[para:borda]{borda weighting}
provides the Borda count system.

\mechanismheader{Instant Runoff}\label{para:candidate-instant-runoff}
The instant runoff mechanism does not use agent-assigned weights, but rather
ranks each proxy on behalf of the agents according to distance.
Once all proxies are ranked, each proxy receives a score according to how
many first place votes they receive.
The proxy with the lowest number of votes is eliminated, and its votes are
redistributed to the second choice for each agent.
This continues until some proxy has the majority ($>$50\%) of votes, at which
time that proxy is selected as \systemtruth.

\mechanismheader{Weighted Instant Runoff}
\label{para:candidate-weighted-instant-runoff}
The weighted instant runoff mechanism works the same as instant runoff, but uses
the weights provided by each agent instead.
Since it is possible that a weighting mechanism will not provide second or
third choices, the agents whose choice is eliminated will need to
re-provide weights each iteration.
Once some proxy has the majority ($>$50\%) of weights it is selected as
\systemtruth.

\subsubsection{Average Mechanisms}\label{subsubsec:average-mechanisms}
\textit{Average mechanisms} are mechanisms that produce a \systemtruth\ that
is not necessarily the position of any one agent, though it may be.
They work by averaging the positions of multiple proxies to some degree.
% TODO: Fill out strengths and weaknesses

\mechanismheader{Mean}
The mean mechanism is another mechanism originally used in
~\cite{Cohensius2017}.
It is reemployed here to make use of the additional weighting mechanisms and
finite voters.
The formula for the mean mechanism is
\begin{equation*}
    \mathbf{mn}(\systemproxies) =
    \frac{1}{\systemweight}
    \sum_{\agent_i \in \systemproxies} {\weightof{\agent_i} \cdot \agent_i}
    \text{,}\label{eq:mean-mechanism}
\end{equation*}
meaning the sum of each proxy multiplied by its weight divided by
\systemweight, where \systemweight\ is the total system weight.
This mechanism is the core of the average mechanisms, and is used as a final
step in most other average mechanisms.

\mechanismheader{Weight by Ranked Choice}
Each proxy is ranked by each inactive agent according to how close they are
to the inactive agent.
The number of times each proxy is ranked first is counted, and the proxy with
the least votes is eliminated and given a score of 1.
The process continues from $1 \mathellipsis n$, where $n$ is the number of
proxies, at which point the mean mechanism is with the scores as the weights.

\mechanismheader{Weight by Instant Runoff}\label{para:avg-instant-runoff}
The weight by instant runoff mechanism is similar to the
\hyperref[para:candidate-instant-runoff]{candidate version} of instant
runoff, but instead of select the majority proxy the remaining votes for
each proxy are used as the weights for the mean mechanism.

\mechanismheader{Averaged Weighted Instant Runoff}
This mechanism is the same as
\hyperref[para:candidate-weighted-instant-runoff]{weighted instant runoff},
but instead of selecting the majority proxy as \systemtruth\ the remaining votes
are used as the weights for the mean mechanism.
Essentially, this is
\hyperref[para:avg-instant-runoff]{weight by instant runoff}, but the weights
applied by the inactive agents are used instead of ranking each proxy.

\subsection{Weighting Mechanisms}\label{subsec:weighting-mechanisms}
Weighting mechanisms are strategies by which inactive agents assign weights
to proxies.
All final weights are required to be positive in order to simplify
calculating \systemtruth through the voting mechanisms.

\mechanismheader{Vote for Closest}
The vote for closest weighting mechanism is the simplest weighting mechanism.
It is an adaptation of the infinite mechanism used in~\cite{Cohensius2017},
where instead of each proxy being assigned the weight of the boundaries
between proxies, each inactive voter simply votes for the proxy closest to
itself.
This is equivalent to assigning a weight of 1 to the closet proxy, and giving
the rest 0.

\mechanismheader{Distance Voting}
This mechanism uses the negative distance between the proxies and an agent as
the weight.
The negative distance is used so the proxy closest to the given agent
receives the highest weight.
Once all distances for all agents are calculated, each weight need to be
adjusted to be positive using the formula
\begin{equation}
    g{(\agent)} = \lvert
    \min{\weightof{\systemproxies}} - \weightof{\agent}
    \rvert
    \text{,}\label{eq:distance-voting-adjustment}
\end{equation}
which gives the absolute value of the minimum weight minus the agent's weight.
This results in the minimum weight being equal to 0, and the highest weight
equal to the distance between the minimum and the maximum weight.

\mechanismheader{Borda}\label{para:borda}
The Borda mechanism allows each inactive voter to weight each proxy.
Each proxy is rated some value in $[1 \mathellipsis n]$, with the closest
given $n$ and the furthest $1$.
These values are used as the weights for whichever voting mechanism is used.
This should result in something similar to distance voting, but with a set
range of scores.


\section{Experiment Design}\label{sec:experiment-design}
% TODO: I don't currently have anything in here about agents with different
%  costs. I might not get to that until after running the cost-less
%  experiments, but I ought to look for opportunities to implement it.
%
%  Costs can potentially be added post-experiment, since the outcome of the
%  experiment does not depend on the costs of the agents. This would
%  primarily allow for identifying a cost-ratio and ideal number of agents.
%
%  It might also be interesting to develop a system that uses a budget
%  without a predefined number of proxies and inactives.

The experiments will be performed in two phases.
First, all data will be collected.
This is done primarily to make all desired data available during the analysis
process.
% FIXME: Is there really nothing else I could say here?
Afterwards, the data will be analysed in an attempt to identify any patterns,
strengths, or weaknesses proxy voting has as a measurement verification tool.

In order to simplify various calculations, the true value for each experiment
will be 0.
This simplifies selecting the positions of each voter without any meaningful
differences, as well as making it trivial to calculate loss by changing the
\hyperref[eq:loss]{loss formula} to
\begin{align}
    \systemloss &=
    (\systemtruth_1 - 0)^2 +
    \mathellipsis +
    (\systemtruth_k - 0)^2
    \nonumber \\
    &=
    (\systemtruth_1)^2 +
    \mathellipsis +
    (\systemtruth_k )^2
    \text{.}\label{eq:adjusted-loss}
\end{align}

Additionally, all extents can be set to 1, which can easily be scaled to any
desired number simply my multiplying by that number.
This, again, should not make any meaningful differences to the final analysis.

\subsection{Data Collection}\label{subsec:data-collection}
% Steps
In order to ensure consistency, each experiment will follow the same general
steps.
% TODO: Fill this out more

\begin{enumerate}[label=\textbf{\arabic*}., leftmargin=2\parindent]
    \titleditem{Generate proxies}
    Given the space in which an experiment is to take place, the system will
    generate a number of proxies using the given distribution.
    The number and distribution of these proxies will depend on the
    parameters of the experiment.

    \titleditem{Generate inactive voters}
    The system will then generate a set of inactive voters.
    Again, the number and distribution these voters will depend on the
    experiment's parameters and will often not be the same as the proxies'
    parameters.
    In particular, the distribution and range used will generally be
    different from the proxies' since the proxies serve as `experts' and so
    should be more accurate.

    \titleditem{Assign weights to proxies}
    Using the given \hyperref[subsec:weighting-mechanisms]{weighting mechanism},
    each inactive voter will apply weights to one or more proxies.
    The system will sum these weights to get a total weight for each proxy.

    \titleditem{Estimate truth}
    Once all the weights have been calculated, the system will coalesce
    the weights into an estimated value using the provided
    \hyperref[subsec:voting-mechanisms]{voting mechanism}.
    This value is the system's estimation of the true value.

    \titleditem{Calculate error}
    Finally, the error of the system can be calculated by using the true
    value and the system's estimation of the truth.
    This error can be used to determine how well the system worked and
    ultimately determine if proxy voting is useful under the given parameters.
\end{enumerate}

% FIXME: I might want to specify how many times a combination is ran
Each combination of parameters will be run a number of times in order to obtain
a fair spread of error for that set of parameters.
After each iteration, the combination of parameters will be recorded as well
as the estimated truth, the error, as well as the rest of the
\hyperref[subsec:criteria]{criteria}.

% TODO: Caveats

\subsection{Analysis}\label{subsec:analysis}
Once all data is collected, analyses will be performed in order to answer a
number of questions.
\begin{enumerate}[label=\textbf{Q\arabic*}., leftmargin=2\parindent]
    \item Which combinations of mechanisms are better than simply averaging
    all agents?
    \item How many inactive agents per proxy should be used?
    \item Which combinations of mechanisms provide the best accuracy for the
    least cost?
\end{enumerate}

% TODO: Fill me out more

\section{Further Work and Improvements}\label{sec:further-work-and-improvements}
% FIXME: This might need to be moved to Chapter 4
% - Multiple levels of expertise. Expert voters could have varying levels of
% accuracy, allowing for more complex systems.
% - (If I decide to restrict experts and untrained to the same mins and maxes)
% Allow for untrained to have a lower restriction, making them less accurate.
% - Independently track time, monetary cost, etc. for system cost (described
% in the cost section of Criteria. Useful identifying multiple systems for
% different costs.

