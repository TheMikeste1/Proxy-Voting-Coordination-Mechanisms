%
%  This document contains chapter 2 of the thesis.
%

\chapter{EXPERIMENT DESIGN}\label{ch:experiment-design}

\section{Experiments}\label{sec:experiments}
- Infinite (a la \cite{Cohensius2017})
- Finite
    - Expert proxies, untrained inactive
    - Untrained proxies, expert inactive
- Finite with agent costs

\subsection{Criteria}\label{subsec:criteria}
- Time (squared?)
- Accuracy/Precision \& Recall
- Cost

\subsection{Parameters}\label{subsec:parameters}
- \# of Proxies
- \# of Non-Proxies
- Distribution of votes
    - Uniform
    - Gaussian
    - Bimodal about center
    - Skewed?
- Dimensions?
- Cost per proxy
- Cost per non-proxy

\section{Mechanisms}\label{sec:mechanisms}
- Explain what mechanisms are, as well as my expansion using weight mechanisms.

\subsection{Voting Mechanisms}\label{subsec:voting-mechanisms}
- Explain what voting mechanisms are

\subsubsection{Candidate Mechanisms}\label{subsubsec:candidate-mechanisms}
- Explain what candidate mechanisms are

- Median
- Borda
- Ranked Choice
- Plurality? Likely not going to work well
- . . .

\subsubsection{Average Mechanisms}\label{subsubsec:average-mechanisms}
- Explain what average mechanisms are

- Mean
- Weight by Borda
- Weight by Ranked Choice
- . . .

\subsection{Weight Mechanisms}\label{subsec:weight-mechanisms}
- Explain what weight mechanisms are

- Vote for closest (implemented by \cite{Cohensius2017})
- Borda
- Ranked Choice
- . . .

\section{Experiment Design}\label{sec:experiment-design}
% Steps
In order to ensure consistency, each experiment will follow the same general
steps.

\begin{enumerate}[label=\textbf{\arabic*}., leftmargin=2\parindent]
    \titleditem{Generate proxies}
    Given the space in which an experiment is to take place, the system will
    generate a number of proxies using the given distribution.
    The number and distribution of these proxies will depend on the
    parameters of the experiment.
    \titleditem{Generate inactive voters}
    The system will then generate a set of inactive voters.
    Again, the number and distribution these voters will depend on the
    experiment's parameters and will often not be the same as the proxies'
    parameters.
    In particular, the distribution and range used will generally be
    different from the proxies' since the proxies serve as `experts' and so
    should be more accurate.
    \titleditem{Assign weights to proxies}
    Using the given \hyperref[subsec:weight-mechanisms]{weighting mechanism},
    each inactive voter will apply weights to one or more proxies.
    The system will sum these weights to get a total weight for each proxy.
    \titleditem{Estimate truth}
    % TODO
    % Voting mechanisms
    \titleditem{Calculate error}
    % TODO
\end{enumerate}

% TODO: Caveats



\section{Further Work and Improvements}\label{sec:further-work-and-improvements}
