%
%  This document contains chapter 2 of the thesis.
%

% TODO: Maybe make a system design section/chapter? This would go over how
%  the actualy system works as opposed to how the experiments are set up.
\chapter{EXPERIMENT DESIGN}\label{ch:experiment-design}
In order to identify the ability of proxy voting to serve as a correctional
method, a number of experiments have been devised.
The primary goal of these analyses is to identify the strengths and
weaknesses of proxy voting under certain conditions and attempt to minimize
the costs of ensuring accurate measurements.

\section{Preliminaries}\label{sec:preliminaries}
\subsection{System Design and Agent Space}\label{subsec:system-design-and
-agent-space}
Each system $\mathbb{S}$ will output some estimation $\mathbb{T}_\mathbb{S}$
of the truth $\mathbb{T}$.
This will be done by allowing a set of agents $\mathbb{A}$ to observe
$\mathbb{T}$ and provide their own estimation $\mathbb{T}_a$, where $a$ is an
individual agent.
These estimations will then be compiled into $\mathbb{T}_\mathbb{S}$ using
the system's \hyperref[subsec:voting-mechanisms]{voting mechanism}.
An agent need not provide the same $\mathbb{T}_a$ for the same $\mathbb{T}$,
and by extension $\mathbb{T}_\mathbb{S}$ may not be the same each time a
system is run.

All agents will operate within a defined space~$\chi$, where $\chi \subseteq
\mathbb{R}^k$ and $k \geq 1$ dimensions, as described
by~\cite[para.~2.1]{Cohensius2017}.
Specifically, this paper will focus on interval spaces of $\chi = [a, b]^1$.
Spaces of $\chi = [a, b]^2$ or higher could also be used, but since
measurements are typically performed one at a time in a singular dimension (one
doesn't measure both the length and width with a single measurement--it takes
two) and would ultimately only increase the complexity of the system, this
exercise is left for further work.
Using the space $\chi$ allows each agent's preference, which for this paper's
purposes represents $\mathbb{T}_a$, to be represented as a position within
this space.
This likewise means $\mathbb{T}_\mathbb{S}$ will be a position within $\chi$.

\subsection{Agent Types}\label{subsec:agent-types}
This study explores the idea of using finite voters, as opposed to the
infinite voters described in~\cite{Cohensius2017}.
These voters' positions will be randomly selected using a distribution, which
will vary from experiment to experiment.
By design, there are two primary types of voters: expert and untrained.

\textit{Expert voters} are agents that give a more accurate estimate of the
true value.
These voters have tighter extremes for estimated values or are chosen from
a more accurate distribution (such as Gaussian about $\mathbb{T}$) allowing
them to be more certain in their estimation.
These voters represent better, though likely more expensive, methods for
taking a measurement.  % TODO: such as. . .

\textit{Untrained voters}, on the other hand, are agents with a worse
distribution or more extreme minimums and maximums.
These agents have a greater degree on error, and while they might be correct
about the true value they are unable to be as certain about their estimations
as the expert voters due to their possibility for error.
Measurement methods represented by these voters are less accurate, though
likely cheaper.  % TODO: such as. . .

%  - Finite with agent costs

\section{Experiment Criteria and Parameters}\label{sec:experiment-criteria
-and-parameters}
Both untrained and expert voters have the same goal: to enhance the accuracy of
the system.
They will work in tandem in order to accomplish this goal.
There are two immediately identifiable ways to use both types of voters which
will be explored in this paper.
First, the experts can be used as proxies and the untrained as inactive voters.
This exploits the idea discussed in~\cite{Miller1969, Mueller1972} of
allowing experts to guide the system while still pulling information from the
other agents.

The second obvious way to employ these agents is using the untrained voters
and the proxies and allowing the experts to transfer their voting power to them.
This is the direct inverse of the previous setup.
The intuition behind this strategy is since untrained voters are likely
cheaper and so can be more numerous, they have a better chance of being closer
to the true value due to the law of large numbers.
Naturally, this technique would be most beneficial when untrained agents use a
distribution with a decent likelihood of the true value being included, such
as Gaussian or uniform.
While this study does not plan to exploit the law of large numbers anywhere
near its full potential, this setup may be beneficial where many measurements
can be taken or when untrained agents are considerably less expensive than
experts.

Both setups will be examined in this study.
Fortunately, the process of how to create the setups is identical.
The only difference is which type of agent serves as proxies.

%  - Finite with agent costs

\subsection{Criteria}\label{subsec:criteria}
% - Time (squared?)/Calculation complexity
%   - Useful to know if humans could use it.
% - Accuracy/Precision & Recall
% - Cost

\subsection{Parameters}\label{subsec:parameters}
% - \# of Proxies
% - \# of Non-Proxies
% - Distribution of votes
%     - Uniform
%     - Gaussian
%     - Bimodal about center
%     - Skewed?
% - Dimensions?
% - Cost per proxy
% - Cost per non-proxy
% - Seed
%     - Metaparameter--allows reproduction of an iteration. Each iteration
%       will have a unique seed.
%     - Might be best to not discuss this in the paper.

\section{Mechanisms}\label{sec:mechanisms}
% - Explain what mechanisms are, as well as my expansion using weight mechanisms.

\subsection{Voting Mechanisms}\label{subsec:voting-mechanisms}
% - Explain what voting mechanisms are

\subsubsection{Candidate Mechanisms}\label{subsubsec:candidate-mechanisms}
% - Explain what candidate mechanisms are
%
% - Median
% - Borda
% - Ranked Choice
% - Plurality? Likely not going to work well
% - . . .

\subsubsection{Average Mechanisms}\label{subsubsec:average-mechanisms}
% - Explain what average mechanisms are
%
% - Mean
% - Weight by Borda
% - Weight by Ranked Choice
% - . . .

\subsection{Weight Mechanisms}\label{subsec:weight-mechanisms}
% - Explain what weight mechanisms are
%
% - Vote for closest (implemented by \cite{Cohensius2017})
% - Borda
% - Ranked Choice
% - . . .


\section{Experiment Design}\label{sec:experiment-design}
% TODO: I don't currently have anything in here about agents with different
%  costs. I might not get to that until after running the cost-less
%  experiments, but I ought to look for opportunities to implement it.
%
%  Costs can potentially be added post-experiment, since the outcome of the
%  experiment does not depend on the costs of the agents. This would
%  primarily allow for identifying a cost-ratio and ideal number of agents.
%
%  It might also be interesting to develop a system that uses a budget
%  without a predefined number of proxies and inactives.

The experiments will be performed in two phases.
First, all data will be collected.
This is done primarily to make all desired data available during the analysis
process.
% FIXME: Is there really nothing else I could say here?
Afterwards, the data will be analysed in an attempt to identify any patterns,
strengths, or weaknesses proxy voting has as a measurement verification tool.

In order to simplify various calculations, the true value for each experiment
will be 0.
This makes it trivial to calculate loss, and also simplifies selecting the
positions of each voter.

% TODO: Talk about distribution range restrictions, such as Gaussian being
% restricted between -1 and 1. I may want to restrict both experts and
% the untrained to the same value to simplify things.

\subsection{Data Collection}\label{subsec:data-collection}
% Steps
In order to ensure consistency, each experiment will follow the same general
steps.
% TODO: Fill this out more

\begin{enumerate}[label=\textbf{\arabic*}., leftmargin=2\parindent]
    \titleditem{Generate proxies}
    Given the space in which an experiment is to take place, the system will
    generate a number of proxies using the given distribution.
    The number and distribution of these proxies will depend on the
    parameters of the experiment.

    \titleditem{Generate inactive voters}
    The system will then generate a set of inactive voters.
    Again, the number and distribution these voters will depend on the
    experiment's parameters and will often not be the same as the proxies'
    parameters.
    In particular, the distribution and range used will generally be
    different from the proxies' since the proxies serve as `experts' and so
    should be more accurate.

    \titleditem{Assign weights to proxies}
    Using the given \hyperref[subsec:weight-mechanisms]{weighting mechanism},
    each inactive voter will apply weights to one or more proxies.
    The system will sum these weights to get a total weight for each proxy.

    \titleditem{Estimate truth}
    Once all the weights have been calculated, the system will coalesce
    the weights into an estimated value using the provided
    \hyperref[subsec:voting-mechanisms]{voting mechanism}.
    This value is the system's estimation of the true value.

    \titleditem{Calculate error}
    Finally, the error of the system can be calculated by using the true
    value and the system's estimation of the truth.
    This error can be used to determine how well the system worked and
    ultimately determine if proxy voting is useful under the given parameters.
\end{enumerate}

% FIXME: I might want to specify how many times a combination is ran
Each combination of parameters will be run a number of times in order to obtain
a fair spread of error for that set of parameters.
After each iteration, the combination of parameters will be recorded as well
as the estimated truth, the error, as well as the rest of the
\hyperref[subsec:criteria]{criteria}.

% TODO: Caveats

\subsection{Analysis}\label{subsec:analysis}
% TODO: What is done after an experiment?


\section{Further Work and Improvements}\label{sec:further-work-and-improvements}
% - Multiple levels of expertise. Expert voters could have varying levels of
% accuracy, allowing for more complex systems.
% - (If I decide to restrict experts and untrained to the same mins and maxes)
% Allow for untrained to have a lower restriction, making them less accurate.

