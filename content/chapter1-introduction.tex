%
%  This document contains chapter 1 of the thesis.
%

\chapter{INTRODUCTION}\label{ch:introduction}
%%%%%%%% This line gets rid of page number on first page of text
\thispagestyle{empty}
%%%%%%%%%%%%%

On May 13th, 2020, the 116th United States Congress introduced a resolution
to permit the use of proxy voting during emergencies for members of the House of
Representatives~\cite{Congress.gov2020}.
This resolution was passed on May 15th, 2020, and was quickly put to use, as
indicated by the letters required to designate a proxy~\cite{Clerk.House.gov2020}.
The hope behind the resolution was to allow members of the House of Representatives
to vote remotely, in order to prevent the spread of COVID-19~\cite{Congress.gov2020}.
However, the question remains as to what degree proxy voting affects the results of
the vote.
In this study, we explore the effects of proxy voting on the results of the vote
under a number of opinion distributions.
We additionally explore how different voting rules affect the results of the vote.

% TODO: State we assume fairly ideal conditions where an inactive voter can choose
% their proxy per-topic.
% TODO: Outline
% Discuss my assumptions
%   - Per-topic proxy selection
%   - Perfect knowledge of potential proxy's positions
%       - This might need to be discussed after the readers have an understanding of
%         the model.
%  - No other factors (e.g. party affiliation)
% Discuss opinion model (Cohensius)
% Background
%   - Proxy voting
%       - Why it is desirable
%   - What are voting rules
% Contribution





The ability to correctly take measurements is an important part of human
progress, both in terms of everyday lives and the advancement of science.
Having the ability to take highly precise and accurate measurements is essential
to maintaining safety, ensuring the accuracy of research, and providing a
higher quality of life.
This is particularly critical in high-risk or high-cost scenarios, such as when
human life is involved, or when the cost of using a sensor, taking a
measurement, or running an experiment is particularly expensive.
This paper explores the idea of replacing highly-accurate, though undesirable,
measurement techniques with a system of alternative, less accurate techniques combined
with an adapted form of proxy voting while still maintaining a sufficient level of
accuracy.
Ideally, this system would employ measurements that are cheaper or less risky, making
them a viable replacement for the more accurate but more costly measurements.

The proxy voting system makes use of a model developed by \etal{Cohensius} in their 2017
article~\cite{Cohensius2017}.
This model places voters' preferences in a metric space \systemspace, such as
in~\autoref{fig:system-metric-space}.
Modeling voters' preferences in such a way allows for continuous preferences instead of
discrete or binary opinions other models might emphasize.
Whereas this study deals with the accuracy and errors of a set of measurements, whose
error also exists in continuous space, such a preference space is useful for this
paper's purposes.
A measurement can be taken, and the resulting value will be its preference.

\com{ We need to rethink the assumptions about the metric space. Consider
multidimensional spaces. Consider when there is no clear ordering. Consider when
there are only discrete choices. If we are thinking about votes that take place in
the House, considering an example from that case would be helpful. Another thought is
that sometimes the choices change once you are actually in a meeting, through
amendment to the original proposal. For example, they are voting on tuition
reimbursement and someone amends the bill to include funding for veterans. How
unhappy is the inactive voter with his choice of proxy when the issues change?   }

\begin{figure}[htbp]
    \centering
    % Built using:
    % https://tex.stackexchange.com/a/148253/277236
    % https://tex.stackexchange.com/a/380491/277236
    \begin{tikzpicture}[scale=1.5]
        \draw(0,0) -- (10,0) ; % Axis
        \foreach \x in  {0, 10} % Vertical higher lines
        \draw[shift={(\x,0)},color=black] (0pt,5pt) -- (0pt,0pt);
        \foreach \x in {0, 5, 10} % Numbers and lower lines
        \draw[shift={(\x,0)},color=black] (0pt,0pt) -- (0pt,-5pt) node[below]{$\x$};

        % Labeled points
        \tkzDefPoint((-4/7 + 1) * 5, 0){agentA}
        \tkzDefPoint((3/4 + 1) * 5, 0){agentB}
        \tkzDefPoint((1/12 + 1) * 5, 0){agentC}
        \tkzLabelPoint[above](agentA){$\truthof{a}$}
        \tkzLabelPoint[above](agentB){$\truthof{b}$}
        \tkzLabelPoint[above](agentC){$\truthof{c}$}

        \foreach \n in {agentA, agentB, agentC}
        \node at (\n)[circle,fill,inner sep=1.75pt]{};
    \end{tikzpicture}

    \caption{Example of a preference metric space, where \truthof{x} represents the
    preference of agent $x$.}
    \label{fig:system-metric-space}
\end{figure}

This model was used in~\cite{Cohensius2017} with an infinite population of voters.
However, since it would be costly, not to mention realistically impossible, to take
infinite measurements, this paper will use the model with a finite population.
This, in conjunction with proxy voting as described in \autoref{sec:background}, form
the core of the system.

% Explain what Cohensius did
%   - Strategic participation
%   - Infinite voters
%   - Voting Mechanisms
% Explain what I'm borrowing and what I'm doing differently
%   - Compulsory use of proxies
%   - Finite voters
%   - Additional voting mechanisms
%   - Weighting mechanisms

\etal{Cohensius} also introduce the concept of \textit{voting mechanisms} or
\textit{voting rules}, which are functions that map a set of preferences
in~\systemspace\ to an outcome that also exists in~\systemspace.
%
\com{The idea of a voting mechanism did NOT start with Cohensius. Give credit to the
proper originator.}
%
This study employs the use of two of their mechanisms, the Mean and Median
mechanisms, as well as introduces additional mechanisms.
Additionally, this study introduces the idea of \textit{weighting mechanisms}, which
determines how agents apply weights to proxies in the proxy vote system.
%
\com{Being able to divide your vote between multiple proxies would be interesting,
    but it could be that you end up hurting your most preferred candidate. Suppose
    the vote is between A and B. You like A slightly more. Suppose you give 60\% of
    your vote to candidate A and 40\% to B. If someone else gave 100\% to B, B wins.
    If you would have given 100\% to A, it would have been a tie.}

While \etal{Cohensius} primarily examine voting mechanisms under random and strategic
participation in a proxy vote system, this study will preselect proxies and require
all other agents to use them.
The reasoning behind this compulsory proxy system is to allow a system designer to
choose a suitable set of measurement techniques as the proxies, and reinforce their
measurements using the other agents.
%
\com{Doesn't make sense for the House case  Better bit to have this restriction.}

Using the concepts discussed above, the system proposed in this study
operates as follows:
A measurement is taken, and the resulting value is subsequently represented as the
measurement's preference.
After taking several measurements, either simultaneously or sequentially, weights will
be applied using a weighting mechanism to a preselected set of measurements which
will then be aggregated with any desired voting mechanism.
The resulting value will be the system's estimation of the actual value.
This system would ideally be implemented as a software layer in a program that
controls sensors used to take measurements, but one could use a simple system employing
simple voting and weighting mechanisms with pen and paper if they are willing to take
the required measurements by hand.


This paper will explore the potential circumstances under which this system will be
beneficial, as well as outline what types of measurement techniques should be proxies
and which mechanisms to use.
In order to perform this analysis, a known truth value \truth\ will be selected.  \com{computed?}
The system proposed above will then estimate the known truth using
values pulled from probability distributions to represent values obtained from
performing real measurements.
The system's estimated truth \systemtruth\ will then be compared to the actual truth
in order to determine how well the system worked.

% \vicki{ In your introduction you need to introduce the problem you are trying to
% solve. I wouldn't jump to contributions immediately.}


\section{Background}\label{sec:background}
\textit{Proxy voting} involves allowing agents to transfer their voting power
from themselves to another agent, known as a \textit{proxy}\cite[para.~1.4]
{Cohensius2017}  \com{No need to give paragraph.  You are mainly giving credit for
the idea.  You can't give Cohesius credit for the idea of proxy voting.}
This adds to the voting power of the proxy, which is called its `weight.'
By doing so, the transferring agent loses their ability to directly affect the
voting game, instead allowing the proxy to affect it on their behalf.
In other words, the transferring agent does not vote in an election but rather
affects the election vicariously through the proxy to which they transferred
their power.
These transferring agents are also known as \textit{inactive voters}, and the
proxies can also be called \textit{active voters}.
In contrast, \textit{direct voting} requires each agent to vote on their own.

Under normal circumstances, proxy voting has a number of advantages over direct voting.
The most obvious advantage is that it allows inactive voters to reduce the work
required of them by choosing a proxy to perform the work of voting for them.
Naturally, this is extremely useful when the cost of voting is high, such as
when remaining educated on a topic is difficult or
time-consuming~\cite[para.~1.1]{Mueller1972}, uncomfortable work such as
standing in long queues is required, or other frustrating or costly obstacles
must be overcome.  \com{Pick an application area and discuss the advantages of proxy voting for them.  The cost of remaining educated indicates the person doesn't have a researched opinion.  That is quite different from not being able to vote because of inconvenience.  In many cases, we want the voters to be informed.  If they aren't, there are reasons to discount their vote.}

In these cases, proxy voting allows voters to skip incurring the costs while
still having their voice heard by allowing a proxy to perform the work once on
behalf of all voters who transfer their power to that proxy.
Of course, a voter would not want to give its voting power to just any proxy.
This study employs a similar technique as used in~\cite{Cohensius2017}, which has the
voters, or agents, pass their voting power to the proxy that is closest in the
preference space.
In addition to only transferring power to the closest, which here is called the
`closest' weighting mechanism, this study examines additional ways to split weight
among several or all proxies.
% \vicki{  But if I'm turning my vote to a proxy, how is my vote heard? You could say
% you impacted the vote, but not that your vote was heard. I don't know how you can say
% the accuracy is increased, unless you know the proxy is smarter than you.
% How do the voters find proxies and how do they decide who they trust? The reader is
% unclear about the setup.}

Additionally, measurements will have unfortunately already incurred their cost, since in
order for them to have a preference they need to be performed.
As such, this paper will not be able to take full advantage of relieving individual
agents of some of their cost.
However, by using less accurate measurement techniques with proxy voting, the hope
is to increase the overall system accuracy without incurring the cost of using more
accurate measurement techniques.

In addition to reducing individual agents' work, proxy voting also often has
the effect of increasing the system welfare by increasing the accuracy of the
system~\cite[sec.~1.1]{Cohensius2017}.
\com{Briefly explain in a phrase or two how it increases accuracy.}
However, this increase does not work in all circumstances and so proxy voting
should be used after investigating the voting mechanism used and the potential
pitfalls of that mechanism under proxy voting.

In some cases, the proxy is also able to transfer their own vote, as well as the rest of
their weight, to yet another proxy   \com{What do you mean by voting weight?}
% \vicki{You haven't told us about voting weight. If I can transfer my vote, why not
% weight? }
This is known as \textit{liquid democracy}, which has its own challenges and
advantages.
However, simple proxy voting, where proxies cannot transfer their voting power,
is sufficient for the research performed in this paper.

Proxy voting and liquid democracy are well-discussed topics with a number of
papers investigating their viability, advantages, and disadvantages.  \com{Cite those papers here.}
Many of these studies focus on political and societal uses for these techniques,
but some of their discoveries can be used for this paper's purposes.

James Miller~\cite{Miller1969} explored the idea of proxy voting as a more
direct method of democracy over representative democracy.
Most of his work focuses on the political and societal advantages of proxy
voting, and so is not directly applicable to the problem at hand.
However, he does bring up the idea this paper calls \textit{expert proxies},
those being individuals who would `vote as [the inactive voter] would if only
[the inactive voter] had the time and knowledge to participate
directly'~\cite[para.~1.3]{Miller1969}.
This concept is exactly what this paper plans to exploit: create and use expert
proxies when generating a measurement and allow less accurate techniques to use
them as proxies.

\etal{Mueller}~\cite{Mueller1972} reaffirms the need for expert proxies and
goes on to discuss how many of such proxies would be needed under random
selection for a sufficiently accurate opinion to be made.
These estimates range from 500 to 1000~\cite[para.~3.2]{Mueller1972}, and
while this study uses preassigned proxies, these proxies will have random
error in their estimations and so \etal{Mueller}'s estimates may come into
play.
However, a primary goal of this paper is to use far fewer measurements than the
quoted 500, as that many measurements would likely be far more expensive than
simply using a single, more costly form of measurement with better accuracy.

However, proxy voting is not without its flaws.
\etal{Gölz} investigated a weakness in liquid democracy dubbed `super voters'
~\cite[para.~1.3]{Golz2021}, which are proxies that receive an extremely large
amount of power.
This can be problematic because the result of a system will be too biased
towards the super voter.  \com{Explain why the super is a problem. It seems part of the proxy philosophy.  
If many voters use proxies,  you aren't going to see the richness of preferences. This shouldn't have been a surprise.}

For this paper's purposes, a poor distribution of measurements may make one
proxy a super voter and increase the error of the system.
% \vicki{here it appears you know absolute truth, and just want to assure the voting
% finds it.}


\section{Contribution}\label{sec:contribution}
This study aims to explore the idea of using proxy voting as a technique to remove the
need to take expensive, dangerous, or otherwise undesirable but more accurate
measurements by employing more desirable but less accurate gauging methods when they
are available.
This is accomplished by first taking a number of measurements to serve as proxies, then
supplementing any lack of precision by allocating votes from other measurement
techniques.
\com{This makes total sense to me, in not employing sensors if they appear not to be
needed.
If all the cheap sensors agree, is there really any point in employing an expensive
sensors.
But is this really proxy voting?  It seems more like "not voting", as the expensive
sensor has no opinion.}
Additionally, this paper experiments with using different voting mechanisms and
weighting mechanisms in an attempt to optimize proxy voting for different
distributions of error.
This paper also determines if using such mechanisms are even worthwhile, or if it
would be better to skip the proxy voting and simply allow each measurement contribute
to the truth directly.
Finally, this paper attempts to identify how many `proxy' measurements need
to be taken to reduce cost while still ensuring accuracy.

This paper will show that, while proxy vote systems are not a perfect tool to
increase a system's accuracy, they may be beneficial when the distribution of error
from a measurement is asymmetrical.
Additionally, this paper will identify the best performing voting and weighting
mechanisms, as well as discuss an ideal range of proxy and inactive agents to be used
in such a system.


\section{Potential Applications}\label{sec:potential-applications}
Proxy vote systems can be used in situations where highly accurate measurements are
difficult, costly, or time-consuming, and cheaper or easier alternatives are available.
While this study will show they will not work in all situations, they do have a
potential use when one of the alternative measurements tends to yield an asymmetrical
distribution of error.

These systems may also have use in ensemble machine learning techniques, though this
idea is not explored in this study.
