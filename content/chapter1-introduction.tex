%
%  This document contains chapter 1 of the thesis.
%

\chapter{INTRODUCTION}\label{ch:introduction}
%%%%%%%% This line gets rid of page number on first page of text
\thispagestyle{empty}
%%%%%%%%%%%%%
On May 13th, 2020, the 116th United States Congress introduced a resolution
to permit the use of proxy voting during emergencies for members of the House of
Representatives~\cite{Congress.gov2020}, which would allow members to participate in
proceedings remotely while still having someone physically present to submit the
member's vote on their behalf.
This resolution was passed on May 15th, 2020, and as indicated by proxy designation
letters sent to the Clerk of the House of Representatives~\cite{Clerk.House.gov2022},
was put to use as soon as five days later.
The purpose behind the resolution was to allow the House of Representatives
to continue operating while still allowing members to quarantine and prevent the
spread of COVID-19~\cite{Congress.gov2020}, as well as to pave the way for future use
of proxy voting in other such emergencies.
The type of proxy voting in Congress is not the traditional type of proxy vote.
Instead, it is more similar to voting remotely, as quarantining members need to
meet via video conference and the delegated proxy is required vote exactly as
directed on behalf of their delegators.
Nevertheless, using proxy voting in such an important part of the legislative process
reignites the discussion on the effectiveness of traditional proxy voting where the
proxy is able to vote on behalf of their delegators as the proxy desires.
The proxy being able to vote as they desire simplifies the voting process, but
creates error in that information about the delegators' preferences is lost.
Does proxy voting introduce too much error into the voting process?
Or is it a viable alternative to direct voting?
If proxy voting does introduce error, perhaps alternative voting mechanisms or systems
can be used to mitigate that effect while still maintaining the benefits of proxy
voting.
This study attempts to answer these questions by analyzing the effects of proxy
voting on the results of a vote under a number of preference distributions.
We employ several voting mechanisms and rank them according to their performance when
applied under several preference distributions.
We additionally explore the use of alternative proxy voting systems, such as
allowing an individual to select a preference dependent on the positions of their
constituents, and determine if they are more effective than the traditional proxy
voting system where a single proxy is chosen.
Finally, as a part of this study we implement and describe voting over a continuous
interval instead of the typical binary (in favor or against) vote.

This topic is of large importance because of the potential ramifications of using
proxy voting directly in the legislative process.
If proxy voting does not have a large effect on the results of the vote, then it
can be used to significantly ease the process and lower the risk of participating in
Congress during pandemics or other emergencies.
Similarly, if proxy voting can be shown to have a large effect on the results of the
vote, it would be important to alter the system to mitigate the effects or stop using
proxy voting in the House altogether.
In either case, it is important to understand the effects of proxy voting in order to
utilize it in the most effective way possible, as well as discover any potential
techniques to improve such a system.


\section{Background}\label{sec:background}
In order to understand the analysis performed in this study, we must first understand
what proxy voting is and how it works.

In a given voting scenario, each agent starts with one vote.
This vote can be called the voter's \textit{weight} or \textit{voting power}, meaning
each agent starts with a weight of one.
The more weight a voter has, the more they can swing the vote in their favor.
\textit{Proxy voting} involves permitting agents to transfer their voting power
from themselves to another agent, known as a \textit{proxy} or a \textit{delegate}.
These transferring agents are also known as \textit{inactive voters},
\textit{delegators} or \textit{delegating agents}, and the proxies can also be called
\textit{active voters}.
Transferring voting power adds to the voting power of the proxy, increasing its
weight by the amount of power transferred to it.
By delegating a proxy, the transferring agent loses their ability to directly vote,
instead allowing the proxy to affect it on their behalf.
In other words, the transferring agent does not vote in an election or other types of
votes, but rather affects the election vicariously through the proxy to which they
transferred their power.
This allows the delegating agent to skip incurring any costs voting without a proxy
may cause.

In contrast, \textit{direct voting} requires each agent to vote on their own, meaning
each agent must incur the costs of voting.
As such, proxy voting has a number of advantages over direct voting.
The most obvious advantage is that it permits delegators to reduce the work and costs
required to participate in a vote by choosing a proxy to perform the work of voting
for them.
Naturally, this is extremely useful when the cost of voting is high, such as
when one is ill or fears they may become ill, or when a member's time can be spent
more efficiently without being physically present for a vote like when attending a
conference or engaging in active research in some location outside {Washington,~D.C}.
Costs such as these are common in voting scenarios, and are further discussed
in~\cite{Gershtein2019}.
Not being required to be present while still being able to participate is, of course,
directly advantageous for members of the House of Representatives, since proxy voting
would allow them to prevent the spread of disease or avoid whatever other emergency
triggered the use of proxy voting.

In these cases, proxy voting allows voters to skip incurring the costs while still
having their voice heard by allowing a proxy to pay the cost once on behalf of all
voters who transfer their power to that proxy.
However, this advantage comes with the disadvantage that the proxies are able to vote
as they themselves see fit, meaning the delegator's weight may not be applied exactly
how they want.
This means a voter would not want to give its voting power to just any proxy, but
rather one that would still be `close enough' to their preference.
There is room for strategy in that an agent could delegate their vote to someone
further away in order to bring the results closer to their opinion.\footnote{
    For example, if an agent that leans only slightly left knows the majority was going
    to vote far left in the preference space, the agent could delegate to someone who
    leans towards the right.
    This delegation would go against the agents desires, but would pull the result of
    the vote more towards the agent's true preference, thereby yielding higher
    utility for that agent.
}
These questions on strategy in proxy voting are important, but in order to first
determine if proxy voting would even be useful, this study employs a similar technique
as used in~\cite{Cohensius2017}, which has the voters pass their voting power to the
proxy that is closest in the preference space without any other strategic reasoning.

The `goodness' of a system here is measured using two primary metrics: 1) the
\textit{error} of the system, meaning how close the system is to the optimal
solution, and 2) the total system \textit{welfare}.
Welfare is a measure of how good, or bad, a situation or action is for an agent.
For example, an agent not being required to vote in person increases that agent's
welfare since it makes it easier for the agent to vote.
Conversely, if the agent is required to incur all the costs of voting, such as when
using direct voting, the agent's welfare is lower due to the increased work the agent
needs to do.
Total system welfare is the sum of the welfare of all agents in the system.
An excellent system would have high accuracy and high welfare, while a poor system
would have low accuracy and low welfare.
In regard to proxy voting, the excellent system would maximize the use of proxies
without degrading the accuracy of the system, while the poor would require all
proxies to vote directly and still have high error.

Proxy voting has been shown to often increase system accuracy when compared to only
allowing active/willing agents to vote~\cite{Cohensius2017}.
This follows intuition: if a voter doesn't vote, the system loses information.
By allowing them to still influence the voting game through proxy, some information
is reintroduced into the system.
Naturally, in terms of system accuracy the ideal situation is when all voters
participate and so the system has the most information possible.
However, system welfare can be enhanced on an individual level by allowing agents
who want to be inactive through the use of a proxy to delegate their vote.
Therefore, when operating under the Congress resolution, system welfare can be
enhanced through proxy voting by allowing ill members to quarantine and ensure the
health of other agents.
As will be discussed in the results section of the thesis,
% \autoref{ch:results},  % TODO: Uncomment for thesis
this additional system welfare comes at the cost of some system accuracy in terms of
the actual preferences of the voters.
% \vicki{Again, we need to understand the rules
% of proxy voting. Is it that the active voter must vote the same way for all of the
% delegated votes? Is it that the active voter must vote on amendments to the vote,
% which cannot be known in advance?}
% MDH: This is discussed in the second and third paragraph of the Assumptions section.
%   In this section, I'm trying to give a primer on proxy voting and the generals of
%   how it works. I've added a quick blurb at the beginning of this section to
%   clarify on the goals of this section.
This cost comes about because proxies will likely not vote exactly the same as their
delegators due to having different preferences.
This causes information about the delegators' preference to be lost and the system's
accuracy to suffer.
The loss in accuracy may be acceptable, however, if the system's welfare is sufficiently
improved.
Perhaps more importantly, proxy voting can also increase system accuracy when a
sufficient number of agents are unable to vote in person, since it allows the system
to recover some lost information by allowing agents to delegate their vote to a proxy
with a similar preference.

% TODO: Move to Future Work section once I have one
% In some cases, the proxy is also able to transfer their own vote, as well as the rest of
% their delegated weight, to yet another proxy.
% This is known as \textit{liquid democracy}, which has its own challenges and advantages.
% Ultimately, the use of liquid democracy is not considered in this study, but since proxy
% voting is a subset of the liquid democracy system, it would likely be possible to
% extend the results of this study to liquid democracy as well.  \vicki{This could be
% described in a future work section, but seems to have no purpose here.}

There are many ways to represent the votes, or preferences, of agents in a system.
Once such way is a continuous space model, such as the one used in~\cite{Cohensius2017}.
This model places voters' preferences in a metric space \systemspace, such as
in~\autoref{fig:system-metric-space}.
In this model, two points that are close together in the metric space represent
similar preferences, while two points that are far apart represent very different
preferences.

\begin{figure}[htbp]
    \centering
    % Built using:
% https://tex.stackexchange.com/a/148253/277236
% https://tex.stackexchange.com/a/380491/277236
\begin{tikzpicture}[scale=7.0]
    \draw(-1,0) -- (1,0) ; % Axis
    \foreach \x in {-1, 0, 1} % Numbers and lower lines
    \draw[shift={(\x,0)},color=black] (0pt,2pt) -- (0pt,0pt);
    \foreach \x in {-1, 0, 1} % Numbers and lower lines
    \draw[shift={(\x,0)},color=black] (0pt,0pt) -- (0pt,-2pt) node[below]{$\x$};

    % Labeled points
    \tkzDefPoint((-4/7), 0){agentA}
    \tkzDefPoint((3/4) , 0){agentB}
    \tkzDefPoint((1/12), 0){agentC}
    \tkzLabelPoint[above](agentA){$\truthof{a}$}
    \tkzLabelPoint[above](agentB){$\truthof{b}$}
    \tkzLabelPoint[above](agentC){$\truthof{c}$}

    \foreach \n in {agentA, agentB, agentC}
    \node at (\n)[circle,fill,inner sep=1.75pt]{};
\end{tikzpicture}
    \caption{
        Example of a 1D preference metric space, where \truthof{x} represents the
        preference of agent $x$.
        The x-axis represents some preference space, where the leftmost point is
        the most preference most against some idea and the rightmost point is the most
        preference most in favor of the same idea.
        Importantly, points towards the center of the space are more ambivalent about
        the topic or prefer a more moderate approach than the extremes on either end.
    }
    \label{fig:system-metric-space}
\end{figure}

Votes are aggregated into an output using a \textit{voting mechanisms} or
\textit{voting rules}.
Two common voting mechanisms are the \textit{plurality} and \textit{mean} voting rules.
The mean mechanism naturally extends to a continuous space, since it can take all
preferences, multiply them by their weights, and then average them.
Plurality, on the other hand, needs to be adapted to operate in a continuous space.
In our implementation, we will treat plurality and any other `candidate` mechanisms
as if the proxies themselves were candidates.
This is similar to the framework in~\cite{Bulteau2021} treating each preference as a
proposal.
As such, the plurality mechanism will select the preference of the active voter with
the most weight.

Proxy voting may seem like an extremely attractive option for the House of
Representatives, and other organizations, to use.
However, it is not without its flaws.
As a simple example, consider a vote taking over preference
space $\systemspace = [-1, 1]$ with agent preferences $\truthof{\agent_1} = -1$,
$\truthof{\agent_2} = 0.25$, $\truthof{\agent_3} = 0.5$, $\truthof{\agent_4} = 1$,
$\truthof{\agent_5} = 0.15$.
Under the mean mechanism, which aggregates opinions by simply taking their mean, if
everyone were to vote we would get the actual preference of the
system: $\systemtruth = 0.18$.
However, if $\agent_2$, $\agent_3$, and $\agent_5$ were to become inactive and select
$\agent_4$ as their proxy, the result would be $\systemtruth = 0.6$.
This is visualized in \autoref{fig:voting-example}.
This is an absolute error of 0.42, or 21\% of the entire preference space!
Ideally the output of a system employing proxy voting would be much closer to the
actual preference of the system, but since the center voters do not have a more
central proxy, the system's output is skewed towards the most extreme voters.

\begin{figure}[htbp]
    \centering
    % Built using:
% https://tex.stackexchange.com/a/148253/277236
% https://tex.stackexchange.com/a/380491/277236
\begin{tikzpicture}[scale=7.0]
    \draw(-1,0) -- (1,0) ; % Axis
    \foreach \x in {-1, 0, 1} % Numbers and lower lines
    \draw[shift={(\x,0)},color=black] (0pt,0pt) -- (0pt,-2pt) node[below]{$\x$};

    % Labeled points
    \tkzDefPoint(-1, 0){agent1}
    \tkzDefPoint(0.25, 0){agent2}
    \tkzDefPoint(0.5, 0){agent3}
    \tkzDefPoint(1, 0){agent4}
    \tkzDefPoint(0.15, 0){agent5}
    \tkzLabelPoint[above](agent1){$\truthof{\agent_1}$}
    \tkzLabelPoint[below](agent2){$\truthof{\agent_2}$}
    \tkzLabelPoint[above](agent3){$\truthof{\agent_3}$}
    \tkzLabelPoint[above](agent4){$\truthof{\agent_4}$}
    \tkzLabelPoint[above](agent5){$\truthof{\agent_5}$}

    \foreach \n in {agent1, agent2, agent3, agent4, agent5}
    \node at (\n)[circle,fill,inner sep=1.75pt]{};


    % Actual preference
    \draw[color=blue, line width=0.5mm, dotted]
    (0.18, 0pt) -- (0.18, -3pt);
    \node[color=blue] at (0.18,-4pt) {$\systemtruth_{actual}$};

    % Preference under proxy vote
    \draw[color=orange, line width=0.5mm, dotted]
    (0.6, 0pt) -- (0.6, -3pt);
    \node[color=orange] at (0.6,-4pt) {$\systemtruth_{proxy}$};
\end{tikzpicture}

    \caption{
        An example vote and its results.
        $\textcolor{blue}{\systemtruth_{actual}}$ is the result when everyone votes,
        and $\textcolor{orange}{\systemtruth_{proxy}}$ is when $\agent_2$, $\agent_3$,
        and $\agent_5$ delegate their vote and make $\agent_4$ a super voter.
    }
    \label{fig:voting-example}
\end{figure}

Previous research has also identified problems with proxy voting.
\etal{Kling} and \etal{Gölz} all investigated a weakness in liquid democracy, a
superset to proxy voting, dubbed `super voters'~\cite{Kling2015,Golz2021}, which are
proxies that receive an extremely large amount of power, while others gain very little.
While \etal{Kling} ultimately determined these proxies tend to use their power
wisely, possibly to avoid estranging those voters who delegate their power to the
proxy, there can be situations where super voters can be problematic with one-off
issues.
For example, in the previous situation $\agent_4$ could be considered a super voter.
If they were to change their preference, say after bribery, threat, or even
something benign such as changing their opinion after a debate, the system's output
could change drastically.
While there are methods to help mitigate the effect of super voters, which
\etal{Gölz}~\cite{Golz2021} explore, the amount of error produced by a proxy vote
system, including that of super voters, is precisely what this paper will explore.


\section{Preliminary Setup and Assumptions}\label{sec:setup-and-assumptions}
An important part of any study is the model it uses to represent the system being
studied.
In this work, we employ a model described by \etal{Cohensius} in their 2017
article~\cite{Cohensius2017}.
This model places voters' preferences, which for our purposes are the Congress
members' opinions on a topic, in a continuous metric~space~\systemspace, as
previously described in~\autoref{sec:background}.
Modeling voters' preferences in such a way allows for continuous preferences instead of
discrete or binary opinions other models might emphasize.
This is beneficial, as it allows for a more realistic representation of voters'
opinions, since it is unlikely that voters' opinions are perfectly binary.
For example, it is likely that some members are passionately in favor of higher
spending on education, while others are passionately against it, and yet others
remain ambivalent or less-passionately for or against.
This works perfectly with the metric space model, as it allows for the passionate
opinions to be at opposite extremes of the metric space, while the less-passionate or
more moderate opinions are closer to the center.

These types of continuous spaces are occasionally used in other studies.
\etal{Bulteau}~\cite{Bulteau2021}, for example, develop and experiment with a framework
for aggregating such preferences with several separate mechanisms.
Similar to this study, Zhang~and~Grossi~\cite{Zhang2022} also employ a continuous
proxy system with weighted proxies, treating weight as a probability instead of a count.

This model also naturally extends to two- or higher-dimensional spaces.
These would include more complex and multi-faceted topics, such as migration, which
would include subtopics such as border restrictions and economic impact, or where to
delegate funds in the United States Budget.
However, as previously mentioned, we will only consider the idealistic situation
where a congress member can choose their proxy for each topic individually.
The model obviously allows for votes on topics that are continuous, such as setting the
budget for the year: instead of simply voting yes or no on some pre-chosen dollar
amount for a budget, the output of the system can serve as the budget amount.
However, discrete and binary issues can be supported too, and there are multiple
ways of interpreting the output of the model for such issues.
For example, one could round the output to the nearest valid choice, such as a 1 or a
0 for binary issues.

Such a model also permits an additional aspect of this study: the ability to vote in
a range instead of aforementioned binary or discrete votes.
The hope behind such a system is to gain a better understanding of the voters'
true preference, which is not possible with binary or discrete votes due to all votes
ultimately being binned into one of several values.
This system will allow one to see if the majority is truly in favor of some idea, or
if they are only slightly in favor of it.
For example, consider a situation where voters are asked to vote on some new law and
are able to vote in the interval of $[-1, 1]$.
If the majority of votes are hovering around the center, say between in the interval
$[-0.25, 0.25]$, then it is likely that the majority of voters are not actually fully
satisfied by the law, but still believe some change is necessary.
This provides a third option: refactoring the law to be more in line with the
voters' opinions.
Such an option is extremely beneficial, since without it voters are encouraged to
vote in the extremes.
This is because anything less than an all-for or all-against vote would be diluted by
those who did vote in the extremes, making the voice of the centrist voters less potent.
With the refactoring option, however, if the population is truly ambivalent to the
topic they would be encouraged to vote towards the center in order to show they want
to rework the law.
This reopens discussion and allows for a more educative process, with the hope that
as more discussion occurs the population will become more informed and result in a
better law.

This study also makes use of \textit{voting mechanisms} or \textit{voting rules},
which are functions that map a set of preferences in~\systemspace\ to an outcome that
also exists in~\systemspace.
These mechanisms are split into two groups, Average Mechanisms and Candidate Mechanisms.

Average mechanisms involve taking the weight of voters and aggregating them into some
number by averaging.
These mechanisms include
\begin{itemize}
    \item Mean
    \item {
        Weight by Instant Runoff, where the weight of an active voter is
        determined by its ranking after Instant Runoff using the number of voters
        using it as a proxy.
        Here, Instant Runoff involves eliminating active voters with the least weight
        until some active voter holds half of the total system weight.
        Once this occurs, a weighted mean is calculated using the remaining active
        voters.
        For this mechanism, total system weight is the number of delegators in the
        system, and an active voter's weight is the number of delegators using it as
        a proxy.
        \vicki{This is a strange definition of instant runoff. Normally, it is an
        option that is removed, not a voter. Ballots are initially counted for each
        elector's top choice, losing candidates are eliminated, and ballots for
        losing candidates are redistributed until one candidate is the top remaining
        choice of a majority of the voters. When the field is reduced to two, it has
        become an "instant runoff" that allows a comparison of the top two candidates
        head-to-head.
        %
        Where are you getting this definition? What motivates it? I don't see any
        references where you give credit for this definition.}
    }
    \item {
        Weight by Ranked Choice, where the system-assigned weight of an active voter is
        determined by its overall ranking against all other voters.
        In other words, the system ranks each active voter according to its total
        delegator-assigned weight.
        Each active voter is then assigned a new weight from 1 to $n$ according to its
        ranking, where $n$ is the number of active voters.
        This means the active voter in first will receive a system-assigned weight of
        $n$, regardless of its total delegator weight, and the active voter in last will
        receive a system-assigned weight of 1.
        \vicki{Again, this isn't the typical definition of ranked choice. Normally it
        is the candidates that are listed in order (for each voter).}
    }
    \item Weight by Weighted Instant Runoff, where the system weight of an active
    voter is determined by its ranking after Instant Runoff using the voter weights
    instead of the number of voters using it as a proxy.
\end{itemize}
All of these mechanisms determine the weight of an active voter, then apply a
weighted mean to the voters' preferences to determine the outcome.
As an example, we'll use the Weight by Ranked Choice mechanism.
Say we have six total voters, and three active voters $a, b, c$.
This leaves three inactive voters who can choose $a$, $b$, or $c$ as a proxy.
Two inactive voters select $b$ as a proxy, and one selects $a$.
This means the system ranking for these active voters would be $b = 1, a = 2, c = 3$.
Since there are 3 active voters, $b$ receives a system-assigned weight of 3, a
receives a weight of 2, and c receives a weight of 1.
From here, the system would calculate the weighted mean of the preferences of $a$,
$b$, and $c$ using the system-assigned weights.

Candidate mechanisms, on the other hand, involve selecting a single voter out of all
active voters and using its preference as the outcome.
They treat each active voter's preference as a potential candidate, similar to how
each preference is treated as a proposal in~\cite{Bulteau2021}.
Ideally, the voter selected would be the proxy that best represents the voters' opinions
and would minimize the total distance between its preference and all other voters'
preferences.
These mechanisms would be most beneficial when a continuous option for a vote does
not exist in the real world, such as yea or nay votes, or choosing which toppings
should be ordered on a pizza.
\begin{itemize}
    \item Median
    \item Instant Runoff, which operates the same as the Average Instant Runoff.
    \item Plurality
    \item Weighted Instant Runoff, which operates the same as the Average Instant
    Runoff.
\end{itemize}
As with Average mechanisms, all these mechanisms work by first determining the weight
of a proxy.
However, instead of then applying a weighted mean, they select the proxy that is
at the median (in the case of the Median mechanism) or that ranks highest (in the case
of the remaining mechanisms).

More details on how each voting mechanism operates will be described in the full thesis.
% in~\autoref{subsec:voting-mechanisms}.  % TODO: Uncomment for thesis

The flexibility of the continuous model in both the discrete and continuous realms, its
ability to use different voting rules, and its easy interpretability, are the reasons
it is employed in this study.
This study will focus primarily on the continuous instead of the discrete output of
the model, since observing the continuous output allows for more granularity in the
differences between proxy and non-proxy voting, as well as in the differences between
voting rules.

Additionally, under normal proxy voting each voter can only delegate a single proxy.
This provides a simple system where a voter can simply select a single individual and
allow them to vote on their behalf.
This has the downside, however, that the proxy might not have as close of an opinion
to the voter as they would like.
This raises the question, what if we allowed a voter to delegate multiple proxies?
They could, for example, delegate their vote to a trusted group, who would then
aggregate their opinions into some preference in the aforementioned model on behalf of
the delegator.
This may reduce the error introduced by using proxies and allow the agent to feel
more secure that their voting power will be used as they desire.
This is not an entirely novel idea, with other authors already creating models to
handle such an approach~\cite{Degrave2014,Colley2021,Golz2021}.

\subsection{Assumptions}\label{subsec:assumptions}
In order to conduct this study, we make a number of assumptions.
First, we assume that each voter is able to choose their proxy, also known as a
delegate, individually for each topic.
This is not currently the case for the House of Representatives, which requires
a proxy to be chosen by letter and so would be difficult to change as different
topics are discussed~\cite{Congress.gov2020}.
However, since it would be fairly simple to implement a new process that allows
per-topic proxies, and since many complex topics can be reduced to a set of related
subtopics, we believe this assumption is reasonable.

Additionally, the resolution currently requires the delegate to vote exactly as the
specific instructions provided by the delegating voter tells them to
vote~\cite{CERP2020, Congress.gov2020}.
This essentially turns proxy into a relay for the delegating voter, which is not how
proxy voting is typically used and is not particularly interesting, since relaying a
vote essentially allows the delegator to vote as if they were present.
This means no information about the delegator's preference is lost.
Unfortunately, it also means they do not gain all the benefits of traditional proxy
voting since they still need to do all the work of casting their vote except for
being present.
The `relay' proxy voting also comes with the downside that delegators will be unable
to have their proxy vote differently if the delegator's preference were to change,
say through deliberation or as new information is provided, limiting the system's
adaptability.
% \vicki{I agree we need this definition of proxy, but we need to motivate it better.
% Basically, we are saying we are making this assumption because the way it is really
% done has no problems. It would be better to say, we make this assumption because it
% us more adaptable (like handling ammendments). Finding previous work which
% motivates/justifies our assumptions would be good.}
% MDH: See the following paragraph. Should I mer

Instead, we consider scenarios where actual proxy voting is used, where the proxy
receives the voting power of the delegating voter, increasing their weight, and
proceeds to votes according to their own preference.
Traditional proxy voting also allows the proxy to update their (and by extension, their
delegators') preferences as new information becomes available.
While traditional proxy voting gives substantial flexibility to the proxy to
operate on behalf of their delegators, this flexibility requires the delegating voter
to choose a proxy who they trust to vote as close to how they themselves would.
This is a process similar to selecting experts, as described by~\cite{Miller1969}
and~\cite{Mueller1972}.
By using traditional proxy voting instead of relay-style voting, we hope to exploit
the advantages of proxy voting that the relay-style does not provide.

Second, we assume that each voter has reasonable knowledge of potential proxies'
opinions, meaning they have a decent idea of the preferences of other proxies.
This will allow them to choose the proxy that has the opinion most similar to their own.
While in reality voters will likely not have perfect knowledge of others' opinions,
it is often not particularly difficult to gauge the opinion of others, especially
those with whom an individual often associates, and so we believe this assumption is
reasonable.

Third, we assume that abstention is not allowed.
In other words, all agents must vote either themselves or by proxy.
The primary reasoning behind this is to simplify the system being used and provide
the system with as much information as possible without needing to account for
extenuating circumstances such as the unexpected incapacitation of a voter.
Additionally, all individuals will have some form of opinion, even if that opinion is
completely neutral.
A neutral opinion can be represented as a preference close to the middle of the
preference space, which decreases the desire for abstention when voting is possible.
Nevertheless, as a baseline, we will explore occasions where those who are not
physically present are unable to vote.

Finally, we assume that there are no factors besides closeness in opinion that affect
the choice of proxy.
This differs from the actual system presently used by the House of Representatives,
which includes restrictions such as a proxy can only serve ten voters~\cite{CERP2020}.
However, we feel removing restrictions such as these leads to a more interesting
discussion, since it allows the use of different voting mechanisms and more extreme
cases.


\section{Previous Work}\label{sec:previous-work}
James Miller~\cite{Miller1969} imagined a governmental system utilizing proxy voting
in 1969 as a more direct form of a representative democracy.\footnote{
    That is to say, Miller envisioned a system where individuals could directly vote
    for an issue, or elect a proxy to vote for them.
    Naturally, any democracy that uses proxy voting is a representative democracy,
    since the proxy is representing the delegator.
    Nevertheless, it can be argued that Miller's proposal could provide a more direct
    democracy since a voter can directly vote for an issue is they so choose.
}
His work focuses on reworking the current House and Senate systems entirely by using a
more-directly involved populace, but his ideas can still be relevant under the current
system.
In particular, he introduces the idea we call \textit{expert proxies},
those being individuals who would `vote as [the delegator] would if only
[the delegator] had the time and knowledge to participate directly'~\cite{Miller1969}.
While for the general populace this could be a very valuable benefit of proxy voting,
it is not entirely desirable for the House of Representatives.
One of the reasons individuals are elected to the House of Representatives is to
research and create laws that are in the best interest of the people on behalf of
the people.
Though the past 25 Congresses have seen anywhere from 10 to over 25 thousand issues
over 2 years, only around 10\% are actually discussed~\cite{GovTrack2022}.
That would be approximately 1000 to 2500 issues per Congress, or about 500 to 1250 per
year.
While this is still a large number of issues, it is the job of a member of the House
of Representatives to learn about, research, and deliberate about each issue.
Additionally, Miller states `a representative should be an expert, or at least
competent, in each field [on which they are voting]'~\cite{Miller1969}.
This reinforces the idea that a member of the House of Representatives should be, as
their title would suggest, a representative of the people and have the responsibility
to become an expert in the issues they are voting on.
To remove this responsibility from the House of Representatives would be to remove
a large portion of their duties, and could easily result in the dictatorship of a few.
As such, we differ from Miller in the sense that all voters ought to be experts in
the field, and so we use proxy voting to allow them to be more efficient in their
duties and avoid spreading disease instead of reworking the system entirely.
Additionally, Miller did not consider using proxy voting for use by members of
Congress as it currently works, which we will explore in this paper.

\etal{Cohensius}~\cite{Cohensius2017} explore the use of proxy voting in a metric space
using three voting mechanisms: mean, median, and majority.
They discovered proxy voting using any of these mechanisms generally produces lower
error than direct voting with active voters alone.
This is not too surprising: reintroducing information lost through inactive voters by
using a proxy system ought to help the system.
Nevertheless, they were able to show that proxy voting is effective under a number of
symmetrical and asymmetrical preference distributions, while under both random and
strategic participation.
However, the majority of their research focuses on voting with infinite populations.
While this work would certainly be applicable to larger populations, since a
population of sufficient size will begin to behave like an infinite
population~($\lim_{x \rightarrow \infty} x = \infty$), we are more interested in the
effects of proxy voting on a relatively small, finite population of 435 members of the
House of Representatives.
As such, we will explore the effects of proxy voting on a finite population of this
size, as well as explore other possible voting mechanisms.

Anurita~Mathur~and~Arnab~Bhattacharyya~\cite{Mathur2017} looked at several voting
mechanisms applied on a single-winner election vote and determined a ranking for
these mechanisms.
They apply these mechanisms on a dataset while looking only at datapoints without a
Condorcet winner.
In their work, they say a mechanism `beats' another if it has a larger fraction of
the population prefer its output over the other's output.
They discover that the GT\footnote{
    Presumably meaning `Game Theory.'
}~method~\cite{Rivest2010} beats all others, the Schulze~method~\cite{Schulze2011}
and Minimax voting mechanisms always agree and beat all other mechanisms besides the
GT method, while Borda beats Copeland and Plurality, and Plurality comes in last.
This study will also look at voting mechanisms and attempt to determine which
mechanism is best suited for proxy voting.
Our work will differ significantly from theirs, however, as we will explore voting
mechanisms used in a continuous voting space instead of discrete-space, single-winner
elections.
Additionally, most of our mechanisms will be different due the mechanisms they used
not being beneficial on a continuous preference space or with a small number of
candidates, or not working well with proxy voting.

Jonas Degrave~\cite{Degrave2014} implemented a simple model to allow voters to
delegate to multiple proxies.
He treated proxy delegations as a digraph where nodes are voters and edges represent
delegating proxies.
He developed two algorithms to determine the weights of each proxy.
The first algorithm, which calls for simply dividing their weight equally among all
those to which the voter delegates, is precisely the model we will use.
This technique allows for a straightforward way to delegate voting power that would
not be confusing to voters.
We additionally augment this technique by looking at how many proxies a voter should be
allowed to delegate.
As with Degrave's approach, a delegator's voting power will be divided equally amongst
all its proxies.
Intuitively, if a voter were to delegate all other voters as proxies the result would
be the same as if the voter had simply not voted.
However, if the voter were to only delegate a single proxy, the result might not be
as ideal to the delegator or system as if they had been able to delegate to two proxies
that would produce a better result.
As such, we ask, if voters need to delegate more than one proxy with equal weighting and
will always select those closest to them, how many proxies should be allotted before
more proxies cease to be useful?

% TODO: Shotgun more previous work here
James Fearon~\cite{Fearon1998} described how cooperation often occurs in two phases:
bargaining and enforcement.
The bargaining phase involves negotiation and determining the terms of an arrangement,
while the enforcement phase involves ensuring that the terms of the arrangement are met.
Generally votes take place at the end of the bargaining phase, and the result is the
agreement to be used in the enforcement phase.
We plan to expand voting to feed back into the bargaining phase by allowing voters to
express their dissatisfaction with either side of an issue by voting towards the
center of the interval.
In the case of binary issues, such as yea/nay issues, if enough agents vote towards the
center the bargaining phase can be restarted to allow more negotiation before
performing a second vote, thereby facilitating deliberation.
For more continuous issues, such as how much money to spend on defense, the
continuous model allows agents to vote for moderate spending by voting towards the
center, instead of high or low spending by voting at one of the extremes.


\section{Proposed Work}\label{sec:contribution}
% \section{Contribution}\label{sec:contribution}
This study will explore how proxy voting affects the vote.
We will look at the difference in outcome between a direct vote using all agents that
can vote in person, and proxy voting where the agents that can not attend in person
instead delegate a proxy.
Votes will take place in a continuous metric space, where an agent is able to vote
anywhere inside the interval $[-1, 1]$, and will be aggregated using a voting
mechanism inside the same space.
Direct voting will use the mean mechanism to aggregate votes, since the average of
all preferences will be the point most equidistant from all voters' preferences in
the metric space, thereby minimizing the distance between all voters' preferences and
maximizing system accuracy.

We will examine proxy voting under eight different voting mechanisms, split into two
groups.
Specifically, these mechanisms are\\
\begin{samepage}
    \textit{Averaging Mechanisms:}
    \begin{itemize}
        \item Mean
        \item Weight by Instant Runoff
        \item Weight by Ranked Choice
        \item Weight by Weighted Instant Runoff
    \end{itemize}
    \textit{Candidate Mechanisms:}
    \begin{itemize}
        \item Median
        \item Instant Runoff
        \item Plurality
        \item Weighted Instant Runoff
    \end{itemize}
\end{samepage}
Each of these mechanisms will be compared against direct voting with all agents and
direct voting with only those agents that are present.
This is done to show what the output of the system would be with all information
(direct voting with all agents), as well as the output with minimum information
(direct voting with only those agents that are present).
Error will be measured by squared error between direct voting with all agents and
whichever mechanism is being applied.

We will additionally investigate all voting mechanisms with each delegator delegating
increasing numbers of proxies.
We will start with the base case of one proxy per delegator, then extending to two
proxies per delegator, then three, and so on until more proxies is either detrimental
to system accuracy or not possible.

Whereas voting on an interval is uncommon and finding a real world dataset using
preferences on an interval currently does not seem possible, these investigations
will be performed using preferences generated from several statistical distributions.
These distributions and their notations are listed in \autoref{tab:distributions-used}.
These various distributions will allow the data gathered to represent situations
where the majority of voters are at either extreme (\betadistribution{0.3}{0.3}),
skewed towards one side (\betadistribution{4}{1}), or mostly indifferent
(\gaussiandist\ and \betadistribution{50}{50}).
Each experiment will have a population of 435, representing the number of members in
the House of Representatives.
For each round, we will experiment with 1 to 434 delegating agents and examine how
error correlates with the number of delegators.

\begin{table}[!htbp]
    % increase table row spacing, adjust to taste
    \renewcommand{\arraystretch}{1.3}

    \caption{
        The distributions to be used to generate preferences.
        Note how each distribution represents a unique population type.
        Additionally, any skewed distributions can be inverted to create a
        distribution that is skewed in the other direction (e.g. a distribution
        skewed in favor can be inverted to create a flipped distribution skewed
        against).
    }
    \label{tab:distributions-used}

    \centering
    \begin{tabular}{|c|c|}
        \hline
        Distribution    & Notation                    \\
        \hhline{|=|=|}
        Uniform         & \uniform{-1}{1}            \\
        \hline
        Normal/Gaussian & \gaussian{0}{\sfrac{1}{3}}  \\
        \hline
        Beta(0.3, 0.3)  & \betadistribution{0.3}{0.3} \\
        \hline
        Beta(4, 4)      & \betadistribution{4}{4}     \\
        \hline
        Beta(4, 1)      & \betadistribution{4}{1}     \\
        \hline
\end{tabular}
\end{table}

% TODO: Fill out synopsis of what is learned after the data is analysed
% This paper will show that, while proxy vote systems are not a perfect tool to
% increase a system's accuracy, they may be beneficial when the distribution of error
% from a measurement is asymmetrical.
% Additionally, this paper will identify the best performing voting and weighting
% mechanisms, as well as discuss an ideal range of proxy and inactive agents to be used
% in such a system.
