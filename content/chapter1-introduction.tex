%
%  This document contains chapter 1 of the thesis.
%

\chapter{INTRODUCTION}\label{ch:introduction}
%%%%%%%% This line gets rid of page number on first page of text
\thispagestyle{empty}
%%%%%%%%%%%%%

The ability to correctly take measurements is an important part of human
progress, both in terms of every day lives and the advancement of science.
Having the ability to take highly precise and accurate measurements is essential
to maintaining safety, ensuring research is accurate, and providing a higher
quality of life.
This is particularly critical in high-risk or high-cost scenarios, such as when
human life is involved, or when the cost of using a sensor, taking a
measurement, or running an experiment is particularly expensive.  % Maybe expand on what "expensive" means?
This paper explores the idea of using an adapted form of proxy voting, as
described in~\cite[para.~I.3]{Miller1969}, as a way of taking multiple
measurements to ensure accuracy while minimizing the cost of taking those
measurements.


\section{Contribution}\label{sec:contribution}
% FIXME: This paragraph ought to be reorded to the final order of experiments
This study aims to explore the idea of using proxy voting as a technique to
improve the accuracy of multiple measurements while reducing the cost by
employing less expensive gauging methods when they are available.
This is accomplished by first taking a number of more expensive measurements to
serve as proxies, then supplementing any lack of precision by allocating
votes from cheaper methods of assessing a metric.
Additionally, this paper experiments with using different voting mechanisms and
weighting mechanisms in an attempt to optimize proxy voting for different
distributions of error.
Finally, this paper attempts to identify how many `proxy' measurements need
to be taken to reduce cost while still ensuring accuracy, and attempts to
identify any benefit of using proxy votes as opposed to simply averaging the
results.

% TODO: Provide a brief description of what is discovered here

% TODO: Maybe explain the structure of the paper as well?


\section{Potential Applications}\label{sec:potential-applications}
- % TODO: List potential applications, such as high risk/cost scenarios


\section{Background}\label{sec:background}
\textit{Proxy voting} involves allowing agents to transfer their voting power
from themselves to another agent, known as a \textit{proxy}\cite[para.~1.4]
{Cohensius2017}.
By doing so, the transferring agent looses their ability to directly affect the
voting game.
In other words, the transferring agent does not vote in an election but rather
affects the election vicariously through the proxy to which they transferred
their power.
These transferring agents are also known as \textit{inactive voters}, and the
proxies can also be called \textit{active voters}.

In contrast, \textit{direct voting} requires each agent to vote on their own
(and so incur the costs).
%This ensures each agent has a direct influence on an election % TODO: Fill out this paragraph

In some cases, the proxy is also able to transfer their vote, as well as all
their weight, to yet another proxy.
This is known as \textit{liquid democracy}, which has its own challenges and
advantages.
However, simple proxy voting, where proxies cannot transfer their voting power,
is sufficient for the research performed in this paper.

Proxy voting has a number of advantages over direct voting.
The most obvious advantage is it allows inactive voters to reduce the work
required of them by choosing a proxy to perform the work of voting for them.
Naturally, this is extremely useful when the cost of voting is high, such as
when remaining educated on a topic is difficult or
time-consuming~\cite[para.~1.1]{Mueller1972}, uncomfortable work such as
standing in long queues is required, or other frustrating or costly obstacles
must be overcome.
In these cases, proxy voting allows voters to skip incurring the costs while
still having their voice heard by allowing a proxy to perform the work once on
behalf of all voting power transferred to it.

In addition to reducing individual agents' work, proxy voting also often has
the effect of increasing the system welfare by increasing the accuracy of the
system~\cite[sec.~1.1]{Cohensius2017}.
However, this increase does not work in all circumstances and so proxy voting
should be used after investigating the voting mechanism used and the potential
pitfalls of that mechanism under proxy voting.

Proxy voting also allows for agents uninformed about the issue at hand to
delegate the voting power to a trusted expert.
This is particularly useful for the purposes of this study.
By using less accurate but cheaper measurements, the system can still yield
increased accuracy and confidence in a result without requiring many more
expensive measurements.

However, proxy voting is not without its flaws.
\etal{Gölz} investigated a weakness in liquid democracy dubbed `super voters'
~\cite[para.~1.3]{Golz2021}, which are proxies that receive such a large amount
of power.
This can be problematic because the result of a system will be too biased
towards the super voter.
For this paper's purposes, a poor distribution of measurements will make one
proxy a super voter and increase the error of the result.


%% TODO: I don't like this paragraph, rewrite it
%While use cases of proxy voting are typically focussed on situations that
%are political or sociological in nature, this study aims to discover similar
%advantages in other realms.
%Primarily, the hope is to reduce the impact of measurement error by using
%a combination of `expensive' and `inexpensive' measurements to obtain a
%highly-accurate metric, about which the system can be confident.


\section{Related Work}\label{sec:related-work}
Proxy voting and liquid democracy are a well-discussed topics with a number of
papers investigating their viability, advantages, and disadvantages.
Many of these studies focus on political and societal uses for these techniques,
but some of their discoveries can be used for this paper's purposes.

James Miller~\cite{Miller1969} explored the idea of proxy voting as a more
direct method of democracy over representative democracy.
Most of his work focuses on the political and societal advantages of proxy
voting, and so is not directly applicable to the problem at hand.
However, he does bring up the idea this paper calls \textit{expert proxies},
those being individuals who would `vote as [the inactive voter] would if only
[the inactive voter] had the time and knowledge to participate
directly'~\cite[para.~I.3]{Miller1969}.
This concept is exactly what this paper plans to exploit: create and use expert
proxies when generating a measurement and allow less accurate techniques to use
them as proxies.


% TODO: Explain Mueller1972


Directly related to this paper,\ \cite{Cohensius2017} explores proxy voting with
infinite voters using set of \textit{mechanisms} or \textit{voting rules}.
These are algorithms used to consolidate votes into the end result.
The mechanisms discussed by \etal{Cohensius} included mean\footnote{An
interesting aside, though not directly relevant to this
study, is \etal{Cohensius} discovered the most extreme opinions will be
active under strategic participation when using a mean voting
mechanism\cite[lemma~9]{Cohensius2017}.}
and median rules on continuous spaces, as well as majority rule on
binary spaces.
\etal{Cohensius} allow for both random and strategic participation, meaning any
agent could decide whether it served as a proxy or not.

\etal{Cohensius}'s work serves as the basis of this study, which expands on
the voting mechanisms used as well as introduces an additional type of
mechanism: weighting mechanisms.
This study also explores using finite voters instead of infinite as discussed
in~\cite{Cohensius2017}, as well as uses a set of proxies instead of random or
strategic participation.
