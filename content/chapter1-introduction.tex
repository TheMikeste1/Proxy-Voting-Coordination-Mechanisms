%
%  This document contains chapter 1 of the thesis.
%

\chapter{INTRODUCTION}\label{ch:introduction}
%%%%%%%% This line gets rid of page number on first page of text
\thispagestyle{empty}
%%%%%%%%%%%%%

The ability to correctly take measurements is an important part of human
progress, both in terms of everyday lives and the advancement of science.
Having the ability to take highly precise and accurate measurements is essential
to maintaining safety, ensuring the accuracy of research, and providing a
higher quality of life.
This is particularly critical in high-risk or high-cost scenarios, such as when
human life is involved, or when the cost of using a sensor, taking a
measurement, or running an experiment is particularly expensive.
This paper explores the idea of using an adapted form of proxy voting, as
described in~\cite[para.~1.3]{Miller1969}, as a way of taking multiple
measurements to ensure accuracy while minimizing the cost of taking those
measurements.
\vicki{ In your introduction you need to introduce the problem you are trying to solve.  I wouldn’t jump to contributions immediately.}

\section{Contribution}\label{sec:contribution}
This study aims to explore the idea of using proxy voting as a technique to
improve the accuracy of multiple measurements while reducing the cost by
employing less expensive gauging methods when they are available.
This is accomplished by first taking a number of more expensive measurements to
serve as proxies, then supplementing any lack of precision by allocating
votes from cheaper methods of assessing a metric.
Additionally, this paper experiments with using different voting mechanisms and
weighting mechanisms in an attempt to optimize proxy voting for different
distributions of error.
Finally, this paper attempts to identify how many `proxy' measurements need
to be taken to reduce cost while still ensuring accuracy, and attempts to
identify any benefit of using proxy votes as opposed to simply averaging the
results.

This paper will show that, while proxy vote systems are not a perfect tool to
increase a system's accuracy, they may be beneficial when a distribution of error is
asymmetrical.
Additionally, this paper will identify the best performing voting and weighting
mechanisms, as well as discuss an ideal range of proxy and inactive agents to be used
in such a system.


\section{Potential Applications}\label{sec:potential-applications}
Proxy vote systems can be used in situations where highly accurate measurements are
difficult, costly, or time-consuming, and cheaper or easier alternatives are available.
While this study will show they will not work in all situations, they do have a
potential use when one of the alternative measurements tends to yield an asymmetrical
distribution of error.

These systems may also have use in ensemble machine learning techniques, though this
idea is not explored in this study.


\section{Background}\label{sec:background}
\textit{Proxy voting} involves allowing agents to transfer their voting power
from themselves to another agent, known as a \textit{proxy}\cite[para.~1.4]
{Cohensius2017}.
By doing so, the transferring agent loses their ability to directly affect the
voting game, instead allowing the proxy to affect it on their behalf.
In other words, the transferring agent does not vote in an election but rather
affects the election vicariously through the proxy to which they transferred
their power.
These transferring agents are also known as \textit{inactive voters}, and the
proxies can also be called \textit{active voters}.

In contrast, \textit{direct voting} requires each agent to vote on their own
(and so incur the costs).

In some cases, the proxy is also able to transfer their vote, as well as all
their weight, to yet another proxy.
\vicki{You haven’t told us about voting weight.  If I can transfer my vote, why not weight? }
This is known as \textit{liquid democracy}, which has its own challenges and
advantages.
However, simple proxy voting, where proxies cannot transfer their voting power,
is sufficient for the research performed in this paper.

Proxy voting has a number of advantages over direct voting.
The most obvious advantage is that it allows inactive voters to reduce the work
required of them by choosing a proxy to perform the work of voting for them.
Naturally, this is extremely useful when the cost of voting is high, such as
when remaining educated on a topic is difficult or
time-consuming~\cite[para.~1.1]{Mueller1972}, uncomfortable work such as
standing in long queues is required, or other frustrating or costly obstacles
must be overcome.
In these cases, proxy voting allows voters to skip incurring the costs while
still having their voice heard by allowing a proxy to perform the work once on
behalf of all voting power transferred to it.
\vicki{  But if I’m turning my vote to a proxy, h ow is my vote heard?  You could say you impacted the vote, but not that your vote was heard.  I don’t know how you can say the accuracy is increased, unless you know the proxy is smarter than you.
How do the voters find  proxies and how do they decide who they trust?  The reader is unclear about the setup.}

In addition to reducing individual agents' work, proxy voting also often has
the effect of increasing the system welfare by increasing the accuracy of the
system~\cite[sec.~1.1]{Cohensius2017}.
However, this increase does not work in all circumstances and so proxy voting
should be used after investigating the voting mechanism used and the potential
pitfalls of that mechanism under proxy voting.

Proxy voting also allows for agents that are uninformed about the issue at hand to
delegate the voting power to a trusted expert.
This is particularly useful for the purposes of this study.
By using less accurate but cheaper measurements, the system can still yield
increased accuracy and confidence in a result without requiring many more
expensive measurements.

However, proxy voting is not without its flaws.
\etal{Gölz} investigated a weakness in liquid democracy dubbed `super voters'
~\cite[para.~1.3]{Golz2021}, which are proxies that receive an extremely large
amount of power.
This can be problematic because the result of a system will be too biased
towards the super voter.
For this paper's purposes, a poor distribution of measurements may make one
proxy a super voter and increase the error of the result.
\vicki{here it appears you know absolute truth, and just want to assure the voting
finds it.}


\section{Related Work}\label{sec:related-work}
\vicki{You need to cite previous work which motivates your approach.  It seems your previous work doesn't really motivate what you are doing.}
Proxy voting and liquid democracy are well-discussed topics with a number of
papers investigating their viability, advantages, and disadvantages.
Many of these studies focus on political and societal uses for these techniques,
but some of their discoveries can be used for this paper's purposes.

James Miller~\cite{Miller1969} explored the idea of proxy voting as a more
direct method of democracy over representative democracy.
Most of his work focuses on the political and societal advantages of proxy
voting, and so is not directly applicable to the problem at hand.
However, he does bring up the idea this paper calls \textit{expert proxies},
those being individuals who would `vote as [the inactive voter] would if only
[the inactive voter] had the time and knowledge to participate
directly'~\cite[para.~1.3]{Miller1969}.
This concept is exactly what this paper plans to exploit: create and use expert
proxies when generating a measurement and allow less accurate techniques to use
them as proxies.

\etal{Mueller}~\cite{Mueller1972} reaffirms the need for expert proxies and
goes on to discuss how many of such proxies would be needed under random
selection for a sufficiently accurate opinion to be made.
These estimates range from 500 to 1000~\cite[para.~3.2]{Mueller1972}, and
while this study uses preassigned proxies, these proxies will have random
error in their estimations and so \etal{Mueller}'s estimates may come into
play.
However, a primary goal of this paper is to use far fewer measurements than the
quoted 500, as that many measurements would likely be far more expensive than
simply using a single, more costly form of measurement with better accuracy.

Directly related to this paper,\ \etal{Cohensius}~\cite{Cohensius2017} explore
proxy voting with infinite voters using a set of \textit{mechanisms}, also
known as \textit{voting rules}.
These are algorithms used to consolidate votes into the end result.
The mechanisms discussed by \etal{Cohensius} include mean\footnote{An
interesting aside, though not directly relevant to this
study, is \etal{Cohensius} discovered that the most extreme opinions will be
active under strategic participation when using a mean voting
mechanism\cite[lemma~9]{Cohensius2017}.}
and median rules in continuous spaces, as well as majority rule in
binary spaces.
\etal{Cohensius} allow for both random and strategic participation, meaning any
agent could decide whether it served as a proxy or not.

\etal{Cohensius}'s work serves as the basis of this study, which expands on
the voting mechanisms used as well as introduces an additional type of
mechanism: weighting mechanisms.
This study also explores the use of finite voters instead of infinite as discussed
in~\cite{Cohensius2017}, and explores the use of a set of proxies instead of
employing random or strategic participation.
While the use of finite voters in this case could be viewed as a two-layered
form of liquid democracy, this idea is not explored by this paper.
