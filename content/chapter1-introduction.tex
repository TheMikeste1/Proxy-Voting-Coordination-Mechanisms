%
%  This document contains chapter 1 of the thesis.
%

\chapter{INTRODUCTION}\label{ch:introduction}
%%%%%%%% This line gets rid of page number on first page of text
\thispagestyle{empty}
%%%%%%%%%%%%%

On May 13th, 2020, the 116th United States Congress introduced a resolution
to permit the use of proxy voting during emergencies for members of the House of
Representatives~\cite{Congress.gov2020}.
This resolution was passed on May 15th, 2020, and as indicated by proxy designation
letters sent to the Clerk of the House of Representatives~\cite{Clerk.House.gov2022},
was put to use as soon as five days later.
The hope behind the resolution was to allow members of the House of Representatives
to vote remotely, in order to prevent the spread of COVID-19~\cite{Congress.gov2020}.
However, the question remains as to what degree proxy voting affects the results of
the vote.
In this study, we explore the effects of proxy voting on the results of the vote
under a number of opinion distributions.
We additionally explore how different voting rules affect the results of the vote.

% TODO: State we assume fairly ideal conditions where an inactive voter can choose
% their proxy per-topic.
% TODO: Outline
% Discuss my assumptions
%   - Per-topic proxy selection
%   - Perfect knowledge of potential proxy's positions
%       - This might need to be discussed after the readers have an understanding of
%         the model.
%  - No other factors (e.g. party affiliation)
% Discuss opinion model (Cohensius)
% Background
%   - Proxy voting
%       - Why it is desirable
%   - What are voting rules
% Contribution

In order to conduct this study, we make a number of assumptions.
First, we assume that each voter is able to choose their proxy per-topic.
This is not currently the case for the House of Representatives, which requires
a proxy to be chosen by letter and so would be difficult to change as different
topics are discussed~\cite{Congress.gov2020}.
However, since it would be fairly simple to implement a new process that allows
per-topic proxies, and to give proxy voting its best chance we believe this assumption
is reasonable.

Second, we assume that each voter has perfect knowledge of potential proxies' opinions.
This will allow them to choose the proxy that has the most similar opinion to their own.
While in reality voters will likely not have perfect knowledge of others' opinions,
it is often not too hard to gauge the opinion of others, particularly those with whom
an individual often associates, and so we believe this assumption is reasonable.

Finally, we assume that there are no factors besides closeness in opinion that affect
the choice of proxy.
For example, we assume a voter would rather choose a proxy for a specific topic who
closely aligns with their opinion on that topic but has a different party affiliation
rather than choose a proxy who has the same party affiliation but a different opinion.

As part of this study, we employ a model developed by \etal{Cohensius} in
their 2017 article~\cite{Cohensius2017}.
This model places voters' preferences, which for our purposes are the congress
members' opinions on a topic, in a metric space \systemspace, such as
in~\autoref{fig:system-metric-space}.
Modeling voters' preferences in such a way allows for continuous preferences instead of
discrete or binary opinions other models might emphasize.
This is beneficial, as it allows for a more realistic representation of voters'
opinions, since it is unlikely that voters' opinions are perfectly binary.
For example, it is likely that some members are very passionate for the
use of proxy voting in the House, while others are passionately against it, and yet
others remain ambivalent or less-passionately for or against.
This works perfectly with the metric space model, as it allows for the passionate
opinions to be at opposite extremes of the metric space, while the less-passionate
opinions are closer to the center.

\begin{figure}[htbp]
    \centering
    % Built using:
    % https://tex.stackexchange.com/a/148253/277236
    % https://tex.stackexchange.com/a/380491/277236
    \begin{tikzpicture}[scale=1.5]
        \draw(0,0) -- (10,0) ; % Axis
        \foreach \x in  {0, 10} % Vertical higher lines
        \draw[shift={(\x,0)},color=black] (0pt,5pt) -- (0pt,0pt);
        \foreach \x in {0, 5, 10} % Numbers and lower lines
        \draw[shift={(\x,0)},color=black] (0pt,0pt) -- (0pt,-5pt) node[below]{$\x$};

        % Labeled points
        \tkzDefPoint((-4/7 + 1) * 5, 0){agentA}
        \tkzDefPoint((3/4 + 1) * 5, 0){agentB}
        \tkzDefPoint((1/12 + 1) * 5, 0){agentC}
        \tkzLabelPoint[above](agentA){$\truthof{a}$}
        \tkzLabelPoint[above](agentB){$\truthof{b}$}
        \tkzLabelPoint[above](agentC){$\truthof{c}$}

        \foreach \n in {agentA, agentB, agentC}
        \node at (\n)[circle,fill,inner sep=1.75pt]{};
    \end{tikzpicture}

    \caption{Example of a 1D preference metric space, where \truthof{x} represents the
    preference of agent $x$.}
    \label{fig:system-metric-space}
\end{figure}

This model also naturally extends to two- or higher-dimensional spaces, such as
the opinions of a congress member on multiple topics.
However, as previously mentioned, we will only consider the idealistic
situation where a congress member can choose their proxy for each topic individually.
The model also allows for votes on topics that are not binary, such setting the
budget for the year;
instead of simply voting yes or no on some pre-chosen budget, the output of the system
can serve as the budget.

% Explain what Cohensius did
%   - Strategic participation
%   - Infinite voters
%   - Voting Mechanisms
% Explain what I'm borrowing and what I'm doing differently
%   - Compulsory use of proxies
%   - Finite voters
%   - Additional voting mechanisms
%   - Weighting mechanisms

This study also makes use of \textit{voting mechanisms} or \textit{voting rules},
which are functions that map a set of preferences in~\systemspace\ to an outcome that
also exists in~\systemspace.
%
% \com{The idea of a voting mechanism did NOT start with Cohensius. Give credit to the
% proper originator.}
% MDH: I've done some research to find the originator of the term, but it seems to be
% a fairly common term with no clear originator. The idea at the very least seems to
% date back all the way to ancient Greece, so I'm hoping it's safe to assume it's a
% common knowledge term.
%
This study employs the use of several common voting mechanisms and adapts them to the
proxy voting system.
What these mechanisms are and how they operate are described
in~\autoref{subsec:voting-mechanisms}.

This paper will explore how proxy voting affects the vote and under what
circumstances it can be used by the House without changing the result.
This will be determined by measuring by how much the system output changes when using
proxy voting from all members voting and only present members voting.
This will be done using continuous outputs instead of binary `yea' or `nay' votes,
since this will allow for more precise measurements in change.


\section{Background}\label{sec:background}
\textit{Proxy voting} involves allowing agents to transfer their voting power
from themselves to another agent, known as a \textit{proxy}.
This adds to the voting power of the proxy, which is called its `weight.'
By delegating a proxy, the transferring agent loses their ability to directly affect the
voting game, instead allowing the proxy to affect it on their behalf.
In other words, the transferring agent does not vote in an election or other types of
voting games, but rather affects the election vicariously through the proxy to which
they transferred their power.
These transferring agents are also known as \textit{inactive voters}, and the
proxies can also be called \textit{active voters}.
In contrast, \textit{direct voting} requires each agent to vote on their own.

Proxy voting has a number of advantages over direct voting.
The most obvious advantage is that it allows inactive voters to reduce the work
required of them by choosing a proxy to perform the work of voting for them.
Naturally, this is extremely useful when the cost of voting is high, such as
when remaining educated on a topic is difficult or
time-consuming~\cite{Mueller1972} uncomfortable work such as standing in
long queues is required, or other frustrating or costly obstacles must be overcome.
For our purposes it is likely undesirable that the agents, being members of Congress,
do not pay the cost of remaining educated, but proxy voting can still be beneficial
to the agents because it would allow them to prevent the spread of disease or avoid
whatever other emergency triggered the use of proxy voting.

In these cases, proxy voting allows voters to skip incurring the costs while
still having their voice heard by allowing a proxy to perform the work once on
behalf of all voters who transfer their power to that proxy.
Of course, a voter would not want to give its voting power to just any proxy.
This study employs a similar technique as used in~\cite{Cohensius2017}, which has the
voters, or agents, pass their voting power to the proxy that is closest in the
preference space.

In addition to reducing individual agents' work, proxy voting also often has
the effect of increasing the system accuracy as opposed to only allowing active
agents to vote~\cite{Cohensius2017}.
This follows intuition: if a voter doesn't vote, the system loses information.
By allowing them to still influence the voting game through proxy, some information
is reintroduced into the system.
Naturally, in terms of system accuracy the ideal situation is when all voters
participate.
However, system welfare can be enhanced on an individual level by allowing agents
who want to be inactive to become inactive.
When operating under the Congress resolution, system welfare can also be enhanced
through proxy voting by allowing ill members to quarantine and ensure the health
of other agents.
As will be discussed in \autoref{ch:results}, this additional system welfare comes at
the cost of some system accuracy in terms of the actual preferences of the voters
since not all information can be restored.

In some cases, the proxy is also able to transfer their own vote, as well as the rest of
their delegated weight, to yet another proxy.
This is known as \textit{liquid democracy}, which has its own challenges and
advantages.
However, simple proxy voting, where proxies cannot transfer their voting power,
is sufficient for the research performed in this paper.
% TODO: Rewrite here and below


\section{Related Work}\label{sec:related-work}
James Miller~\cite{Miller1969} proposed using proxy voting as a more direct
method of democracy rather than representative democracy.
His work focuses on reworking the current House and Senate systems entirely by using a
more-directly involved populace, but his ideas can still be relevant under the current
system.
In particular, he introduces the idea we call \textit{expert proxies},
those being individuals who would `vote as [the inactive voter] would if only
[the inactive voter] had the time and knowledge to participate
directly'~\cite[para.~1.3]{Miller1969}.
In other words, these are well-studied proxies who are familiar with the topic at hand.
Additionally, he states, `a representative should be an expert, or at least
competent, in each field [on which they are voting]'~\cite[para.~2.7]{Miller1969}.
Whereas the past 25 Congresses have seen anywhere from 10 to over 25 thousand issues
over 2 years (though ultimately only around 10\% are actually
discussed)~\cite{GovTrack2022}, it would be nearly impossible for any individual
congress member to remain informed on all topics.
While the current resolution allows vote by proxy only for emergencies, it is
possible the resolution could be expanded to exploit this idea.

However, proxy voting is not without its flaws.
\etal{Kling} investigated a weakness in liquid democracy dubbed `super voters'
~\cite[para.~1.3]{Kling2015}, which are proxies that receive an extremely large
amount of power, while others gain very little.
While they determined these proxies tend to use their power wisely, possible to avoid
estranging those voters who delegate their power to the proxy, there can be
situations where they can be problematic with one-off issues.
As a simple example, consider a vote taking over preference
space $\systemspace = [-1, 1]$ with agent preferences $\truthof{\agent_1} = -1$,
$\truthof{\agent_2} = 0.25$, $\truthof{\agent_3} = 0.5$, $\truthof{\agent_4} = 1$,
$\truthof{\agent_5} = 0.15$.
Under the mean mechanism, discussed in \autoref{para:mean}, if everyone were to vote
we would get the actual preference of the system, $\systemtruth = 0.18$.
However, if $\agent_2$, $\agent_3$, and $\agent_5$ were to become inactive and select
$\agent_4$ as their proxy, making them a super voter, the result would be
$\systemtruth = 0.6$.
This is visualized in \autoref{fig:voting-example}.
This is an absolute error of 0.42, or 21\% of the entire preference space!
While $\agent_4$ would certainly be the preferred proxy, if $\agent_5$ were to
instead strategically delegate their vote to $\agent_1$, the result would
be $\systemtruth = 0.2$, which is closer to the actual system preference.
However, without knowing to whom the other agents are delegating their vote, or even
if they will, $\agent_5$ has no reason to be strategic.

\begin{figure}[htbp]
    \centering
    % Built using:
    % https://tex.stackexchange.com/a/148253/277236
    % https://tex.stackexchange.com/a/380491/277236
    \begin{tikzpicture}[scale=7.0]
        \draw(-1,0) -- (1,0) ; % Axis
        \foreach \x in {-1, 0, 1} % Numbers and lower lines
        \draw[shift={(\x,0)},color=black] (0pt,0pt) -- (0pt,-2pt) node[below]{$\x$};

        % Labeled points
        \tkzDefPoint(-1, 0){agent1}
        \tkzDefPoint(0.25, 0){agent2}
        \tkzDefPoint(0.5, 0){agent3}
        \tkzDefPoint(1, 0){agent4}
        \tkzDefPoint(0.15, 0){agent5}
        \tkzLabelPoint[above](agent1){$\truthof{\agent_1}$}
        \tkzLabelPoint[below](agent2){$\truthof{\agent_2}$}
        \tkzLabelPoint[above](agent3){$\truthof{\agent_3}$}
        \tkzLabelPoint[above](agent4){$\truthof{\agent_4}$}
        \tkzLabelPoint[above](agent5){$\truthof{\agent_5}$}

        \foreach \n in {agent1, agent2, agent3, agent4, agent5}
        \node at (\n)[circle,fill,inner sep=1.75pt]{};


        % Actual preference
        \draw[color=blue, line width=0.5mm, dotted]
        (0.18, 0pt) -- (0.18, -3pt);
        \node[color=blue] at (0.18,-4pt) {$\systemtruth_{actual}$};

        % Preference under proxy vote
        \draw[color=orange, line width=0.5mm, dotted]
        (0.6, 0pt) -- (0.6, -3pt);
        \node[color=orange] at (0.6,-4pt) {$\systemtruth_{proxy}$};
    \end{tikzpicture}

    \caption{
        An example vote and its results.
        $\textcolor{blue}{\systemtruth_{actual}}$ is the result when everyone votes,
        and $\textcolor{orange}{\systemtruth_{proxy}}$ is when $\agent_2$, $\agent_3$,
        and $\agent_5$ delegate their vote and make $\agent_4$ a super voter.
    }
    \label{fig:voting-example}
\end{figure}


\section{Contribution}\label{sec:contribution}
This study aims to explore the use of proxy voting for Congress.
We examine how much of an effect proxy voting has on the outcome of a vote under a
number of distributions of opinions, as well as with various numbers of inactive
agents.
Additionally, this paper experiments with using different voting mechanisms in order
to determine if these proxy votes should be handled differently from normal voting.
Finally, this paper will watch for the prevalence of super voters and attempt to
assess any effect, beneficial or harmful, they have on the system.

% TODO: Fill out synopsis of what is learned after the data is analysed
% This paper will show that, while proxy vote systems are not a perfect tool to
% increase a system's accuracy, they may be beneficial when the distribution of error
% from a measurement is asymmetrical.
% Additionally, this paper will identify the best performing voting and weighting
% mechanisms, as well as discuss an ideal range of proxy and inactive agents to be used
% in such a system.

% TODO: Maybe combine this section with "Contribution," explaining how we hope to be
%   able to apply what's learned here to actual scenarios?
% \section{Potential Applications}\label{sec:potential-applications}
% Proxy vote systems can be used in situations where highly accurate measurements are
% difficult, costly, or time-consuming, and cheaper or easier alternatives are available.
% While this study will show they will not work in all situations, they do have a
% potential use when one of the alternative measurements tends to yield an asymmetrical
% distribution of error.
%
% These systems may also have use in ensemble machine learning techniques, though this
% idea is not explored in this study.
