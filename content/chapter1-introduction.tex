%
%  This document contains chapter 1 of the thesis.
%

\chapter{INTRODUCTION}\label{ch:introduction}
%%%%%%%% This line gets rid of page number on first page of text
\thispagestyle{empty}
%%%%%%%%%%%%%

\section{Preliminaries}\label{sec:preliminaries}
The ability to correctly take measurements is an important part of human
progress, both in terms of every day lives and the advancement of science.
Having the ability to take highly precise and accurate measurements is essential
to maintaining safety, ensuring research is accurate, and providing a higher
quality of life.
This is particularly critical in high-risk or high-cost scenarios, such as when
human life is involved, or when the cost of using a sensor or taking a
measurement is particularly expensive.  % Maybe expand on what "expensive" means?
This paper explores the idea of using an adapted form of proxy voting, as
described in~\cite{Cohensius2017}, as a way of taking multiple measurements to
ensure accuracy while minimizing the cost of taking those measurements.

\textit{Proxy voting} involves allowing agents to transfer their voting power
from themselves to another agent, known as a \textit{proxy}\cite[para. 1.4]
{Cohensius2017}.
In doing so, the transferring agent looses their ability to directly affect the
voting game.
In other words, the transferring agent does not vote in an election but rather
affects the election vicariously through the proxy to which they transferred
their power.
These transferring agents are also known as \textit{inactive voters}, and the
proxies can also be called \textit{active voters}.

Normally the primary purpose of using proxy voting is to increase the welfare
of the system by reducing the work required from the inactive voters.
This is particularly useful when the process of voting is difficult or costly
in some manner, such as when standing in long queues is required, frustrating
restrictions are in place, or some other cost is incurred.
In these cases, proxy voting allows agents to become inactive, thereby allowing
the agents to skip paying the costs while still having their voice be heard
through the proxies.

While use cases of proxy voting are typically focussed on situations that
are political or sociological in nature, this study aims to discover similar
advantages in other realms.
Primarily, the hope is to reduce the impact of measurement error by using
a combination of `expensive' and `inexpensive' measurements to obtain a
highly-accurate metric, about which the system can be confident.


\section{Potential Applications}\label{sec:potential-applications}
- % TODO: List potential applications, such as high risk/cost scenarios

\section{Previous Work}\label{sec:previous-work}
Proxy voting and other similar ideas have been explored a number of times
primarily in the political and social science spheres\cite{Cohensius2017, Mueller1972, Zhang2022, Goelz2021}.
These studies typically focus on the advantages, disadvantages, as well as
augmentations of proxy voting when used by a society.

- "Proxy Voting for Better Outcomes"\cite{Cohensius2017} explores the idea of
  using proxy voters to reduce, and even minimize under some voting mechanisms,
  the social cost of voting.
    - Paper assumes infinite voters, but I use finite voters (as few as possible)
    - Discuss voting mechanisms and discoveries
    - Explain how what I'm doing is different

Research to be done:
- Any similar papers
- Finite proxy voting
% Those below will likely require a specific type of sensor
- Current sensor error mitigation techniques
- Typical sensor error rates
- Sensor variance
- Sensor error distributions (e.g.\ Gaussian)

