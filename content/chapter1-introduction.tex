%
%  This document contains chapter 1 of the thesis.
%

\chapter{INTRODUCTION}\label{ch:introduction}
%%%%%%%% This line gets rid of page number on first page of text
\thispagestyle{empty}
%%%%%%%%%%%%%
On May 13th, 2020, the 116th United States Congress introduced a resolution
to permit the use of proxy voting during emergencies for members of the House of
Representatives~\cite{Congress.gov2020}, which would allow members to participate in
proceedings remotely while still having someone physically present to submit the
member's vote on their behalf.
This resolution was passed on May 15th, 2020, and as indicated by proxy designation
letters sent to the Clerk of the House of Representatives~\cite{Clerk.House.gov2022},
was put to use as soon as five days later.
The purpose behind the resolution was to allow the House of Representatives
to continue operating while still allowing members to quarantine and prevent the
spread of COVID-19~\cite{Congress.gov2020}, as well as to pave the way for future use
of proxy voting in other such emergencies.
The type of proxy voting in Congress is not the traditional type of proxy vote.
Instead, it is more similar to voting remotely, similar to how one might attend a
meeting via video conference. Using proxy voting in such an important part of the
legislative process, reignites the discussion on the effectiveness of proxy voting.
\vicki{I split up the previous sentence.  See if it still has the meaning you intended.}
Does proxy voting introduce too much error into the voting process?
Or is it a viable alternative to direct voting?
If proxy voting does introduce error, perhaps alternative voting mechanisms or systems
can be used to mitigate that effect while still maintaining the benefits of proxy
voting.
This study attempts to answer these questions by analyzing the effects of proxy
voting on the results of a vote under a number of preference distributions.
We employ several voting mechanisms and rank them according to their performance when
applied under several preference distributions.
We additionally explore the use of alternative proxy voting systems, such as
allowing an individual to select a preference dependent on the positions of their
constituents, and determine if they are more effective than the traditional proxy
voting system where a single proxy is chosen.
Finally, as a part of this study we implement and describe voting over a continuous
interval instead of the typical binary (in favor or against) vote.

This topic is of large importance because of the potential ramifications of using
proxy voting directly in the legislative process.
If proxy voting does not have a large effect on the results of the vote, then it
can be used to significantly ease the process and lower the risk of participating in
Congress during pandemics or other emergencies.
Similarly, if proxy voting can be shown to have a large effect on the results of the
vote, it would be important to alter the system to mitigate the effects or stop using
proxy voting in the House altogether.
In either case, it is important to understand the effects of proxy voting in order to
utilize it in the most effective way possible, as well as discover any potential
techniques to improve such a system.


\section{Background}\label{sec:background}
\textit{Proxy voting} involves allowing agents to transfer their voting power
from themselves to another agent, known as a \textit{proxy} or a \textit{delegate}.
This adds to the voting power of the proxy, which is called its `weight.'
By delegating a proxy, the transferring agent, also known as the \textit{delegator},
loses their ability to directly vote, instead allowing the proxy to affect it on
their behalf.
In other words, the transferring agent does not vote in an election or other types of
votes, but rather affects the election vicariously through the proxy to which they
transferred their power.
These transferring agents are also known as \textit{inactive voters}, and the proxies
can also be called \textit{active voters}.

In contrast, \textit{direct voting} requires each agent to vote on their own.
Proxy voting has a number of advantages over direct voting.
The most obvious advantage is that it permits delegators to reduce the work
required of them to participate in a vote by choosing a proxy to perform the work of
voting for them.
Naturally, this is extremely useful when the cost of voting is high, such as
when one is ill or fears they may become ill, or when a member's time can be spent
more efficiently without being physically present for a vote like when attending a
conference or engaging in active research in some location outside {Washington,~D.C}.
This is, of course, directly applicable to the House of Representatives, since proxy
voting would allow them to prevent the spread of disease or avoid whatever other
emergency triggered the use of proxy voting.

In these cases, proxy voting allows voters to skip incurring the costs while still
having their voice heard by allowing a proxy to pay the cost once on behalf of all
voters who transfer their power to that proxy.
Of course, a voter would not want to give its voting power to just any proxy.
This study employs a similar technique as used in~\cite{Cohensius2017}, which has the
voters, or agents, pass their voting power to the proxy that is closest in the
preference space.

The `goodness' of a system here is measured using two primary metrics: the
\textit{accuracy} or \textit{error} of the system, meaning how close the system is to
the optimal solution, and the total system \textit{welfare}.
Welfare is a measure of how good a situation or action is for an agent.
For example, an agent not being required to vote in person increases that agent's
welfare since it makes it easier for the agent to vote.
Total system welfare is the sum of the welfare of all agents in the system.
An excellent system would have high accuracy and high welfare, while a poor system
would have low accuracy and low welfare.

Proxy voting has been shown to often increase system accuracy when compared to only
allowing active/willing agents to vote~\cite{Cohensius2017}.
This follows intuition: if a voter doesn't vote, the system loses information.
By allowing them to still influence the voting game through proxy, some information
is reintroduced into the system.
Naturally, in terms of system accuracy the ideal situation is when all voters
participate and so the system has the most information possible.
However, system welfare can be enhanced on an individual level by allowing agents
who want to be inactive through the use of a proxy to delegate their vote.
Therefore, when operating under the Congress resolution, system welfare can be
enhanced through proxy voting by allowing ill members to quarantine and ensure the
health of other agents.
As will be discussed in the results section of the thesis,
% \autoref{ch:results},  % TODO: Uncomment for thesis
this additional system welfare comes at the cost of some system accuracy in terms of
the actual preferences of the voters.
This cost comes about because voters are generally unable to perfectly express their
preference and so rely on their proxy to vote in a way that is close to their
preference.
Not all information can be restored through the use of a proxy since they are also
unable to perfectly express their preference nor those of their delegator(s),
\vicki{Unclear.  who is "they"?   What are you assuming about the proxy?  If my vote is .8, must I choose the nearest proxy?  Why don't I know the preference of the delegator?  Isn't the delgator the inactive voter?}
which causes this information to be lost and the system's accuracy to suffer.
However, if the results of the system are not too far from the optimal solution, then
this loss of accuracy might be worth the additional system welfare produced by allowing
agents delegate their vote.

\vicki{Could we also claim that a certain percentage of the votes would be lost if no proxy were allowed?  In other words, someone may be genuinely unable to attend.}

In some cases, the proxy is also able to transfer their own vote, as well as the rest of
their delegated weight, to yet another proxy.
This is known as \textit{liquid democracy}, which has its own challenges and advantages.
Ultimately, the use of liquid democracy is not considered in this study, but since proxy
voting is a subset of the liquid democracy system, it would likely be possible to
extend the results of this study to liquid democracy as well.

Proxy voting may seem like an extremely attractive option for the House of
Representatives, and other organizations, to use.
However, previous research has already shown it is not without its flaws.
\etal{Kling} and \etal{Gölz} all investigated a weakness in liquid democracy dubbed
`super voters'~\cite{Kling2015,Golz2021}, which are proxies that receive an extremely
large amount of power, while others gain very little.
While \etal{Kling} ultimately determined these proxies tend to use their power
wisely, possibly to avoid estranging those voters who delegate their power to the
proxy, there can be situations where super voters can be problematic with one-off
issues.
As a simple example, consider a vote taking over preference
space $\systemspace = [-1, 1]$ with agent preferences $\truthof{\agent_1} = -1$,
$\truthof{\agent_2} = 0.25$, $\truthof{\agent_3} = 0.5$, $\truthof{\agent_4} = 1$,
$\truthof{\agent_5} = 0.15$.
Further details on such a preference space model and its mechanics are described in
\autoref{sec:setup-and-assumptions}.
\vicki{Better not to refer to future section, when the knowledge is needed here.}
Under the mean mechanism, which aggregates opinions by simply taking their mean, if
everyone were to vote we would get the actual preference of the
system: $\systemtruth = 0.18$.
However, if $\agent_2$, $\agent_3$, and $\agent_5$ were to become inactive and select
$\agent_4$ as their proxy, making them a super voter, the result would be
$\systemtruth = 0.6$.
\vicki{But the problem isn't that there is a super voter.  The problem is that the proxy has no close choice.  When you introduce the term super voter, I assumed you were going to say that the super voter didn't vote as directed, and because he had so much power, that was a problem.}
This is visualized in \autoref{fig:voting-example}.
This is an absolute error of 0.42, or 21\% of the entire preference space!
Ideally the output of a system employing proxy voting would be much closer to the
actual preference of the system.
While there are methods to help mitigate the effect of super voters, which
\etal{Gölz}~\cite{Golz2021} explore, the amount of error produced by a proxy vote
system, including that of super voters, is precisely what this paper will explore.

\begin{figure}[htbp]
    \centering
    % Built using:
% https://tex.stackexchange.com/a/148253/277236
% https://tex.stackexchange.com/a/380491/277236
\begin{tikzpicture}[scale=7.0]
    \draw(-1,0) -- (1,0) ; % Axis
    \foreach \x in {-1, 0, 1} % Numbers and lower lines
    \draw[shift={(\x,0)},color=black] (0pt,0pt) -- (0pt,-2pt) node[below]{$\x$};

    % Labeled points
    \tkzDefPoint(-1, 0){agent1}
    \tkzDefPoint(0.25, 0){agent2}
    \tkzDefPoint(0.5, 0){agent3}
    \tkzDefPoint(1, 0){agent4}
    \tkzDefPoint(0.15, 0){agent5}
    \tkzLabelPoint[above](agent1){$\truthof{\agent_1}$}
    \tkzLabelPoint[below](agent2){$\truthof{\agent_2}$}
    \tkzLabelPoint[above](agent3){$\truthof{\agent_3}$}
    \tkzLabelPoint[above](agent4){$\truthof{\agent_4}$}
    \tkzLabelPoint[above](agent5){$\truthof{\agent_5}$}

    \foreach \n in {agent1, agent2, agent3, agent4, agent5}
    \node at (\n)[circle,fill,inner sep=1.75pt]{};


    % Actual preference
    \draw[color=blue, line width=0.5mm, dotted]
    (0.18, 0pt) -- (0.18, -3pt);
    \node[color=blue] at (0.18,-4pt) {$\systemtruth_{actual}$};

    % Preference under proxy vote
    \draw[color=orange, line width=0.5mm, dotted]
    (0.6, 0pt) -- (0.6, -3pt);
    \node[color=orange] at (0.6,-4pt) {$\systemtruth_{proxy}$};
\end{tikzpicture}

    \caption{
        An example vote and its results.
        $\textcolor{blue}{\systemtruth_{actual}}$ is the result when everyone votes,
        and $\textcolor{orange}{\systemtruth_{proxy}}$ is when $\agent_2$, $\agent_3$,
        and $\agent_5$ delegate their vote and make $\agent_4$ a super voter.
    }
    \label{fig:voting-example}
\end{figure}


\section{Preliminary Setup and Assumptions}\label{sec:setup-and-assumptions}
An important part of any study is the model it uses to represent the system being
studied.
In this work, we employ a model developed by \etal{Cohensius} in their 2017
article~\cite{Cohensius2017}.
This model places voters' preferences, which for our purposes are the congress
members' opinions on a topic, in a metric space \systemspace, such as
in~\autoref{fig:system-metric-space}.
Modeling voters' preferences in such a way allows for continuous preferences instead of
discrete or binary opinions other models might emphasize.
This is beneficial, as it allows for a more realistic representation of voters'
opinions, since it is unlikely that voters' opinions are perfectly binary.
For example, it is likely that some members are passionately in favor of higher
spending on education, while others are passionately against it, and yet others
remain ambivalent or less-passionately for or against.
This works perfectly with the metric space model, as it allows for the passionate
opinions to be at opposite extremes of the metric space, while the less-passionate
opinions are closer to the center.

\begin{figure}[htbp]
    \centering
    % Built using:
% https://tex.stackexchange.com/a/148253/277236
% https://tex.stackexchange.com/a/380491/277236
\begin{tikzpicture}[scale=1.45]
    \draw(0,0) -- (10,0) ; % Axis
    \foreach \x in  {0, 10} % Vertical higher lines
    \draw[shift={(\x,0)},color=black] (0pt,5pt) -- (0pt,0pt);
    \foreach \x in {0, 5, 10} % Numbers and lower lines
    \draw[shift={(\x,0)},color=black] (0pt,0pt) -- (0pt,-5pt) node[below]{$\x$};

    % Labeled points
    \tkzDefPoint((-4/7 + 1) * 5, 0){agentA}
    \tkzDefPoint((3/4 + 1) * 5, 0){agentB}
    \tkzDefPoint((1/12 + 1) * 5, 0){agentC}
    \tkzLabelPoint[above](agentA){$\truthof{a}$}
    \tkzLabelPoint[above](agentB){$\truthof{b}$}
    \tkzLabelPoint[above](agentC){$\truthof{c}$}

    \foreach \n in {agentA, agentB, agentC}
    \node at (\n)[circle,fill,inner sep=1.75pt]{};
\end{tikzpicture}
    \caption{
        Example of a 1D preference metric space, where \truthof{x} represents the
        preference of agent $x$.
        The x-axis represents some preference space, where the leftmost point is
        the most preference most against some idea and the rightmost point is the most
        preference most in favor of the same idea.
    }
    \label{fig:system-metric-space}
\end{figure}
\vicki{This figure should really come before Figure 1.1}

This model also naturally extends to two- or higher-dimensional spaces.
These would include more complex and multi-faceted topics, such as migration, which
would include subtopics such as border restrictions and economic impact, or where to
delegate funds in the United States Budget.
However, as previously mentioned, we will only consider the idealistic situation
where a congress member can choose their proxy for each topic individually.
The model obviously allows for votes on topics that are continuous, such as setting the
budget for the year: instead of simply voting yes or no on some pre-chosen budget,
the output of the system can serve as the budget.
\vicki{Better as "budget amount" - as budgets are more complicated.  If money was cut in half, funding for each item wouldn't be cut in half.}
However, discrete and binary issues can be supported too, and there are multiple
ways of interpreting the output of the model for such issues.
For example, one could round the output to the nearest valid choice, such as a 1 or a
0 for binary issues.

Such a model also permits an additional aspect of this study: the ability to vote in
a range instead of aforementioned binary or discrete votes.
The hope behind such a system is to gain a better understanding of the voters'
true preference, which is not possible with binary or discrete votes due to all votes
ultimately being binned into one of several values.
This system will allow one to see if the majority is truly in favor of some idea, or
if they are only slightly in favor of it.
For example, consider a situation where voters are asked to vote on some new law and
are able to vote in the interval of $[-1, 1]$.
If the majority of votes are hovering around the center, say between in the interval
$[-0.25, 0.25]$, then it is likely that the majority of voters are not actually fully
satisfied by the law, but still believe some change is necessary.
This provides a third option: refactoring the law to be more in line with the
voters' opinions.
\vicki{I like this option.  Without it, one might say they are diluting their vote by not being a strong yes or no.  I also wonder if there would be political pressure on someone who didn't vote definitively.  I suppose we could also claim that their actual  interval vote was private, but the public manifestation was the extreme.
I think you need to discuss the ramifications of voting on an interval.
}

This study also makes use of \textit{voting mechanisms} or \textit{voting rules},
which are functions that map a set of preferences in~\systemspace\ to an outcome that
also exists in~\systemspace.
%
% \com{The idea of a voting mechanism did NOT start with Cohensius. Give credit to the
% proper originator.}
% MDH: I've done some research to find the originator of the term, but it seems to be
% a fairly common term with no clear originator. The idea at the very least seems to
% date back all the way to ancient Greece, so I'm hoping it's safe to assume it's a
% common knowledge term.
%
We employ several common voting mechanisms and adapt them to the proxy voting system.
Which mechanisms are used and how they operate will be described in the full thesis.
% in~\autoref{subsec:voting-mechanisms}.  % TODO: Uncomment for thesis

The flexibility of this model in both the discrete and continuous realms, its ability
to use different voting rules, and its easy interpretability, are the reasons it is
employed in this study.
This study will focus primarily on the continuous instead of the discrete output of
the model, since observing the continuous output allows for more granularity in the
differences between proxy and non-proxy voting, as well as in the differences between
voting rules.

Additionally, under normal proxy voting each voter can only delegate a single proxy.
This provides a simple system where a voter can simply select a single individual and
allow them to vote on their behalf.
This has the downside, however, that the proxy might not have as close of an opinion
to the voter as they would like.
This raises the question, what if we allowed a voter to delegate multiple proxies?
They could, for example, delegate their vote to a trusted group, who would then
aggregate their opinions into some preference in the aforementioned model on behalf of
the delegator.
This may reduce the error introduced by using proxies and allow the agent to feel
more secure that their voting power will be used as they desire.
This is not an entirely novel idea, with other authors already creating models to
handle it~\cite{Degrave2014,Colley2021,Golz2021}.

\subsection{Assumptions}\label{subsec:assumptions}
In order to conduct this study, we make a number of assumptions.
First, we assume that each voter is able to choose their proxy, also known as a
delegate, individually for each topic.
This is not currently the case for the House of Representatives, which requires
a proxy to be chosen by letter and so would be difficult to change as different
topics are discussed~\cite{Congress.gov2020}.
However, since it would be fairly simple to implement a new process that allows
per-topic proxies, and since many complex topics can be reduced to a set of related
subtopics, we believe this assumption is reasonable.

Additionally, the resolution currently requires the delegate to vote exactly as the
specific instructions provided by the delegating voter tells them to
vote~\cite{CERP2020, Congress.gov2020}.
This essentially turns proxy into a relay for the delegating voter, which is not how
proxy voting is typically used and is not particularly useful.
Instead, we consider scenarios where actual proxy voting is used, where the proxy
receives the voting power of the delegating voter, and the proxy can vote
however they choose.
\vicki{Isn't the question whether we just change the weight of the proxy?  Are we assuming the proxy can say, "My votes are .8, .75, and .9.  I always assumed it was just the weight that changed, as otherwise it isn't interesting.  Why is proxy the way it is?  It does seem a bit simpler to just change the weight.  I think we need to make our assumptions clear earlier, as it affects your example Figure 1.1.}
This requires the delegating voter to choose a proxy who they trust to vote as close
to how they would as possible if only they were able to participate in the direct vote,
similar to selecting experts as described in~\cite{Miller1969} and~\cite{Mueller1972}.

Second, we assume that each voter has perfect knowledge of potential proxies' opinions.
This will allow them to choose the proxy that has the most similar opinion to their own.
While in reality voters will likely not have perfect knowledge of others' opinions,
it is often not particularly difficult to gauge the opinion of others, especially
those with whom an individual often associates, and so we believe this assumption is
reasonable.

Third, we assume that abstention is not allowed.
In other words, all agents must vote either themselves or by proxy.
This is reasonable because all individuals will have some form of opinion, even if
that opinion is completely neutral.
Such an opinion is handled by the model this study employs, which will be discussed
later.

Finally, we assume that there are no factors besides closeness in opinion that affect
the choice of proxy.
This differs from the actual system presently used by the House of Representatives,
which includes restrictions such as a proxy can only serve ten voters~\cite{CERP2020}.
However, we feel removing restrictions such as these leads to a more interesting
discussion, since it allows the use of different voting mechanisms and more extreme
cases.


\section{Previous Work}\label{sec:previous-work}
% \vicki{
%     This isn't previous work.
%     Previous work is more like: Researcher B looked at proxy voting in two space
%     using manhattan distance between proxy and inactive voter.
%     Using the methods of X,Y, and Z, they discovered that proxy voting had an error
%     of Q when used on population T. However, they did not consider W (which we intend
%     to explore).
%     In other words, your previous work needs to give the best alternative to what you
%     are doing.
%     You will also compare yourself to the alternative.
%     We need to know what others have done in order to (1) know if you've done
%     anything useful (2) understand your credentials in doing this work.
%     Previous work will also describe standard methods of doing analysis, so we can
%     see the motivation for the way you designed the tests.
% }
% FIXME: This might need to be moved to Background. It feels somewhere in between
%   both Previous Work and Background.
James Miller~\cite{Miller1969} imagined a governmental system utilizing proxy voting
 in 1969 as a more direct form of a representative democracy.\footnote{
    That is to say, Miller envisioned a system where individuals could directly vote
    for an issue, or elect a proxy to vote for them.
    Naturally, any democracy that uses proxy voting is a representative democracy,
    since the proxy is representing the delegator.
    Nevertheless, it can be argued that Miller's proposal could provide a more direct
    democracy since a voter can directly vote for an issue is they so choose.
}
His work focuses on reworking the current House and Senate systems entirely by using a
more-directly involved populace, but his ideas can still be relevant under the current
system.
In particular, he introduces the idea we call \textit{expert proxies},
those being individuals who would `vote as [the delegator] would if only
[the delegator] had the time and knowledge to participate directly'~\cite{Miller1969}.
While for the general populace this could be a very valuable benefit of proxy voting,
it is not entirely desirable for the House of Representatives.
One of the reasons individuals are elected to the House of Representatives is to
research and create laws that are in the best interest of the people on behalf of
the people.
Though the past 25 Congresses have seen anywhere from 10 to over 25 thousand issues
over 2 years, only around 10\% are actually discussed~\cite{GovTrack2022}.
That would be approximately 1000 to 2500 issues per Congress, or about 500 to 1250 per
year.
While this is still a large number of issues, it is the job of a member of the House
of Representatives to learn about, research, and deliberate about each issue.
Additionally, Miller states `a representative should be an expert, or at least
competent, in each field [on which they are voting]'~\cite{Miller1969}.
This reinforces the idea that a member of the House of Representatives should be, as
their title would suggest, a representative of the people and have the responsibility
to become an expert in the issues they are voting on.
To remove this responsibility from the House of Representatives would be to remove
a large portion of their duties, and could easily result in the dictatorship of a few.
As such, we differ from Miller in the sense that all voters ought to be experts in
the field, and so we use proxy voting to allow them to be more efficient in their
duties and avoid spreading disease instead of reworking the system entirely.
Additionally, Miller did not consider using proxy voting for use by members of
Congress as it currently works, which we will explore in this paper.

\etal{Cohensius}~\cite{Cohensius2017} explore the use of proxy voting in a metric space
using three voting mechanisms: mean, median, and majority.
They discovered proxy voting using any of these mechanisms generally produces lower
error than direct voting with active voters alone.
This is not too surprising: reintroducing information lost through inactive voters by
using a proxy system ought to help the system.
Nevertheless, they were able to show that proxy voting is effective under a number of
symmetrical and asymmetrical preference distributions, while under both random and
strategic participation.
However, the majority of their research focuses on voting with infinite populations.
While this work would certainly be applicable to larger populations, since a
population of sufficient size will begin to behave like an infinite
population~($\lim_{x \rightarrow \infty} x = \infty$), we are more interested in the
effects of proxy voting on a relatively small, finite population of 435 members of the
House of Representatives.
As such, we will explore the effects of proxy voting on a finite population of this
size, as well as explore other possible voting mechanisms.
\vicki{List them.  This seems more interesting than the finite set.}

This study will additionally look at voting mechanisms and attempt to determine which
mechanism is best suited for proxy voting.
This is not unlike~\cite{Mathur2017}, who looked at several voting mechanisms applied
on a single-winner election vote and determined a ranking for these mechanisms.
Our work will differ significantly from theirs, however, as we will explore voting
mechanisms used under proxy voting instead of single-winner elections, and most of
our mechanisms will be different due to some incapability of the mechanisms the \vicki{they?} used
on a continuous preference space.
\vicki{Talk about their results here.}

We also look at how allowing a voter to delegate multiple proxies affects the vote,
as described towards the end of \autoref{sec:setup-and-assumptions}.
\vicki{In this section,  you lead with the results of others, and make your intended changes a following comment.  For example, Joe used Borda voting with population X to discover that it was subject to Y. yada yada yada.   We adapt his method to allow proxy voting.}
Jonas Degrave~\cite{Degrave2014} implements a simple model to allow voters to
delegate to multiple proxies, specifically by simply dividing their weight amongst
all those to which the voter delegates.
This is precisely the model we will use, since it allows for a straightforward way to
delegate voting power that would not be confusing to voters.
However, we additionally look at how many proxies a voter should be allowed to
delegate.
Intuitively, if a voter were to delegate all other voters as proxies the result would
be the same as if the voter had simply not voted.
However, if the voter were to only delegate a single proxy, the result might not be
as ideal to the delegator as if they had been able to delegate to two proxies that
would produce a better result for the delegator.
As such, we ask, if voters can delegate more than one proxy, how many proxies should
they be allowed to delegate?
\vicki{Is it "should" or how many are useful?  I'm guessing that more than 3 proxies is not that helpful.  I'm assuming the vote would be split equally?  Otherwise, it really isn't interesting as they could get their exact value.

More citations of previous work would be better. Even if you barely mention their accomplishments, it gives you credibility.}

% TODO: Outline methods for performing analysis and test design


\section{Proposed Work}\label{sec:contribution}
% \section{Contribution}\label{sec:contribution}
This study aims to explore the impact the use of proxies has on a vote,
particularly in the House of Representatives, where high accuracy and participation
is required.
We examine how much of an effect in terms of error proxy voting has on the outcome
of a vote under a number of distributions of opinions, as well as with various
numbers of delegators.
\vicki{give more details as to what you are proposing}
We additionally employ a continuous space preference model to represent agents
preferences, and introduce the ability allow agents to express their preferences as
any number within the range of this model.
The error of a proxy vote will be calculated by measuring the distance in this space
between the proxy vote outcome, the non-proxy vote outcome, as well as comparing what
would happen if only the active agents, meaning the agents who would not use a
proxy, voted.
Employing this model will allow for a more granular measurement of error, as well as
allow the agents to be more expressive with their preferences.

We additionally experiment with several voting mechanisms in order to determine if
proxy votes should be handled differently from normal voting, as well as observe at
what ratios between delegates and delegators such mechanisms are effective.
This paper hopes to identify the best voting mechanism for proxy voting, as well as
determine if proxy voting is a viable technique to increase system welfare without
having too large an impact on the outcomes of a vote.

Finally, we explore how many proxies any given voter should be allowed to delegate.
The purpose of this exploration is to determine if delegating multiple proxies can
help produce a more accurate result by providing more information to the system, and
at what point this effect begins to decrease or hurt the system.

% TODO: Fill out synopsis of what is learned after the data is analysed
% This paper will show that, while proxy vote systems are not a perfect tool to
% increase a system's accuracy, they may be beneficial when the distribution of error
% from a measurement is asymmetrical.
% Additionally, this paper will identify the best performing voting and weighting
% mechanisms, as well as discuss an ideal range of proxy and inactive agents to be used
% in such a system.
