%
%  This document contains chapter 1 of the thesis.
%

\chapter{INTRODUCTION}\label{ch:introduction}
%%%%%%%% This line gets rid of page number on first page of text
\thispagestyle{empty}
%%%%%%%%%%%%%

On May 13th, 2020, the 116th United States Congress introduced a resolution
to permit the use of proxy voting during emergencies for members of the House of
Representatives~\cite{Congress.gov2020}.
This resolution was passed on May 15th, 2020, and as indicated by proxy designation
letters sent to the Clerk of the House of Representatives~\cite{Clerk.House.gov2020},
was put to use as soon as five days later.
The hope behind the resolution was to allow members of the House of Representatives
to vote remotely, in order to prevent the spread of COVID-19~\cite{Congress.gov2020}.
However, the question remains as to what degree proxy voting affects the results of
the vote.
In this study, we explore the effects of proxy voting on the results of the vote
under a number of opinion distributions.
We additionally explore how different voting rules affect the results of the vote.

% TODO: State we assume fairly ideal conditions where an inactive voter can choose
% their proxy per-topic.
% TODO: Outline
% Discuss my assumptions
%   - Per-topic proxy selection
%   - Perfect knowledge of potential proxy's positions
%       - This might need to be discussed after the readers have an understanding of
%         the model.
%  - No other factors (e.g. party affiliation)
% Discuss opinion model (Cohensius)
% Background
%   - Proxy voting
%       - Why it is desirable
%   - What are voting rules
% Contribution

In order to conduct this study, we make a number of assumptions.
First, we assume that each voter is able to choose their proxy per-topic.
This is not currently the case for the House of Representatives, which requires
a proxy to be chosen by letter and so would be difficult to change as different
topics are discussed~\cite{Congress.gov2020}.
However, since it would be fairly simple to implement a new process that allows
per-topic proxies, and to give proxy voting its best chance we believe this assumption
is reasonable.

Second, we assume that each voter has perfect knowledge of potential proxies' opinions.
This will allow them to choose the proxy that has the most similar opinion to their own.
While in reality voters will likely not have perfect knowledge of others' opinions,
it is often not too hard to gauge the opinion of others, particularly those with whom
an individual often associates, and so we believe this assumption is reasonable.

Finally, we assume that there are no factors besides closeness in opinion that affect
the choice of proxy.
For example, we assume a voter would rather choose a proxy for a specific topic who
closely aligns with their opinion on that topic but has a different party affiliation
rather than choose a proxy who has the same party affiliation but a different opinion.

As part of this study, we employ a model developed by \etal{Cohensius} in
their 2017 article~\cite{Cohensius2017}.
This model places voters' preferences, which for our purposes are the congress
members' opinions on a topic, in a metric space \systemspace, such as
in~\autoref{fig:system-metric-space}.
Modeling voters' preferences in such a way allows for continuous preferences instead of
discrete or binary opinions other models might emphasize.
This is beneficial, as it allows for a more realistic representation of voters'
opinions, since it is unlikely that voters' opinions are perfectly binary.
For example, it is likely that some members are very passionate for the
use of proxy voting in the House, while others are passionately against it, and yet
others remain ambivalent or less-passionately for or against.
This works perfectly with the metric space model, as it allows for the passionate
opinions to be at opposite extremes of the metric space, while the less-passionate
opinions are closer to the center.

\begin{figure}[htbp]
    \centering
    % Built using:
    % https://tex.stackexchange.com/a/148253/277236
    % https://tex.stackexchange.com/a/380491/277236
    \begin{tikzpicture}[scale=1.5]
        \draw(0,0) -- (10,0) ; % Axis
        \foreach \x in  {0, 10} % Vertical higher lines
        \draw[shift={(\x,0)},color=black] (0pt,5pt) -- (0pt,0pt);
        \foreach \x in {0, 5, 10} % Numbers and lower lines
        \draw[shift={(\x,0)},color=black] (0pt,0pt) -- (0pt,-5pt) node[below]{$\x$};

        % Labeled points
        \tkzDefPoint((-4/7 + 1) * 5, 0){agentA}
        \tkzDefPoint((3/4 + 1) * 5, 0){agentB}
        \tkzDefPoint((1/12 + 1) * 5, 0){agentC}
        \tkzLabelPoint[above](agentA){$\truthof{a}$}
        \tkzLabelPoint[above](agentB){$\truthof{b}$}
        \tkzLabelPoint[above](agentC){$\truthof{c}$}

        \foreach \n in {agentA, agentB, agentC}
        \node at (\n)[circle,fill,inner sep=1.75pt]{};
    \end{tikzpicture}

    \caption{Example of a 1D preference metric space, where \truthof{x} represents the
    preference of agent $x$.}
    \label{fig:system-metric-space}
\end{figure}

This model also naturally extends to two- or higher-dimensional spaces, such as
the opinions of a congress member on multiple topics.
However, as previously mentioned, we will only consider the idealistic
situation where a congress member can choose their proxy for each topic individually.
The model also allows for votes on topics that are not binary, such setting the
budget for the year;
instead of simply voting yes or no on some pre-chosen budget, the output of the system
can serve as the budget.

% Explain what Cohensius did
%   - Strategic participation
%   - Infinite voters
%   - Voting Mechanisms
% Explain what I'm borrowing and what I'm doing differently
%   - Compulsory use of proxies
%   - Finite voters
%   - Additional voting mechanisms
%   - Weighting mechanisms

This study also makes use of \textit{voting mechanisms} or \textit{voting rules},
which are functions that map a set of preferences in~\systemspace\ to an outcome that
also exists in~\systemspace.
%
% \com{The idea of a voting mechanism did NOT start with Cohensius. Give credit to the
% proper originator.}
% MDH: I've done some research to find the originator of the term, but it seems to be
% a fairly common term with no clear originator. The idea at the very least seems to
% date back all the way to ancient Greece, so I'm hoping it's safe to assume it's a
% common knowledge term.
%
This study employs the use of several common voting mechanisms and adapts them to the
proxy voting system.
What these mechanisms are and how they operate are described
in~\autoref{subsec:voting-mechanisms}.

This paper will explore how proxy voting affects the vote and under what
circumstances it can be used by the House without changing the result.
This will be determined by measuring by how much the system output changes when using
proxy voting from all members voting and only present members voting.
This will be done using continuous outputs instead of binary `yea' or `nay' votes,
since this will allow for more precise measurements in change.


\section{Background}\label{sec:background}
\textit{Proxy voting} involves allowing agents to transfer their voting power
from themselves to another agent, known as a \textit{proxy}.
This adds to the voting power of the proxy, which is called its `weight.'
By delegating a proxy, the transferring agent loses their ability to directly affect the
voting game, instead allowing the proxy to affect it on their behalf.
In other words, the transferring agent does not vote in an election or other types of
voting games, but rather affects the election vicariously through the proxy to which
they transferred their power.
These transferring agents are also known as \textit{inactive voters}, and the
proxies can also be called \textit{active voters}.
In contrast, \textit{direct voting} requires each agent to vote on their own.

Proxy voting has a number of advantages over direct voting.
The most obvious advantage is that it allows inactive voters to reduce the work
required of them by choosing a proxy to perform the work of voting for them.
Naturally, this is extremely useful when the cost of voting is high, such as
when remaining educated on a topic is difficult or
time-consuming~\cite{Mueller1972} uncomfortable work such as standing in
long queues is required, or other frustrating or costly obstacles must be overcome.
For our purposes it is likely undesirable that the agents, being members of Congress,
do not pay the cost of remaining educated, but proxy voting can still be beneficial
to the agents because it would allow them to prevent the spread of disease or avoid
whatever other emergency triggered the use of proxy voting.

In these cases, proxy voting allows voters to skip incurring the costs while
still having their voice heard by allowing a proxy to perform the work once on
behalf of all voters who transfer their power to that proxy.
Of course, a voter would not want to give its voting power to just any proxy.
This study employs a similar technique as used in~\cite{Cohensius2017}, which has the
voters, or agents, pass their voting power to the proxy that is closest in the
preference space.

In addition to reducing individual agents' work, proxy voting also often has
the effect of increasing the system accuracy as opposed to only allowing active
agents to vote~\cite{Cohensius2017}.
This follows intuition: if a voter doesn't vote, the system loses information.
By allowing them to still influence the voting game through proxy, some information
is reintroduced into the system.
Naturally, in terms of system accuracy the ideal situation is when all voters
participate.
However, system welfare can be enhanced on an individual level by allowing agents
who want to be inactive to become inactive.
When operating under the Congress resolution, system welfare can also be enhanced
through proxy voting by allowing ill members to quarantine and ensure the health
of other agents.
As will be discussed in \autoref{ch:results}, this additional system welfare comes at
the cost of some system accuracy since not all information can be restored.

In some cases, the proxy is also able to transfer their own vote, as well as the rest of
their delegated weight, to yet another proxy.
This is known as \textit{liquid democracy}, which has its own challenges and
advantages.
However, simple proxy voting, where proxies cannot transfer their voting power,
is sufficient for the research performed in this paper.

James Miller~\cite{Miller1969} explored the idea of proxy voting as a more
direct method of democracy over representative democracy.
Most of his work focuses on the political and societal advantages of proxy
voting, and so is not directly applicable to the problem at hand.
However, he does bring up the idea this paper calls \textit{expert proxies},
those being individuals who would `vote as [the inactive voter] would if only
[the inactive voter] had the time and knowledge to participate
directly'~\cite[para.~1.3]{Miller1969}.
This concept is exactly what this paper plans to exploit: create and use expert
proxies when generating a measurement and allow less accurate techniques to use
them as proxies.

\etal{Mueller}~\cite{Mueller1972} reaffirms the need for expert proxies and
goes on to discuss how many of such proxies would be needed under random
selection for a sufficiently accurate opinion to be made.
These estimates range from 500 to 1000~\cite[para.~3.2]{Mueller1972}, and
while this study uses preassigned proxies, these proxies will have random
error in their estimations and so \etal{Mueller}'s estimates may come into
play.
However, a primary goal of this paper is to use far fewer measurements than the
quoted 500, as that many measurements would likely be far more expensive than
simply using a single, more costly form of measurement with better accuracy.

However, proxy voting is not without its flaws.
\etal{Gölz} investigated a weakness in liquid democracy dubbed `super voters'
~\cite[para.~1.3]{Golz2021}, which are proxies that receive an extremely large
amount of power.
This can be problematic because the result of a system will be too biased
towards the super voter.  \com{Explain why the super is a problem. It seems part of
the proxy philosophy.
If many voters use proxies,  you aren't going to see the richness of preferences.
This shouldn't have been a surprise.}

For this paper's purposes, a poor distribution of measurements may make one
proxy a super voter and increase the error of the system.
% \vicki{here it appears you know absolute truth, and just want to assure the voting
% finds it.}

% While \etal{Cohensius} primarily examine voting mechanisms under random and strategic
% participation in a proxy vote system, this study will preselect proxies and require
% all other agents to use them.
% The reasoning behind this compulsory proxy system is to allow a system designer to
% choose a suitable set of measurement techniques as the proxies, and reinforce their
% measurements using the other agents.
% %
% \com{Doesn't make sense for the House case Better bit to have this restriction.}


\section{Contribution}\label{sec:contribution}
This study aims to explore the idea of using proxy voting as a technique to remove the
need to take expensive, dangerous, or otherwise undesirable but more accurate
measurements by employing more desirable but less accurate gauging methods when they
are available.
This is accomplished by first taking a number of measurements to serve as proxies, then
supplementing any lack of precision by allocating votes from other measurement
techniques.
\com{This makes total sense to me, in not employing sensors if they appear not to be
needed.
If all the cheap sensors agree, is there really any point in employing an expensive
sensors.
But is this really proxy voting? It seems more like "not voting", as the expensive
sensor has no opinion.}
Additionally, this paper experiments with using different voting mechanisms and
weighting mechanisms in an attempt to optimize proxy voting for different
distributions of error.
This paper also determines if using such mechanisms are even worthwhile, or if it
would be better to skip the proxy voting and simply allow each measurement contribute
to the truth directly.
Finally, this paper attempts to identify how many `proxy' measurements need
to be taken to reduce cost while still ensuring accuracy.

This paper will show that, while proxy vote systems are not a perfect tool to
increase a system's accuracy, they may be beneficial when the distribution of error
from a measurement is asymmetrical.
Additionally, this paper will identify the best performing voting and weighting
mechanisms, as well as discuss an ideal range of proxy and inactive agents to be used
in such a system.


\section{Potential Applications}\label{sec:potential-applications}
Proxy vote systems can be used in situations where highly accurate measurements are
difficult, costly, or time-consuming, and cheaper or easier alternatives are available.
While this study will show they will not work in all situations, they do have a
potential use when one of the alternative measurements tends to yield an asymmetrical
distribution of error.

These systems may also have use in ensemble machine learning techniques, though this
idea is not explored in this study.
