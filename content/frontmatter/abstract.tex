%
%

\begin{abstract}
% A space is needed before the text starts so that the first paragraph
% is indented properly. Max 350 words.

    % TODO: Rewrite
    In May 2020, the 116th United States Congress passed a resolution to permit the
    use of proxy voting during emergencies for members of the House of Representatives.
    Proxy voting is a system of voting where a voter can assign another voter, called
    a proxy, to vote on their behalf.
    As with other voting systems, the votes are then aggregated into a final result
    using a voting rule or mechanism.
    Ideally, the result of proxy voting would be the same, or as close to as
    possible, to the result if everyone were to vote.
    This resolution was almost immediately put into practice and remains active at
    the time of this study.
    While it was designed to reduce the transmission of disease such as COVID-19, if
    proxy voting can be shown to be beneficial, the resolution could be expanded to
    create a more efficient government system.
    However, proxy voting in Congress has been regarded with some skepticism and the
    question remains as to if proxy voting has too great of an effect on the result
    of the vote.
    In this study, we give proxy voting its best chance and explore its effects on
    the results of the vote under a number of opinion distributions.
    We additionally explore how different voting rules affect the results of the vote.
    We employ an opinion space model and determine the result of a vote as a point in
    the model space, instead of the vote simply passing or failing, in order to
    increase the granularity of the results and determine how much error is
    introduced due to proxy voting.
    Finally, we attempt to determine under which circumstances proxy voting should be
    used, if any, and identify strengths and weaknesses of using proxy voting in the
    modern House of Representatives.
    % TODO: Put in a bit of what we discover


\end{abstract}


% Local Variables:
% TeX-master: "newhead"
% End:
