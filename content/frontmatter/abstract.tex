%
%  Time-stamp: "[abstract.tex] last modified by Scott Budge (scott) on 2017-01-10 (Tuesday, 10 January 2017) at 16:54:14 on goga.ece.usu.edu"
%
%  Info: $Id: abstract.tex 998 2017-03-21 16:44:33Z scott $   USU
%  Revision: $Rev: 998 $
% $LastChangedDate: 2017-03-21 10:44:33 -0600 (Tue, 21 Mar 2017) $
% $LastChangedBy: scott $
%

\begin{abstract}
% A space is needed before the text starts so that the first paragraph
% is indented properly. Max 350 words.
% TODO

    A proxy voting system is a system in which agents are able to give their vote to
    other agents, known as proxies.
    These proxies vote on behalf of the agents who gave up their vote.
    The idea behind this type of voting system is it allows agents without the time
    or knowledge necessary to properly become informed on a topic to become
    inactive and pass their vote to a proxy that does.
    The hope is this proxy will be an expert on the topic and so yield a more
    informed end decision.
    This study looks into exploiting this proxy-inactive agent relationship in order
    to increase the accuracy of a measurement system while using less expensive forms
    of measurement.
    Multiple types of voting, dubbed `voting mechanisms,' are identified and tested,
    as well as several ways to allow inactive agents to divide their votes amongst
    proxies named `weighting mechanisms.'
    While it is ultimately discovered that proxy vote systems should not be used in
    all situations, this study is able to identify potential applications for
    employing proxy vote systems as well as lay out potential avenues for additional
    research.

\end{abstract}


% Local Variables:
% TeX-master: "newhead"
% End:
