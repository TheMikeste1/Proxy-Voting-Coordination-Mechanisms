%
%

\begin{abstract}
% A space is needed before the text starts so that the first paragraph
% is indented properly. Max 350 words.

    % TODO: Rewrite
    A proxy voting system is a system in which agents are able to give their vote to
    other agents, known as proxies.
    These proxies vote on behalf of the agents who gave up their vote.
    The idea behind this type of voting system is to allow agents without the time
    %
    \com{Still seems to rely on earlier formulation of the problem. We need a
    situation where voters have a meaningful opinion but are unable to vote. When you
    had to actually go to the polls, it wasn't uncommon for someone to have a
    meaningful opinion but be unable/unwilling to vote. If we could replicate that
    situation where they are able to select a proxy voter, we could justify our study.}
    %
    or knowledge necessary to properly become informed on a topic to become
    inactive and pass their vote to a proxy that does.
    The hope is this proxy will be an expert on the topic and therefore will yield a
    more informed end decision.
    This study investigates the ability of a proposed system to be used as a way to
    replace expensive, but highly accurate, measurements with less expensive,
    though less accurate, measurements through the use of proxy voting by exploiting
    this proxy-inactive agent relationship.
    %
    \com{
        Here's another idea. In the House, people have been voting by proxy because
        they have health issues and don't want the contact. An article I read said
        80\% of members had used that option. What if we study how that practice
        effects voting results? The "truth" is the actual vote if everyone voted. We
        are seeing how close proxy voting comes.}
    %
    Multiple types of voting, dubbed `voting mechanisms,' are developed and tested,
    as well as several ways to allow inactive agents to divide their votes amongst
    proxies named `weighting mechanisms.'
    While it was ultimately discovered that proxy vote systems should not be used in
    all situations, this study was able to identify potential applications for
    employing proxy vote systems as well as lay out potential avenues for additional
    research.

\end{abstract}


% Local Variables:
% TeX-master: "newhead"
% End:
