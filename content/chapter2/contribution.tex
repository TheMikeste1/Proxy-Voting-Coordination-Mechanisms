\section{Contribution}\label{sec:contribution}
We explore proxy voting in a unified-vote/single-winner single-dimension
continuous space model.
We employ three well-known $L_p$ mechanisms, specifically
\begin{enumerate}
    \item {
        Median ($L_1$)
    }
    \item {
        Mean ($L_2$)
    }
    \item {
        Mid-range ($L_\infty$)
    }
\end{enumerate}
Each of these mechanisms will additionally be applied to direct voting with all
agents and direct voting with only those agents that are present.
This is done to show what the output of the system would be with all information
(direct voting with all agents), as well as the output with minimum information
(direct voting with only those agents that are present).
The error will be the distance between the result and direct voting with all agents.

We will additionally apply what we've dubbed `coordination mechanisms,' which are
techniques by which proxies and constituents work together to determine how their
weight will be allocated, meaning how the proxy will vote on their behalf.
The mechanisms we will explore are described
in~\autoref{subsec:coordination-mechanisms}.
These mechanisms are used in an attempt to simulate real-world consequences of proxy
voting, as well as identify potential techniques to deter or mitigate a proxy's
ability to swing the system by aggregating a large amount of weight and abusing it.

Whereas voting on a continuous interval is uncommon and finding a real world dataset
using preferences on an interval currently does not seem possible, these investigations
are performed using preferences generated from several statistical distributions.
We will use various distributions to allow the gathered data to represent
different distributions of voters in the real world.
These distributions include well-known statistical distributions, such as the uniform
distribution, the Gaussian distribution, as well as a few beta distributions.
The distributions used and their notations are listed
in~\autoref{tab:distributions-used}.
% \vicki{Why is normal indifferent?  Define the beta distribution.}
% MDH: The normal distribution can be characterized as indifferent or centralist
% because many of the votes are about the center. In reality, each of these
% distributions could represent any number of different population types; I'm just
% providing potential types so the reader doesn't need to on their own. I've added a
% sentence clarifying this in the table caption.
% The beta distribution is a well-known statistical distribution. I'm using it
% because it can easily create diverse shapes by tweaking its parameters. I've added
% a sentence clarifying these distributions are well-known.
Each experiment will have a population of 24.
For each round, we will experiment with 1 to 23 delegating agents and examine how
error correlates with the number of delegators.  \vicki{Why 23?  Shouldn't it be influenced by the size of the voter pool?}

\begin{table}[!htbp]
    % increase table row spacing, adjust to taste
    \renewcommand{\arraystretch}{1.3}

    \caption{
        The distributions to be used to generate preferences.
        Note how each distribution represents a population type.
        These types are representative, and any distribution could potentially
        represent a different population type that shares the same shape as the
        distribution.
        Additionally, any skewed distributions can be inverted to create a
        distribution that is skewed in the other direction (e.g. a distribution
        skewed in favor can be inverted to create a flipped distribution skewed
        against).
    }
    \label{tab:distributions-used}

    \centering
    \begin{tabular}{|r|l|c|l|}
    \hline
    \thead{Distribution} & \thead{Notation} & \thead{Symmetrical?} & \thead{Population Type}
    \\
    \hhline{|=|=|=|=|}
    Uniform & \uniform{-1}{1} & \ding{51} & Evenly spread
    \\
    \hline
    Normal/Gaussian & \gaussian{0}{\sfrac{1}{3}} & \ding{51} & Mostly
    centrist/indifferent
    \\
    \hline
    Beta(0.3, 0.3) & \betadistribution{0.3}{0.3} & \ding{51} & At either extreme
    \\
    \hline
    Beta(50, 50) & \betadistribution{50}{50} & \ding{51} & Strongly
    centrist/indifferent
    \\
    \hline
    Beta(4, 1) & \betadistribution{4}{1} & \ding{55} & Skewed in favor
    \\
    \hline
\end{tabular}
\end{table}

We will show that proxy voting with the right combination of mechanisms generally
yields considerably lower error than active-only voting.
We also show proxy voting is beneficial even when agents' preferences change.
However, proxy voting appears to be least effective on highly-polarized topics.
