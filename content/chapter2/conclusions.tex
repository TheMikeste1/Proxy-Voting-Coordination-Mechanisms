\section{Conclusions}\label{sec:conclusions}
We have shown proxy voting is able to produce low-error results, even when the
delegating portion of the population is large.
We have additionally shown the Mean voting mechanism with the Mean and Median
coordination mechanisms achieve the lowest error.
These mechanism combinations are generally able to produce results with less than a 5\%
change in outcome when the delegating portion is more than half the total
population and produce considerably less when the delegating portion is lower.

We have additionally shown proxy voting is at its weakest with highly polarized
topics, such as those represented by \betadistribution{0.3}{0.3}.
In these cases, agents should make an extra effort to participate in deliberation and
vote in person instead of by proxy.

However, while proxy voting certainly makes it easier for agents to be represented,
agents should be strongly encouraged or required to participate in deliberations that
may alter their preferences.
When a proxy performs this work on behalf of an agent, the outcome of the vote may
introduce more error than would otherwise be seen with only active voters participating.

Finally, we have shown that proxy voting is a powerful tool that consistently
performs better than not allowing inactive agents to vote.
By employing proxy voting, systems will be able to maintain their accuracy while
increasing the total system utility.
