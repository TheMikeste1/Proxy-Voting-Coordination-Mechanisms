\section{Preliminary Setup}\label{sec:preliminary-setup}

\subsection{The Model}\label{subsec:the-model}
An important part of any study is the model it uses to represent the system being
studied.
We employ a model described by \etal{Cohensius} in their 2017
article~\cite{Cohensius2017}.
This model places voters' preferences in a single-dimension continuous
metric~space~\systemspace, such as in~\autoref{fig:system-metric-space}.
In this model, two points that are close together in the metric space represent
similar preferences, while two points that are far apart represent very different
preferences.
As a point moves further away from the agent's preference, the agent likes it less.
This model works best when an upper and lower bound is provided, such as only
allowing agents to vote in the interval $[-1, 1]$, to prevent agents from voting
extremely far in either direction and so potentially biasing the vote.

The difference between multiple equidistant points in the model may not be equivalent
to the agents.
For example, an agent who must spend \$30 may prefer spending more money rather than
less.
As such, even though \$29 and \$31 are equidistant to \$30, an agent may vastly
prefer \$31 over \$29.
However, for our model, we will assume agents only care about the distance from their
preference, rather than if it is greater or less than some amount.

\begin{figure}[htbp]
    \centering
    % Built using:
% https://tex.stackexchange.com/a/148253/277236
% https://tex.stackexchange.com/a/380491/277236
\begin{tikzpicture}[scale=7.0]
    \draw(-1,0) -- (1,0) ; % Axis
    \foreach \x in {-1, 0, 1} % Numbers and lower lines
    \draw[shift={(\x,0)},color=black] (0pt,2pt) -- (0pt,0pt);
    \foreach \x in {-1, 0, 1} % Numbers and lower lines
    \draw[shift={(\x,0)},color=black] (0pt,0pt) -- (0pt,-2pt) node[below]{$\x$};

    % Labeled points
    \tkzDefPoint((-4/7), 0){agentA}
    \tkzDefPoint((3/4) , 0){agentB}
    \tkzDefPoint((1/12), 0){agentC}
    \tkzLabelPoint[above](agentA){$\truthof{a}$}
    \tkzLabelPoint[above](agentB){$\truthof{b}$}
    \tkzLabelPoint[above](agentC){$\truthof{c}$}

    \foreach \n in {agentA, agentB, agentC}
    \node at (\n)[circle,fill,inner sep=1.75pt]{};
\end{tikzpicture}
    \caption{
        Example of a 1D continuous preference metric space, where \truthof{x} represents
        the preference of agent $x$.
        The x-axis represents some preference space.
        An agent can have a preference anywhere within this space.
        One way to interpret the model is to have the leftmost point be the most
        against some idea and the rightmost point is the most in favor of the same idea.
        Importantly, points towards the center of the space are the most ambivalent,
        neutral, or central on the idea.
    }
    \label{fig:system-metric-space}
\end{figure}

Such a model is extremely flexible and can be interpreted in different ways.
When applied to binary for-or-against voting problems, options can be placed at either
extreme of the interval.
As an example, say a group is choosing whether to put pineapple on pizza.
Using the interval $[-1, 1]$, we can place `no pineapple' at -1, and `yes pineapple'
at 1.
Agents vote according to their preference: -1 for no pineapple, 1 for pineapple.
So far, everything works the same as a normal vote.
However, due to the continuous nature of the voting space, the agents can vote
anywhere in the given interval.
Therefore, agents who don't care one way or the other can vote at 0 instead of being
forced to choose an option.
Additionally, those that are only slightly in favor (meaning they would prefer
pineapple but don't really care that much) can choose some value between 0 and 1.
Once the votes are aggregated, the result can be rounded to whichever option is
closest.
The continuous space allows agents to better express themselves according to what
their actual preference is, instead of being forcibly binned into one value or the
other.

Alternatively, if the majority of the options are about the center, it may be a sign
neither option is satisfactory.
Using the pineapple on pizza example, we can reinterpret 0 to mean the agents don't
mind pineapple on pizza, but don't personally want it.
These agents would vote around 0 to avoid imposing.
By making the agents aware of how 0 will be interpreted, they can express their
dissatisfaction by voting at or around 0.
Normally, voting goes through a bargaining and an enforcement
phase~\cite{Fearon1998}, but if a sufficient number of agents vote close to 0, the
group can reopen discussion about the topic and re-bargain before conducting a new vote.
This enhances the cooperation aspect of voting, and creates a fundamental change:
voting is no longer the end of the process, but rather part of it.

Finally, the continuous space model allows for another, very powerful interpretation:
interpreting the votes and result as continuous values.
For example, imagine Congress is voting on how much to allocate to the defense budget.
In this scenario, Congress can decide to allocate anywhere between \$0 and \$1,000.
\footnote{
    Naturally, these values are not realistic and are meant to be representative.
}
A voter can place their vote anywhere between those values, and the result of the
vote can be how much Congress allocates.
Say there are three voters, each with their own preferences, and all voters are
active (no voter delegates their vote).
Agent $a$ prefers \$750, $b$ prefers \$600, and $c$ prefers \$250.
One way we could aggregate these preferences is by finding their mean.
By averaging these preferences, the system would output
$\frac{\$(750 + 600 + 250)}{3 \text{ voters}} = \frac{\$1600}{3 \text{ voters}} =
\$533.33$,
which would be the amount allocated to the defence budget.
Being able to vote on continuous problems and yield a continuous output, as well as
working with binary issues, makes the model extremely flexible and allows it to
tackle any number of problems.

We are also able to easily calculate error using a continuous space.
The error can simply be the distance from the result of the system when all agents
are active, and the result under proxy voting.

The flexibility of the continuous model in both the discrete and continuous realms, its
ability to use different voting rules, easy interpretability, and easy error
calculation are the reasons it is employed in this study.
We will focus primarily on the continuous instead of the discrete output of the
model, since observing the continuous output allows for more granularity in the
differences between proxy and non-proxy voting, as well as in the differences between
voting rules.

\subsection{Voting mechanisms}\label{subsec:voting-mechanisms}
This study also makes use of \textit{voting mechanisms} or \textit{voting rules},
which are functions that map a set of preferences in~\systemspace\ to an outcome that
also exists in~\systemspace.
For these, we take inspiration from \etal{Bulteau}'s~\cite{Bulteau2021} work in
aggregating one-dimensional single-winner elections by using their $L_p$ aggregation
methods, as well as mixing in plurality.
$L_p$ aggregation methods work by minimizing the sum of distances to the power of $p$
($d^{\,p}$, where $d$ is a distance) between a possible solution and the voters'
preferences.
Naturally, since~\cite{Bulteau2021} did not use weighted proxies, these methods need
to be adjusted to allow for weight.
With \agentweight\ representing the weight of an agent \agent\ and \systemproxies\ as
the ordered set of active voters, the mechanisms we use are
\begin{enumerate}
    \item {
        \textit{Median ($L_1$)}, defined as
        $\textbf{md}(\systemproxies) =
    \min\left\{
        \agent_i \in \systemproxies \text{ s.t. }
            \weightof{\agent_i} +
            \sum_{\agent_j \in \systemproxies}^{\agent_{i - 1}} \weightof{\agent_j}
        \geq \frac{\systemweight}{2}
\right\}$.
        Essentially, the median is the agent whose additional weight makes the sum of
        weights become equal to or greater than half the total weight of the system.
        The sum will occur in the same order as the ordering of voters in
        \systemproxies.
    }
    \item {
        \textit{Mean ($L_2$)}, defined as
        $\mathbf{mn}(\systemproxies) =
    \frac{1}{\systemweight}
    \sum_{\agent_i \in \systemproxies} {\weightof{\agent_i} \cdot \agent_i}$.
        This is a typical weighted average.
    }
    \item {
        \textit{Mid-range ($L_\infty$)}, defined as
        $\mathbf{mr}(\systemproxies) =
    \frac{
        \agent_{\text{lowest}} + \agent_{\text{highest}}
    }
    {2}
% TODO: I'm not so sure this equation is correct. Should it simply be the unweighted version.
%   Consider how mean and median are calculated. They make sense because each agent has a weight of 1. However, unweighted midrange ignores those weights and simple takes the larges and smallest votes. Should the "weighted" version not also ifnore those weights?$, meaning the weighted
        preference of the agent with the lowest preference plus the weighted
        preference of the agent with the highest preference, divided by the sum of
        their weights.
    }
\end{enumerate}
Additionally, we will apply unweighted versions of these mechanisms against all active
voters (to simulate if proxy voting was not allowed), and all voters, inactive and
active (to simulate the best case where all agents vote).
These additional calculations serve as baselines to determine how well proxy voting
works in these scenarios.

\subsection{Coordination Mechanisms}\label{subsec:coordination-mechanisms}
There are also several ways an individual proxy can agree to cast its vote own behalf
of its constituents.
Each has different advantages and disadvantages for the proxy, its constituents, and
the system as a whole.
We will examine five different `coordination' mechanisms:
\begin{enumerate}
    \item {
        \textit{No preference change - Expert}.
        The proxy allocates its weight to its own preference.
    }
    \item {
        \textit{No preference change - Cooperative Mean}.
        The proxy allocates its weight to the mean of its own and its constituents'
        preferences.
    }
    \item {
        \textit{No preference change - Cooperative Median}.
        The proxy allocates its weight to the median of its own and its constituents'
        preferences.
    }
    \item {
        \textit{Preference change - Expert}.
        The proxy allocates its weight to its own preference.
        This is useful for cases when a proxy is deemed by its constituents as an
        `expert', as discussed by James Miller~\cite{Miller1969}.
        However, its preference has changed as it has participated in deliberation
        and discussion with the other active voters.
    }
    \item {
        \textit{Preference change - Expert with Consequences}.
        The proxy is an expert and allocates its weight to its own preference, but its
        preference has changed.
        However, due to this change the proxy is only allowed to apply its own
        weight, leaving its constituents unrepresented.
        This represents situations where a proxy is considered an expert, but the
        constituents impose a condition that the proxy can only represent them if
        they vote a certain way.
    }
\end{enumerate}
These mechanisms are used in an attempt to simulate real-world consequences of proxy
voting, as well as identify potential techniques to deter or mitigate a proxy's
ability to swing the system by aggregating a large amount of weight and abusing it.
