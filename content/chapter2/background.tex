\section{Background}\label{sec:\chptindicator-background}

\subsection{What is proxy voting?}\label{subsec:\chptindicator-what-is-proxy-voting?}
\textit{Proxy voting} is a group of methods by which individuals who are unable or
uninterested in voting in person can still have their voices heard through the use of
a \textit{proxy}, who is an individual authorized to act for another.
Agents are able to delegate another individual to be their proxy.
We call the delegating agent a \textit{delegator} or an \textit{inactive voter}, and
the group of agents that delegate to a proxy its \textit{constituents}.
Proxies are also known as \textit{delegates} or \textit{active voters}.

When voting, each individual starts with a certain number of votes that it can
allocate to any given option.
Typically, this number is one.
The number of votes a voter has are called its \textit{weight} or \textit{voting power}.
For example, an individual with a weight of $n$ who votes for $x$ has the same impact
as if $n$ distinct voters with a weight of 1 picked $x$.
When delegating a proxy, the delegator's weight is transferred to the proxy,
increasing their total weight by the amount of power transferred.
The more weight a voter has, the more they can swing the vote in their favor.

Upon selecting a proxy, a delegator is no longer able to vote directly.
Instead, their delegate votes on their behalf.
In contrast, \textit{direct voting} requires each agent to vote on their own, meaning
each agent must incur the costs of voting or not have their voice heard.
These costs may be tangible, such as needing to pay for gas, or intangible, such as
the time or effort required to vote in-person.
Such costs are common in voting, and are further discussed by~\cite{Gershtein2019}.
Error is introduced into the result of direct voting when an agent is unable or
unwilling to pay these costs, since information is lost when agents do not share
their preference via a vote.
In votes with discrete options, this error could be anything from a different result
(in the worst case) or a slightly different count of the votes (for example, 10
votes in favor instead of 11).
In these cases, such as in times of injury or illness, proxy voting provides an
excellent avenue through which the agent can still have its voice heard and reduce
the error in the system.

Proxy voting is beneficial for the delegates as well.
By working on behalf of their constituents, a proxy has a larger voice in discussions
and deliberations since they are representing more agents.
This allows them to have a larger influence and achieve a greater impact as topics
are debated.

Proxy voting systems are not, however, without flaws.
When agents do not participate in person, they are not able to participate in
deliberation.
Such discussions are vital to the voting process, since as agents confer, they gain
access to new information which may change their preference.
Inactive agents do not benefit from this deliberation; only active agents do.
As such, proxies have to participate in deliberation and change their preference on
behalf of their constituents.
It is possible the proxy will change its preference in such a way that some of its
constituents would not.
If the proxy is only allowed to cast one vote on behalf of all its constituents, those
constituents will have their vote applied in a way they find less preferable than
simply paying the costs.
In this `unified-vote' model, an agent must choose a proxy they are confident would
change their preference in the same way they would `if only [they] had the time and
knowledge to participate directly'~\cite{Miller1969}, or risk having their vote
misallocated.
When a vote is misallocated, error is again introduced into the system, and may
arguably be worse than if the inactive agent hadn't voted at all since it may change
the result of the vote from what it would with all agents participating.

As an alternative to only casting one vote on behalf of all constituents, the proxy
could be allowed to vote once per constituent, and can even be required to vote
precisely as each constituent requests.
This is the system the 116th United States Congress introduced in May 2020 in an
attempt to reduce the risk of COVID-19 in the House of
Representatives~\cite{CERP2020, Congress.gov2020}: the proxies effectively `relay' their
delegators' votes individually and for each inactive voter.
As such, information about the inactive voters' preferences is not lost.
The process is effective in that every voter will have their vote allocated exactly
as they want, but comes with the obvious downside that each proxy has to keep track
of how each individual constituent wants their vote relayed.
This naturally increases the complexity and work required by the proxy.
Additionally, the inactive voters still need to be active in the deliberation process
in order to properly participate in the process.
This makes `relay-voting' only effective for occasions when agents have been able to
participate in the full process except for the voting itself and know exactly how
they want their vote allocated.

In this study, we will tackle some of the problems associated with unified-vote proxy
voting.
By determining methods to minimize the error produced by the model, the system will
yield the benefits of unified-vote deliberation while still producing a near-perfect
result.
