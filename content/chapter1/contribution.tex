\section{Proposed Work}\label{sec:contribution}
% \section{Contribution}\label{sec:contribution} % TODO: Uncomment for the thesis
%%%%%%%%% OLD %%%%%%%%%
This study will explore how proxy voting affects the vote.
We will look at the difference in outcome between a direct vote using all agents that
can vote in person, and proxy voting where the agents that can not attend in person
instead delegate a proxy.
Votes will take place in a continuous metric space, where an agent is able to vote
anywhere inside the interval $[-1, 1]$, and will be aggregated using a voting
mechanism inside the same space.

We will examine proxy voting using four different voting mechanisms.
Specifically, these mechanisms are
\begin{enumerate}
    \item Median ($L_1$)
    \item Mean ($L_2$)
    \item Mid-range ($L_\infty$)
    \item Plurality
\end{enumerate}
Each of these mechanisms will additionally be applied to direct voting with all
agents and direct voting with only those agents that are present.
This is done to show what the output of the system would be with all information
(direct voting with all agents), as well as the output with minimum information
(direct voting with only those agents that are present).
Error will be measured by squared error between each of these baselines.

We will also employ four coordination mechanisms in an attempt to determine ways
proxies and its constituents can work together to enhance their own welfare, as well
as mitigate potential damages that could be caused by a `rogue' proxy.
These mechanisms are
\begin{enumerate}
    \item No preference change - Selfish
    \item No preference change - Cooperative
    \item Preference change - Selfish
    \item Preference change - Consequences
\end{enumerate}
These mechanisms will allow us to identify how a proxy can best represent its
constituents.

Whereas voting on an interval is uncommon and finding a real world dataset using
preferences on an interval currently does not seem possible, these investigations
will be performed using preferences generated from several statistical distributions.
The distributions and their notations are listed in \autoref{tab:distributions-used}.
These various distributions will allow the data gathered to represent situations
where the majority of voters are at either extreme (\betadistribution{0.3}{0.3}),
skewed towards one side (\betadistribution{4}{1}), or mostly indifferent
(\gaussiandist\ and \betadistribution{50}{50}).
Each experiment will have a population of 435, representing the number of members in
the House of Representatives.
For each round, we will experiment with 1 to 434 delegating agents and examine how
error correlates with the number of delegators.

\begin{table}[!htbp]
    % increase table row spacing, adjust to taste
    \renewcommand{\arraystretch}{1.3}

    \caption{
        The distributions to be used to generate preferences.
        Note how each distribution represents a unique population type.
        Additionally, any skewed distributions can be inverted to create a
        distribution that is skewed in the other direction (e.g. a distribution
        skewed in favor can be inverted to create a flipped distribution skewed
        against).
    }
    \label{tab:distributions-used}

    \centering
    \begin{tabular}{|r|l|c|l|}
    \hline
    \thead{Distribution} & \thead{Notation} & \thead{Symmetrical?} & \thead{Population Type}
    \\
    \hhline{|=|=|=|=|}
    Uniform & \uniform{-1}{1} & \checkmark & Evenly spread
    \\
    \hline
    Normal/Gaussian & \gaussian{0}{\sfrac{1}{3}} & \checkmark & Mostly
    centrist/indifferent
    \\
    \hline
    Beta(0.3, 0.3) & \betadistribution{0.3}{0.3} & \checkmark & At either extreme
    \\
    \hline
    Beta(50, 50) & \betadistribution{50}{50} & \checkmark & Strongly
    centrist/indifferent
    \\
    \hline
    Beta(4, 1) & \betadistribution{4}{1} & & Skewed in favor
    \\
    \hline
\end{tabular}
\end{table}

% TODO: Fill out synopsis of what is learned after the data is analysed
% This paper will show that, while proxy vote systems are not a perfect tool to
% increase a system's accuracy, they may be beneficial when the distribution of error
% from a measurement is asymmetrical.
% Additionally, this paper will identify the best performing voting and weighting
% mechanisms, as well as discuss an ideal range of proxy and inactive agents to be used
% in such a system.
