\section{Proposed Work}\label{sec:contribution}
% \section{Contribution}\label{sec:contribution} % TODO: Uncomment for the thesis
We will explore proxy voting in a unified-vote/single-winner single-dimension
continuous space model.
We employ three well-known $L_p$ mechanisms, specifically
\begin{enumerate}
    \item {
        \textit{Median ($L_1$)}
    }
    \item {
        \textit{Mean ($L_2$)}
    }
    \item {
        \textit{Mid-range ($L_\infty$)}
    }
\end{enumerate}
Each of these mechanisms will additionally be applied to direct voting with all
agents and direct voting with only those agents that are present.
This is done to show what the output of the system would be with all information
(direct voting with all agents), as well as the output with minimum information
(direct voting with only those agents that are present).
Error will be measured the distance of the result to each of these baselines.

We will additionally apply with what we've dubbed `coordination mechanisms,' which
are techniques by which proxies and constituents work together to determine how their
weight will be allocated, meaning how the proxy will vote on their behalf.
These mechanisms are
\begin{enumerate}
    \item {
        No preference change - Expert
    }
    \item {
        No preference change - Cooperative Mean
    }
    \item {
        No preference change - Cooperative Median
    }
    \item {
        Preference change - Expert
    }
    \item {
        Preference change - Expert with Consequences
    }
\end{enumerate}
These mechanisms are used in an attempt to simulate real-world consequences of proxy
voting, as well as identify potential techniques to deter or mitigate a proxy's
ability to swing the system by aggregating a large amount of weight and abusing it.

Whereas voting on in a continuous interval is uncommon and finding a real world dataset
using preferences on an interval currently does not seem possible, these investigations
will be performed using preferences generated from several statistical distributions.
The distributions and their notations are listed in \autoref{tab:distributions-used}.
These various distributions will allow the data gathered to represent situations
where the majority of voters are at either extreme (\betadistribution{0.3}{0.3}),
skewed towards one side (\betadistribution{4}{1}), or mostly indifferent
(\gaussiandist\ and \betadistribution{50}{50}).
Each experiment will have a population of 435, representing the number of members in
the House of Representatives.
For each round, we will experiment with 1 to 434 delegating agents and examine how
error correlates with the number of delegators.

\begin{table}[!htbp]
    % increase table row spacing, adjust to taste
    \renewcommand{\arraystretch}{1.3}

    \caption{
        The distributions to be used to generate preferences.
        Note how each distribution represents a unique population type.
        Additionally, any skewed distributions can be inverted to create a
        distribution that is skewed in the other direction (e.g. a distribution
        skewed in favor can be inverted to create a flipped distribution skewed
        against).
    }
    \label{tab:distributions-used}

    \centering
    \begin{tabular}{|r|l|c|l|}
    \hline
    \thead{Distribution} & \thead{Notation} & \thead{Symmetrical?} & \thead{Population Type}
    \\
    \hhline{|=|=|=|=|}
    Uniform & \uniform{-1}{1} & \ding{51} & Evenly spread
    \\
    \hline
    Normal/Gaussian & \gaussian{0}{\sfrac{1}{3}} & \ding{51} & Mostly
    centrist/indifferent
    \\
    \hline
    Beta(0.3, 0.3) & \betadistribution{0.3}{0.3} & \ding{51} & At either extreme
    \\
    \hline
    Beta(50, 50) & \betadistribution{50}{50} & \ding{51} & Strongly
    centrist/indifferent
    \\
    \hline
    Beta(4, 1) & \betadistribution{4}{1} & \ding{55} & Skewed in favor
    \\
    \hline
\end{tabular}
\end{table}

% TODO: Fill out synopsis of what is learned after the data is analysed
% This paper will show that, while proxy vote systems are not a perfect tool to
% increase a system's accuracy, they may be beneficial when the distribution of error
% from a measurement is asymmetrical.
% Additionally, this paper will identify the best performing voting and weighting
% mechanisms, as well as discuss an ideal range of proxy and inactive agents to be used
% in such a system.
