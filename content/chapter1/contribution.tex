\section{Contribution}\label{sec:contribution}
We explore proxy voting in a unified-vote/single-winner single-dimension
continuous space model.
We employ three well-known $L_p$ mechanisms, specifically
\begin{enumerate}
    \item {
        Median ($L_1$)
    }
    \item {
        Mean ($L_2$)
    }
    \item {
        Mid-range ($L_\infty$)
    }
\end{enumerate}
Each of these mechanisms will additionally be applied to direct voting with all
agents and direct voting with only those agents that are present.
This is done to show what the output of the system would be with all information
(direct voting with all agents), as well as the output with minimum information
(direct voting with only those agents that are present).
The error will be the distance between the result and direct voting with all agents.

We additionally apply what we've dubbed `coordination mechanisms,' which are
techniques by which proxies and constituents work together to determine how their
weight will be allocated, meaning how the proxy will vote on their behalf.
The mechanisms we explore are described
in~\autoref{subsec:coordination-mechanisms}.
These mechanisms are used in an attempt to simulate real-world consequences of proxy
voting, as well as identify potential techniques to deter or mitigate a proxy's
ability to swing the system by aggregating a large amount of weight and abusing it by
misallocating it towards a vote the constituents don't prefer.
% 05/05/2023: \vicki{
%   Not clear what the abuse refers to.
%   They got the weight fairly.
%   Is the abuse in threatening to change the preference for all?
%   Not all aggregation methods would permit abuse???
% }
% I've added something to specify "abuse" means "misallocating."
% Some voting mechanisms (which I assume is what is meant by "aggregation methods")
% may not permit abuse, but inactive voters don't participate in the voting
% mechanisms; they vote through the proxy.
% This means the coordination mechanism needs to discourage the proxy from abusing
% (misallocating) its constituents votes.
%
% If "aggregation methods" was supposed to mean "coordination mechanisms," the
% methods that don't permit abuse only apply to when the proxy and its constituents
% are choosing their group preference, but wouldn't help when the proxy actually goes
% to cast the vote. To combat misallocation abuse, we introduce the "with
% consequences" coordination mechanisms.

Whereas voting on a continuous interval is uncommon and finding a real world dataset
using preferences on an interval currently does not seem possible, these investigations
are performed using preferences generated from several statistical distributions.
We use various distributions to allow the gathered data to represent
different distributions of voters in the real world.
These distributions include well-known statistical distributions, such as the uniform
distribution, the Gaussian distribution, as well as a few beta distributions.
The distributions used and their notations are listed
in~\autoref{tab:distributions-used}.

\begin{table}[!htbp]
    % increase table row spacing, adjust to taste
    \renewcommand{\arraystretch}{1.3}

    \caption{
        The distributions to be used to generate preferences.
        Note how each distribution represents a population type.
        These types are representative, and any distribution could potentially
        represent a different population type that shares the same shape as the
        distribution.
        Additionally, any skewed distributions can be inverted to create a
        distribution that is skewed in the other direction (e.g. a distribution
        skewed in favor can be inverted to create a flipped distribution skewed
        against).
    }
    \label{tab:distributions-used}

    \centering
    \begin{tabular}{|r|l|c|l|}
    \hline
    \thead{Distribution} & \thead{Notation} & \thead{Symmetrical?} & \thead{Population Type}
    \\
    \hhline{|=|=|=|=|}
    Uniform & \uniform{-1}{1} & \ding{51} & Evenly spread
    \\
    \hline
    Normal/Gaussian & \gaussian{0}{\sfrac{1}{3}} & \ding{51} & Mostly
    centrist/indifferent
    \\
    \hline
    Beta(0.3, 0.3) & \betadistribution{0.3}{0.3} & \ding{51} & At either extreme
    \\
    \hline
    Beta(50, 50) & \betadistribution{50}{50} & \ding{51} & Strongly
    centrist/indifferent
    \\
    \hline
    Beta(4, 1) & \betadistribution{4}{1} & \ding{55} & Skewed in favor
    \\
    \hline
\end{tabular}
\end{table}

Initial experiments show larger populations behave similarly to those with fewer.
As such, each experiment will have a population of 24.
We chose this number because prior experimentation shows it to be a good balance
between having few enough agents that an agent going inactive can change the
results, while not so few that an agent going inactive is catastrophic.
Since having no delegates or all delegates would not be interesting, we will
experiment with 1 to 22 inactive agents and examine how error correlates with
the number of delegators.
We choose 1 inactive agent as the minimum because the case with 0 agents is not
interesting due to there not being any proxies.
% 05/05/2023: % \vicki{I thought you did consider 0?}
% No, I never have. As we've discussed, having 0 inactive agents is the same as not
% using proxy voting and is not interesting. This plan was specified in the proposal.
% Instead, I was asked to extend my graphs to include 0 (which I did by actually
% running the simulations with 0 inactive agents).
Similarly, we chose 22 as the maximum because the case with 24 agents means there are
not active agents to vote, and the case with 23 simply devolves into the coordination
mechanism replacing the voting mechanism.

For each experiment, agents will randomly select some preference using the current
distribution.
Afterward, one agent
% \vicki{sounds like various numbers of inactive voters are used.
% Can you state that initially?}
% MDH: This is stated in the previous paragraph. I've reworded it slightly to make it
% more clear.
will become inactive and choose the closest active agent to be their proxy.
Following this, the proxies and their constituents will apply a coordination
mechanism, and each voting mechanism will be applied.
The next coordination mechanism will be used, then each voting mechanism will be
applied again.
Once all coordination mechanisms have been used, the number of inactive agents will
increase 22 agents are inactive.
% 05/05/2023: \vicki{Earlier you said 22 is the maximum}
% Whoops, that was a leftover from when I was running up to 23. I've fixed it to be
% correct.
Finally, agents will select a new preference from the next distribution and the
process will begin again.
This will continue until all possible combinations have been used.  \vicki{For each experiment, are preferences and inactive voters re-selected?  The way you discuss it, it sounds like the steps are additive.  If they aren't there is no need to describe it as a progression. 
 You should also state that each experiment was run 1024 times.}

We will show that proxy voting with the right combination of mechanisms generally
yields considerably lower error than active-only voting.
We also show proxy voting is beneficial even when agents' preferences change.
However, proxy voting appears to be least effective on highly-polarized topics.
