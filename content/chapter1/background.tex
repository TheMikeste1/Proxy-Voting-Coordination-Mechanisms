\section{Background}\label{sec:background}
In order to understand the analysis performed in this study, we must first understand
what proxy voting is and how it works.

In a given voting scenario, every individual starts with one vote.
This vote can be called the voter's \textit{weight} or \textit{voting power}, meaning
each agent starts with a weight of one.
The more weight a voter has, the more they can swing the vote in their favor.
\textit{Proxy voting} involves permitting agents to transfer their voting power
from themselves to another agent, known as a \textit{proxy} or a \textit{delegate}.
These transferring agents are also known as \textit{inactive voters},
\textit{delegators} or \textit{delegating agents}, and the proxies can also be called
\textit{active voters}.
Transferring voting power adds to the voting power of the proxy, increasing its
weight by the amount of power transferred to it.
The group of delegators that transfer their power to the same proxy we'll call its
\textit{constituents}.
By delegating a proxy, the transferring agent loses their ability to directly vote,
instead allowing the proxy to vote on their behalf.
In other words, the transferring agent does not vote in an election or other types of
votes, but rather affects the election vicariously through the proxy to which they
transferred their power.
This allows the delegating agent to skip incurring any costs voting without a proxy
may cause.

In contrast, \textit{direct voting} requires each agent to vote on their own, meaning
each agent must incur the costs of voting.
Proxy voting has a number of advantages over direct voting.
The most obvious advantage is that it permits delegators to reduce the work and costs
required to participate in a vote by choosing a proxy to perform the work of voting
for them.
Naturally, this is extremely useful when the cost of voting is high, such as
when one is ill or fears they may become ill, or when a member's time can be spent
more efficiently without being physically present for a vote such as when attending a
conference or engaging in active research in some location outside {Washington,~D.C}.
Costs such as these are common in voting scenarios, and are further discussed
in~\cite{Gershtein2019}.
Not being required to be present while still being able to participate is, of course,
directly advantageous for members of the House of Representatives, since proxy voting
would allow them to prevent the spread of disease or avoid whatever other emergency
triggered the use of proxy voting.

There are many ways a proxy could cast their vote on behalf of their constituents.
For example, the proxy could talk to its constituents to see what would make each of
them the most happy and determine an option the would best represent the group as a
whole.
Alternatively, the proxy might decide to selfishly enhance its own voice by voting its
own preference with all the weight allocated to it.
\vicki{Calling it selfish seems too strong. The inactive voters are asking a favor.
It seems weird to say, "Please represent me, but alter your vote to be more what I
want."
}
Such an act might have consequences, such as the proxy no longer being allowed to
represent its constituents and so losing its additional weight.
We will call these different techniques `coordination mechanisms.'
Naturally, requiring the proxy to coordinate with its constituents increases the work
of the proxy, but can provide benefits for the systems in terms of accuracy, as well
as in terms for the proxy by allowing them to aggregate more weight and so allow
their voice to be stronger.
We will explore several of these methods and their impacts in our analysis.

Proxy voting allows voters to skip incurring the costs while still having their voice
heard by allowing a proxy to pay the cost once on behalf of all voters who transfer
their power to that proxy.
However, this advantage comes with the disadvantage that the proxies will likely not
vote in the exact same way as their constituents, meaning the delegators' weights may
not be applied exactly how they want.
The proxy could, of course, confer with its constituents in order determine a vote
that would make them all happy.
However, the proxy could also vote selfishly by voting using only their own preference.
How proxies coordinate with their constituents has a large impact on both the welfare
and the accuracy of the system, and so we will examine proxy voting under both
selfish and cooperative scenarios.

Similarly, proxy voting can be beneficial for the proxies.
By serving as a proxy, an agent can potentially swing a vote more in its favor.
Serving as a proxy also allows the agent to have a stronger voice during
deliberation by showing how many other agents agree with it, or at least how many
have similar opinions.

Since a proxy can vote how it wants, a voter would not want to give its voting power to
just any proxy, but rather one that would still be `close enough' to their preference.
Similarly, a proxy would not want to be a delegate for a voter who does not share a
similar preference, since if the proxy were to attempt to cooperate that voter may
vastly change how the proxy can vote.
There is room for strategy in that a voter could delegate to someone further away in
order to bring the results closer to their opinion.\footnote{
    For example, if an agent that leans only slightly left knows the majority was going
    to vote far left in the preference space, the agent could delegate to someone who
    leans towards the right.
    This delegation would go against the agents desires, but would pull the result of
    the vote more towards the agent's true preference, thereby yielding higher
    utility for that agent.
}\vicki{But this isn't really a function of proxy voting. This is just strategic
voting. They could/would do the same thing if they were physically present.}
These questions on strategy in proxy voting are important, but in order to first
determine if proxy voting would even be useful, this study employs a similar technique
as used in~\cite{Cohensius2017}, which has the voters pass their voting power to the
proxy that is closest in the preference space without any other strategic reasoning.

The `goodness' of a system here is measured using two primary metrics: 1) the
\textit{error} of the system, meaning how close the system is to the optimal
solution, and 2) the total system \textit{welfare}.
Welfare is a measure of how good, or bad, a situation or action is for an agent.
For example, an agent not being required to vote in person increases that agent's
welfare since it makes it easier for the agent to vote.
Conversely, if the agent is required to incur all the costs of voting, such as when
using direct voting, the agent's welfare is lower due to the increased work the agent
needs to do.
Total system welfare is the sum of the welfare of all agents in the system.
An excellent system would have high accuracy and high welfare, while a poor system
would have low accuracy and low welfare.
In regard to proxy voting, the excellent system would maximize the use of proxies
without degrading the accuracy of the system, while the poor would require all
proxies to vote directly and still have high error.
\vicki{Carried to extreme, however, everyone would vote remotely. This discounts the
valuable discussion that happens in meetings.}

Proxy voting has been shown to increase system accuracy when compared to only
allowing active/willing agents to vote~\cite{Cohensius2017}.
This follows intuition: if a voter doesn't vote, the system loses information.
By allowing them to still influence the voting game through proxy, some information
is reintroduced into the system.
Naturally, in terms of system accuracy the ideal situation is when all voters
participate and so the system has the most information possible.
However, system welfare can be enhanced on an individual level by allowing agents
who want to be inactive through the use of a proxy to delegate their vote.
When operating under the Congress resolution, system welfare can be enhanced through
proxy voting by allowing ill members to quarantine and ensure the health of other
agents.
As will be discussed in the results section of the thesis,
% \autoref{ch:results},  % TODO: Uncomment for thesis
this additional system welfare comes at the cost of some system accuracy in terms of
the actual preferences of the voters.
This cost comes about because proxies will likely not vote exactly the same as their
constituents due to having different preferences.
\vicki{If this were the real problem, we could just make proxies vote exactly as
their constituents desired. They could just vote for each one individually. {.2, .3,
.33, .5}. To me the problem is that votes change with discussion, so the proxy is
entrusted to make decisions as amendments and new issues are raised.
%
Before going into specific examples, give us the structure for how proxies operate.
%
Aggregation: mean, median, lp aggregation, plurality
%
Voting: continuous space, finite set of options
    (Note, we are assuming even a finite set of options are ordered. This isn't
    always the case, right? If my vote is between Sarah, Joe, and John, there may not
    be a linear ordering that all agree to.
    %
    Proxy: averages votes of constituents. Votes their own preference.
}

This causes information about the delegators' preferences to be lost and the system's
accuracy to suffer.
The loss in accuracy may be acceptable, however, if the system's welfare is sufficiently
improved.
Perhaps more importantly, proxy voting can also increase system accuracy when a
sufficient number of agents are unable to vote in person, since it allows the system
to recover some lost information by allowing agents to delegate their vote to a proxy
with a similar preference.

% TODO: Move to Future Work section once I have one
% In some cases, the proxy is also able to transfer their own vote, as well as the
% rest of
% their delegated weight, to yet another proxy.
% This is known as \textit{liquid democracy}, which has its own challenges and
% advantages.
% Ultimately, the use of liquid democracy is not considered in this study, but since
% proxy
% voting is a subset of the liquid democracy system, it would likely be possible to
% extend the results of this study to liquid democracy as well.  \vicki{This could be
% described in a future work section, but seems to have no purpose here.}

There are many ways to represent the votes, or preferences, of agents in a system.
Once such way is a continuous space model, such as the one used in~\cite{Cohensius2017}.
This model places voters' preferences in a metric space \systemspace, such as
in~\autoref{fig:system-metric-space}.
In this model, two points that are close together in the metric space represent
similar preferences, while two points that are far apart represent very different
preferences.

\begin{figure}[htbp]
    \centering
    % Built using:
% https://tex.stackexchange.com/a/148253/277236
% https://tex.stackexchange.com/a/380491/277236
\begin{tikzpicture}[scale=7.0]
    \draw(-1,0) -- (1,0) ; % Axis
    \foreach \x in {-1, 0, 1} % Numbers and lower lines
    \draw[shift={(\x,0)},color=black] (0pt,2pt) -- (0pt,0pt);
    \foreach \x in {-1, 0, 1} % Numbers and lower lines
    \draw[shift={(\x,0)},color=black] (0pt,0pt) -- (0pt,-2pt) node[below]{$\x$};

    % Labeled points
    \tkzDefPoint((-4/7), 0){agentA}
    \tkzDefPoint((3/4) , 0){agentB}
    \tkzDefPoint((1/12), 0){agentC}
    \tkzLabelPoint[above](agentA){$\truthof{a}$}
    \tkzLabelPoint[above](agentB){$\truthof{b}$}
    \tkzLabelPoint[above](agentC){$\truthof{c}$}

    \foreach \n in {agentA, agentB, agentC}
    \node at (\n)[circle,fill,inner sep=1.75pt]{};
\end{tikzpicture}
    \caption{
        Example of a 1D preference metric space, where \truthof{x} represents the
        preference of agent $x$.
        The x-axis represents some preference space, where the leftmost point is
        the most preference most against some idea and the rightmost point is the most
        preference most in favor of the same idea.
        Importantly, points towards the center of the space are more ambivalent
        \vicki{ambivalent seems too strong. According to the dictionary: it means you
        have contradictory or mixed feelings about it. Couldn't you just like the
        centrist approach?  }  about
        the topic or prefer a more moderate approach than the extremes on either end.
    }
    \label{fig:system-metric-space}
\end{figure}

Votes are aggregated into an output using a \textit{voting mechanisms} or
\textit{voting rules}.
Two common voting mechanisms are the \textit{plurality} and \textit{mean} voting rules.
The mean mechanism inherently works to a continuous space, since it can take all
preferences, multiply them by their weights, and then average them.
Plurality  \vicki{Define first, before adapting.}, on the other hand, needs to be
adapted to operate in a continuous space.
In our implementation, we will treat plurality as if the proxies themselves were
candidates.
This is similar to the framework in~\cite{Bulteau2021} treating each preference as a
proposal.
As such, the plurality mechanism will select the preference of the active voter with
the most weight.
\vicki{This doesn't seem right, as it gives the advantage to proxy voters. On a
continuous space, plurality doesn't really make sense. This makes more sense to me.
If there are three real options (-1, 0, 1), count the number of voters in equal
ranges about these points. Select the option with the most votes.}

Proxy voting may seem like an extremely attractive option for the House of
Representatives, and other organizations, to use.  \vicki{We aren't ready for this
example as we don't know how the proxies deal with a variety of constituent votes.}
However, it is not without its flaws.
As a simple example, consider a vote taking over preference
space $\systemspace = [-1, 1]$ with agent preferences $\truthof{\agent_1} = -1$,
$\truthof{\agent_2} = 0.25$, $\truthof{\agent_3} = 0.5$, $\truthof{\agent_4} = 1$,
$\truthof{\agent_5} = 0.15$.
Under the mean mechanism, which aggregates opinions by simply taking their mean, if
everyone were to vote we would get the actual preference of the
system: $\systemtruth = 0.18$.
However, if $\agent_2$, $\agent_3$, and $\agent_5$ were to become inactive and select
$\agent_4$ as their proxy, the result would be $\systemtruth = 0.6$.
This is visualized in \autoref{fig:voting-example}.
This is an absolute error of 0.42, or 21\% of the entire preference space!
Ideally the output of a system employing proxy voting would be much closer to the
actual preference of the system, but since the center voters do not have a more
central proxy, the system's output is skewed towards the most extreme voters.

\begin{figure}[htbp]
    \centering
    % Built using:
% https://tex.stackexchange.com/a/148253/277236
% https://tex.stackexchange.com/a/380491/277236
\begin{tikzpicture}[scale=7.0]
    \draw(-1,0) -- (1,0) ; % Axis
    \foreach \x in {-1, 0, 1} % Numbers and lower lines
    \draw[shift={(\x,0)},color=black] (0pt,0pt) -- (0pt,-2pt) node[below]{$\x$};

    % Labeled points
    \tkzDefPoint(-1, 0){agent1}
    \tkzDefPoint(0.25, 0){agent2}
    \tkzDefPoint(0.5, 0){agent3}
    \tkzDefPoint(1, 0){agent4}
    \tkzDefPoint(0.15, 0){agent5}
    \tkzLabelPoint[above](agent1){$\truthof{\agent_1}$}
    \tkzLabelPoint[below](agent2){$\truthof{\agent_2}$}
    \tkzLabelPoint[above](agent3){$\truthof{\agent_3}$}
    \tkzLabelPoint[above](agent4){$\truthof{\agent_4}$}
    \tkzLabelPoint[above](agent5){$\truthof{\agent_5}$}

    \foreach \n in {agent1, agent2, agent3, agent4, agent5}
    \node at (\n)[circle,fill,inner sep=1.75pt]{};


    % Actual preference
    \draw[color=blue, line width=0.5mm, dotted]
    (0.18, 0pt) -- (0.18, -3pt);
    \node[color=blue] at (0.18,-4pt) {$\systemtruth_{actual}$};

    % Preference under proxy vote
    \draw[color=orange, line width=0.5mm, dotted]
    (0.6, 0pt) -- (0.6, -3pt);
    \node[color=orange] at (0.6,-4pt) {$\systemtruth_{proxy}$};
\end{tikzpicture}

    \caption{
        An example vote and its results.
        $\textcolor{blue}{\systemtruth_{actual}}$ is the result when everyone votes,
        and $\textcolor{orange}{\systemtruth_{proxy}}$ is when $\agent_2$, $\agent_3$,
        and $\agent_5$ delegate their vote and make $\agent_4$ a super voter.
    }
    \label{fig:voting-example}
\end{figure}

Previous research has also identified problems with proxy voting.
\etal{Kling} and \etal{Gölz} all investigated a weakness in liquid democracy, a
superset to proxy voting, dubbed `super voters'~\cite{Kling2015,Golz2021}, which are
proxies that receive an extremely large amount of power, while others gain very little.
While \etal{Kling} ultimately determined these proxies tend to use their power
wisely, possibly to avoid estranging those voters who delegate their power to the
proxy, there can be situations where super voters can be problematic with one-off
issues.
For example, in the previous situation $\agent_4$ could be considered a super voter.
If they were to change their preference, say after bribery, threat, or even
something benign such as changing their opinion after a debate, the system's output
could change drastically.
While there are methods to help mitigate the effect of super voters, which
\etal{Gölz}~\cite{Golz2021} explore, the amount of error produced by a proxy vote
system, including that of super voters, is precisely what this paper will explore.
