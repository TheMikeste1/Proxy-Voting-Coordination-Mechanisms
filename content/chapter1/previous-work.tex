\section{Previous Work}\label{sec:previous-work}
Proxy voting is a well-discussed topic, and we build off many author's previous work.

Jonas Degrave~\cite{Degrave2014} implemented a simple model to calculate how much
weight a proxy has.
The model even allows for agents to delegate multiple proxies.
He treated proxy delegations as a digraph where nodes are voters and edges represent
delegating proxies, and created two algorithms to determine the weights of each proxy.
The first algorithm constructs a linear equation for each agent, consisting of the
agent's original weight plus the weight of the agents the delegate to it.
The second employs an adjacency matrix.
Ultimately, both algorithms simply sum the total weight allocated to a proxy.
We will employ the first technique, since it allows for a straightforward way to
delegate voting power that would not be confusing to voters.
However, we will not take full advantage of it by disallowing proxies to delegate to
other proxies, as well as only allowing each agent to delegate one proxy.
We implement these limitations so as to not cloud the focus of our study, that being
minimizing the error and enhancing deliberation in
single-vote/single-winner/single-dimension continuous space scenarios.

Related to weight, Zhang~and~Grossi~\cite{Zhang2022} explore treating weight as a
probability instead of a count, in addition to simply transferring voting power to
the proxy.
They assume that any agent is able to delegate to any other agent.
We share this assumption, though we will only focus on transferring voting power
directly instead of treating it as a probability.

Anurita~Mathur~and~Arnab~Bhattacharyya~\cite{Mathur2017} looked at several voting
mechanisms applied on a single-winner election vote and determined a ranking for
these mechanisms.
\textit{Voting mechanisms}, also known as \textit{aggregation mechanisms}, are
algorithms that take the preferences of all active voters and turn them in to the
output of the system.
They apply these mechanisms on a dataset while looking only at datapoints without a
Condorcet winner.
In their work, they say a mechanism `beats' another if it has a larger fraction of
the population prefer its output over the other's output.
They discover that the GT\footnote{
    Presumably meaning `Game Theory.'
}~method~\cite{Rivest2010} beats all others, the Schulze~method~\cite{Schulze2011}
and Minimax voting mechanisms always agree and beat all other mechanisms besides the
GT method, while Borda beats Copeland and Plurality, and Plurality comes in last.
This study will also look at voting mechanisms and attempt to determine which
mechanism is best suited for proxy voting.
Our work will differ significantly from theirs, however, as we will explore voting
mechanisms used in a continuous voting space instead of discrete-space, single-winner
elections.
Additionally, most of our mechanisms will be different due the mechanisms they used
not being beneficial on a continuous preference space or with a small number of
candidates, or not working well with proxy voting.

\etal{Cohensius}~\cite{Cohensius2017} explore the use of proxy voting in a
continuous metric space using three voting mechanisms: mean, median, and majority.
They discovered proxy voting using any of these mechanisms generally produces lower
error than direct voting with active voters alone.
This is not too surprising: reintroducing information lost through inactive voters,
regardless of the method, ought to help the system.
Nevertheless, they were able to show that proxy voting is effective under a number of
symmetrical and asymmetrical preference distributions, while under both random and
strategic participation.
However, the majority of their research focuses on voting with infinite populations.
While this work would certainly be applicable to larger populations, since a
population of sufficient size will begin to behave like an infinite
population\footnote{
    Naturally, $\lim_{x \rightarrow \infty} x = \infty$.
}, we are more interested in more realistic proxy voting in smaller, finite
populations.
Specifically, we arbitrarily choose a finite population of 435 agents, that being the
number of members of the House of Representatives.
As such, we will explore the effects of proxy voting on a finite population of this
size, as well as explore other possible voting mechanisms, inside this continuous
metric space.

\etal{Bulteau}~\cite{Bulteau2021}, develop and experiment with a framework for
aggregating preferences in several space using $L_p$ mechanisms.
$L_p$ aggregation methods work by minimizing the sum of distances to the power of $p$
($d^{\,p}$, where $d$ is a distance) between a possible solution and the voters'
preferences.
These mechanisms are particularly useful because they allow fine-tuning of the
aggregation method by changing $p$.
Specifically, in single-dimension single-winner continuous models, they employ $L_1$
(median),
$L_2$ (mean), and $L_{\infty}$ (mid-range, meaning the point between the highest and
lowest agent preference).
They additionally treat any possible value in the space as a potential output as the
system.
They provide a number of remarks and observations for each of these mechanisms, which
we will be borrowing.
However, they examine these mechanisms with weighted votes, which we will have.
As such, we will adapt each of the mechanisms to use weighted votes so they can work
with proxies, and examine how they operate in the continuous space.

James Miller~\cite{Miller1969} imagined a governmental system utilizing proxy voting
in 1969 as a more direct form of a representative democracy.\footnote{
    That is to say, Miller envisioned a system where individuals could directly vote
    for an issue, or elect a proxy to vote for them.
    Naturally, any democracy that uses proxy voting is a representative democracy,
    since the proxy is representing the delegator.
    Nevertheless, it can be argued that Miller's proposal could provide a more direct
    democracy since a voter can directly vote for an issue is they so choose.
}
His work focuses on reworking the current House and Senate systems entirely by using a
more-directly involved populace, but his ideas can still be relevant under the current
system.
In particular, he introduces the idea we call \textit{expert proxies},
those being individuals who would `vote as [the delegator] would if only
[the delegator] had the time and knowledge to participate directly'~\cite{Miller1969}.
Additionally, Miller states `a representative should be an expert, or at least
competent, in each field [on which they are voting]'~\cite{Miller1969}.
This is true both in the government as well as other situations where decisions are
made by voting.
As such, we consider scenarios where a proxy is, dubbed, at least according to its
constituents, an `expert,' and so the constituents will go with its preference
instead of their own.
