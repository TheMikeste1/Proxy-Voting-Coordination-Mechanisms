\section{Previous Work}\label{sec:previous-work}
Proxy voting is a well-discussed topic, and we build off many author's previous work.

Jonas Degrave~\cite{Degrave2014} implemented a simple model to calculate 
weight of a proxy.
The model even allows for agents to delegate multiple proxies.
He treated delegations as a digraph where nodes are voters and edges represent
a delegation.
% 05/05/2023: \vicki{
%   This sounds like the proxies delegate, but actually they are the
%   ones delegated to.
% }
% Rephrased
He then created two methods to determine the weights of each proxy.
The first method constructs a linear equation for each agent, consisting of the
agent's original weight plus the weight of the agents that delegate to it.
The second employs an adjacency matrix and uses linear algebra to calculate the
solution.
Ultimately, both methods simply sum the total weight allocated to a proxy.
We will employ the first technique, since it allows for a straightforward way to
delegate voting power that would not be confusing to voters.
However, we will not take full advantage of it by disallowing proxies to delegate to
other proxies, as well as only allowing each agent to delegate one proxy.
Such proxy-to-proxy delegation is known as \textit{liquid democracy}, which has its own
challenges and advantages.
Unfortunately, it can have the effect that delegates might not know who wields their
voting power, and so we will focus on enhancing deliberation in
unified-vote/single-winner/single-dimension continuous space scenarios.

Related to weight, Zhang~and~Grossi~\cite{Zhang2022} explore treating weight as a
probability instead of a count, in addition to simply transferring voting power to
the proxy.
In the probability model, inactive agents spread their weight out amongst several
proxies.
This probability represents the odds the agent would delegate their full weight to that
proxy.
The probability model allows Zhang~and~Grossi to probabilistically predict the
outcome of a vote, and use that model to see if agents can correctly identify some
world state.
They assume that any agent is able to delegate to any other agent.
We share this assumption, though we will only focus on transferring voting power
directly instead of treating it as a probability.

Anurita~Mathur~and~Arnab~Bhattacharyya~\cite{Mathur2017} looked at several voting
mechanisms applied on a single-winner election vote and determined a ranking for
these mechanisms.
\textit{Voting mechanisms}, also known as \textit{aggregation mechanisms}, are
algorithms that take the preferences of all active voters and turn them into the
output of the system.
% 05/05/2023 :\vicki{
%   Won't you really use it as the mechanism by which a proxy
%   aggregates the votes of its constituents, rather than the way the whole systems
%   selects a winner?
% }
% No. What you are describing is a coordination mechanism. Additionally, I'm
% describing how they used it.
They apply these mechanisms on a dataset while looking only at datapoints without a
Condorcet winner.
A Condorcet winner is a candidate that wins in every possible one-on-one election
against other candidates.
These types of winners are important because they are ubiquitously preferred over
all other candidates in an election, and their absence means there is not always a
clear winner.
In their work, they say a mechanism `beats' another if it has a larger fraction of
the population prefer its output over the other's output.
They discover that the maximal lottery rule, also known as the
Game~Theory~(GT)~method~\cite{Rivest2010} beats all others, the
Schulze~method~\cite{Schulze2011} and Minimax voting mechanisms always agree and beat
all other mechanisms besides the GT method, while Borda beats Copeland and Plurality,
and Plurality comes in last.
These mechanisms are briefly described in~\autoref{tab:mathur-voting-mechanisms}.
This study will also look at voting mechanisms and attempt to determine which
mechanism is best suited for proxy voting.

Our work will differ significantly from theirs, however, as we will explore voting
mechanisms used in a continuous voting space instead of a discrete-space.
Additionally, our mechanisms will be different from the ones they employed since
their mechanisms either do not work in a continuous preference space, or do not work
well with proxy voting.
% \vicki{
%   This discussion simplifies the differences between voting methods such as
%   manipulability or anomalies which occur.
% }

\begin{table}[htbp]
    % increase table row spacing, adjust to taste
    \renewcommand{\arraystretch}{1.3}

    \caption{
        Definitions for the voting mechanisms used by~\cite{Mathur2017}.
        $n$ represents the number of candidates for some vote.
    }
    \label{tab:mathur-voting-mechanisms}

    \centering
    \begin{tabular}{| r | p{0.80\linewidth} |}
    \hline
    \thead[r]{Voting \\ Mechanism} & \thead[l]{Description}  \\
    \hhline{|=|=|}
    Borda & {
        Agents rank each candidate, 1 to $n$.
        The candidate ranked first receives $n - 1$ points, the candidate in second
        $n - 2$, etc.
        Whichever candidate receives the most points total wins.
    } \\
    \hline
    Copeland & {
        Agents rank candidates 1 up to $n$.
        Agents are able to leave some blank, all of which will be counted as last place.
        After all agents have ranked the candidates, they are compared pairwise.
        If a candidate is more often ranked better than another, it gets one point.
        If they are more often ranked worse, they get 0 points.
        When the number of ranked better vs ranked worse is equal, the candidate
        receives half a point.
        The candidate with the largest score wins.
    } \\
    \hline
    GT & {
        Agents rank each candidate, and a pairwise margin matrix is generated.
        In the matrix, position $(x, y)$ is the number of rankings that prefer $x$ over
        $y$, minus the number of rankings that prefer $y$ over $x$.
        A zero-sum game is defined, and an optimal mixed strategy computed.
        % \vicki{explain the zero sum}
        A zero-sum game is a situation in which for every point gained by one side,
        the other side looses and equal number of points.
        The winner is then chosen randomly using the optimal mixed strategy.
        See~\cite{Rivest2010} for more details.
    } \\
    \hline
    Minimax & {
        Agents rank every candidate from 1 to $n$.
        Candidates are compared pairwise, where candidate $x$ is compared to
        candidate $y$.
        The score of each comparison is the number of votes candidate $y$ receives
        more than candidate $x$.
        Then, the maximum score for each candidate $x$ (meaning, its largest
        pairwise defeat) is determined.
        The candidate with the lowest (or minimum) maximum score is selected as the 
        winner.
    } \\
    \hline
    Plurality & {
        Each agent selects their favorite candidate.
        The candidate with the most votes wins.
    } \\
    \hline
    Schulze & {
        Agents rank candidates, and are allowed to skip rankings, give the same rank
        twice, or not rank a candidate.
        Candidates with the same ranking are said to be equally preferred, and
        unranked candidates are ranked last.
        Then, a weighted directed graph connecting agents is generated.
        An edge going from candidate $A$ to candidate $B$ with a weight of 10 means
        10 more agents prefer $A$ over $B$ than $B$ over $A$.
        Using this graph, multiple paths are generated between candidates.
        The strength of the path is the sum of the weight of each edge in the path,
        and the path with the highest strength from $A$ is compared to the strongest
        path from $B$.
        The same occurs with paths from $A$ to $C$, $B$ to $C$, etc.
        The candidate that has stronger paths to each individual agent than all other
        agents wins.
    } \\
    \hline
\end{tabular}

\end{table}
% \vicki{
%     In Schulze, unclear why we need a path.
%     Isn't there a direct link between any two candidates?
%     with GT, not clear what is meant by a "zero-sum" game is then applied.
% }
% MDH: I'm not using these mechanisms, so I'm not sure it's beneficial to
% explain them more in detail. In reality, each mechanism could take up an entire
% page with mechanisms an explanations, so I've left it to the authors of those papers
% (Schulze and Rivest) to explain them.
%
% Schulze does have a direct link between any candidate, but a direct link is not
% necessarily the strongest path. I've made some changes to hopefully clarify.

\etal{Cohensius}~\cite{Cohensius2017} explore the use of proxy voting in a
continuous metric space using two voting mechanisms: mean, median.
They also explore proxy voting in a binary space using majority.
They discovered proxy voting using any of these mechanisms generally produces lower
error than direct voting with active voters alone.
This is not too surprising: reintroducing information lost through inactive voters,
regardless of the method, ought to help the system.
Nevertheless, they were able to show that proxy voting is effective under a number of
symmetrical and asymmetrical preference distributions, while under both random and
strategic participation.
However, the majority of their research focuses on voting with infinite populations.
While our work would certainly be applicable to larger populations, since a
population of sufficient size will begin to behave like an infinite population\footnote{
    Naturally, $\lim_{x \rightarrow \infty} x = \infty$.
}, we are more interested in more realistic proxy voting in smaller, finite populations.
As such, we will explore the effects of proxy voting on a finite population, as well
as explore other possible voting mechanisms, inside this continuous metric space.

\etal{Bulteau}~\cite{Bulteau2021}, develop and experiment with a framework for
aggregating preferences in several spaces using $L_p$ mechanisms.
$L_p$ aggregation methods work by minimizing the sum of distances to the power of $p$
($d^{\,p}$, where $d$ is a distance) between a possible solution and the voters'
preferences.
These mechanisms are particularly useful because they allow fine-tuning of the
aggregation method by changing $p$.
Specifically, in single-dimension single-winner continuous models, they employ $L_1$
(median),
$L_2$ (mean), and $L_{\infty}$ (mid-range, meaning the point between the highest and
lowest agent preference).
They additionally treat any possible value in the space as a potential output as the
system.
They provide a number of remarks and observations for each of these mechanisms, which
we will be borrowing.
However, they examine these mechanisms without weighted votes, which we will have.
As such, we will adapt each of the mechanisms to use weighted votes so they can work
with proxies, and examine how they operate in the continuous space.

James Miller~\cite{Miller1969} imagined a governmental system utilizing proxy voting
in 1969 as a more direct form of a representative democracy.\footnote{
    That is to say, Miller envisioned a system where individuals could directly vote
    for an issue, or elect a proxy to vote for them.
    Naturally, any democracy that uses proxy voting is a representative democracy,
    since the proxy is representing the delegator.
    Nevertheless, it can be argued that Miller's proposal could provide a more direct
    democracy since a voter can directly vote for an issue if they so choose.
}
His work focuses on reworking the current House and Senate systems entirely by using a
more directly involved populace, but his ideas can still be relevant under the current
system.
In particular, he introduces the idea we call \textit{expert proxies},
those being individuals who would `vote as [the delegator] would if only
[the delegator] had the time and knowledge to participate directly'~\cite{Miller1969}.
Additionally, Miller states `a representative should be an expert, or at least
competent, in each field [on which they are voting]'~\cite{Miller1969}.
This is true both in the government as well as other situations where decisions are
made by voting.
As such, we consider scenarios where a proxy is dubbed, at least according to its
constituents, an `expert,' and so the constituents will go with its preference
instead of their own.
