\section{Preliminary Setup and Assumptions}\label{sec:setup-and-assumptions}

% Our Model
%   - Continuous space
%       - Intepretation of results
%       - Error
%   - Voting mechanisms
%   - Cooperation mechanisms
% Assumptions
An important part of any study is the model it uses to represent the system being
studied.
We employ a model described by \etal{Cohensius} in their 2017
article~\cite{Cohensius2017}.
This model places voters' preferences in a single-dimension continuous
metric~space~\systemspace, such as in~\autoref{fig:system-metric-space}.
In this model, two points that are close together in the metric space represent
similar preferences, while two points that are far apart represent very different
preferences.
This model works best when an upper and lower bound is provided, such as only
allowing agents to vote in the interval $[-1, 1]$.

\begin{figure}[htbp]
    \centering
    % Built using:
% https://tex.stackexchange.com/a/148253/277236
% https://tex.stackexchange.com/a/380491/277236
\begin{tikzpicture}[scale=7.0]
    \draw(-1,0) -- (1,0) ; % Axis
    \foreach \x in {-1, 0, 1} % Numbers and lower lines
    \draw[shift={(\x,0)},color=black] (0pt,2pt) -- (0pt,0pt);
    \foreach \x in {-1, 0, 1} % Numbers and lower lines
    \draw[shift={(\x,0)},color=black] (0pt,0pt) -- (0pt,-2pt) node[below]{$\x$};

    % Labeled points
    \tkzDefPoint((-4/7), 0){agentA}
    \tkzDefPoint((3/4) , 0){agentB}
    \tkzDefPoint((1/12), 0){agentC}
    \tkzLabelPoint[above](agentA){$\truthof{a}$}
    \tkzLabelPoint[above](agentB){$\truthof{b}$}
    \tkzLabelPoint[above](agentC){$\truthof{c}$}

    \foreach \n in {agentA, agentB, agentC}
    \node at (\n)[circle,fill,inner sep=1.75pt]{};
\end{tikzpicture}
    \caption{
        Example of a 1D continuous preference metric space, where \truthof{x} represents
        the preference of agent $x$.
        The x-axis represents some preference space.
        An agent can have a preference anywhere within this space.
        One way to interpret the mosel is to have the leftmost point be the most
        against some idea and the rightmost point is the most in favor of the same idea.
        Importantly, points towards the center of the space are the most ambivalent,
        neutral, or central on the idea.
    }
    \label{fig:system-metric-space}
\end{figure}

Such a model is extremely flexible and can be interpreted in different ways.
When applied to binary for-or-against voting problems, options can be placed at either
extreme of the interval.
For example, say a group is choosing whether or not to put pineapple on pizza.
Using the interval $[-1, 1]$, we can place `no pineapple' at -1, and `yes pineapple'
at 1.
Agents vote according to their preference: -1 for no pineapple, 1 for pineapple.
So far, everything works the same as with a normal vote.
However, due to the continuous nature of the voting space and that agents can vote
anywhere in the given interval.
Therefore, agents who don't care one way or the other can vote at 0 instead of being
forced to choose an option.
Additionally, those that are only slightly in favor can choose some value in between
0 and 1.
Once the votes are aggregated, the result can be rounded to whichever option is
closest.
The continuous space allows agents to better express themselves according to what
their actual preference is, instead of being forcibly binned into one value or the
other.

Alternatively, if the majority of the options are about the center, it may be a sign
neither option is satisfactory.
Using the pineapple on pizza example, we can reinterpret 0 to mean the agents want
pineapple on the pizza, but perhaps feel the amount proposed is too much.
By making the agents aware of how 0 will be interpreted, they can express their
dissatisfaction by voting at or around 0.
Normally, voting goes through a bargaining and an enforcement
phase~\cite{Fearon1998}, but if a sufficient number of agents vote close to 0, the
group can reopen discussion about the topic and re-bargain before conducting a new vote.
This enhances the cooperation aspect of voting, and creates a fundamental change:
voting is no longer the end of the process, but rather part of it.

Finally, the continuous space model allows for another, very powerful ability: voting
on continuous problems directly.
For example, say Congress is voting on how much to allocate to the defense budget.
Congress can decide to allocate anywhere between \$0 and \$1,000 (naturally, these
values are not realistic and are meant to be representative).
A voter can place their vote anywhere between those values, and the result of the
vote will how much Congress allocates.
Say there are three voters, each with their own preferences, and all voters are
active (no voter delegates their vote).
Agent $a$ prefers \$750, $b$ prefers \$600, and $c$ prefers \$250.
By averaging these preferences, the system would output
$\frac{\$(750 + 600 + 250)}{3 \text{ voters}} = \frac{\$1600}{3 \text{ voters}} =
\$533.33$, which would be the amount allocated to the defence budget.
Being able to vote on continuous problems and yield a continuous output, as well as
working with binary issues, makes the model extremely flexible and allows it to
tackle any number of problems.



%%%%%%%%% OLD %%%%%%%%%
Modeling voters' preferences in such a way allows for continuous preferences instead of
discrete or binary opinions other models might emphasize.
This is beneficial, as it allows for a more realistic representation of voters'
opinions, since it is unlikely that voters' opinions are perfectly binary.
For example, it is likely that some members are passionately in favor of higher
spending on education, while others are passionately against it, and yet others
remain ambivalent or less-passionately for or against.
This works perfectly with the metric space model, as it allows for the passionate
opinions to be at opposite extremes of the metric space, while the less-passionate or
more moderate opinions are closer to the center.


This model also naturally extends to two- or higher-dimensional spaces.
These would include more complex and multi-faceted topics, such as migration, which
would include subtopics such as border restrictions and economic impact, or where to
delegate funds in the United States Budget.
However, as previously mentioned, we will only consider the idealistic situation
where a congress member can choose their proxy for each topic individually.
The model obviously allows for votes on topics that are continuous, such as setting the
budget for the year: instead of simply voting yes or no on some pre-chosen dollar
amount for a budget, the output of the system can serve as the budget amount.
However, discrete and binary issues can be supported too, and there are multiple
ways of interpreting the output of the model for such issues.
For example, one could round the output to the nearest valid choice, such as a 1 or a
0 for binary issues.

The hope behind such a system is to gain a better understanding of the voters'
true preference, which is not possible with binary or discrete votes due to all votes
ultimately being binned into one of several values.
This system will allow one to see if the majority is truly in favor of some idea, or
if they are only slightly in favor of it.
For example, consider a situation where voters are asked to vote on some new law and
are able to vote in the interval of $[-1, 1]$.
If the majority of votes are hovering around the center, say in the interval
$[-0.25, 0.25]$, then it is likely that the majority of voters are not actually fully
satisfied by the law, but still believe some change is necessary.
In other words, neither option is satisfactory.
\vicki{Here you are assuming that the middle option isn't really a choice. There are
only two options. Since we don't know which options are really possible, this is
confusing. Try considering the cases (binary, continuous) separately.  }
This provides a third option: refactoring the law to be more in line with the
voters' opinions.
Such an option is extremely beneficial, since without it voters are encouraged to
vote in the extremes.
This is because anything less than an all-for or all-against vote would be diluted by
those who did vote in the extremes, making the voice of the centrist voters less potent.
With the refactoring option, however, if the population is truly ambivalent to the
topic they would be encouraged to vote towards the center in order to show they want
to rework the law.
This reopens discussion and allows for a more educative process, with the hope that
as more discussion occurs the population will become more informed and result in a
better law.

In addition to facilitating an additional, neutral option on binary issues,
continuous interval voting allows for better voting on continuous matters.


This study also makes use of \textit{voting mechanisms} or \textit{voting rules},
which are functions that map a set of preferences in~\systemspace\ to an outcome that
also exists in~\systemspace.
For these, we take inspiration from \etal{Bulteau}'s~\cite{Bulteau2021} work in
aggregating one-dimensional single-winner elections by using their $L_p$ aggregation
methods, as well as mixing in plurality.
$L_p$ aggregation methods work by minimizing the sum of distances to the power of $p$
($d^{\,p}$, where $d$ is a distance) between a possible solution and the voters'
preferences.
Naturally, since~\cite{Bulteau2021} did not use weighted proxies, these methods need
to be adjusted to allow for weight.
With $w(a)$ representing the weight of an agent $a$ and \systemproxies\ as the ordered
set of active voters, the mechanisms we use are
\begin{enumerate}
    \item {
        \textit{Median ($L_1$)}, defined as
        $\textbf{md}(\systemproxies) =
    \min\left\{
        \agent_i \in \systemproxies \text{ s.t. }
            \weightof{\agent_i} +
            \sum_{\agent_j \in \systemproxies}^{\agent_{i - 1}} \weightof{\agent_j}
        \geq \frac{\systemweight}{2}
\right\}$.
        Essentially, the median is the agent whose additional weight makes the sum of
        weights becomes equal to or greater than half the total weight of the system.
        The sum will occur in the same order as the ordering of voters in
        \systemproxies.
    }
    \item {
        \textit{Mean ($L_2$)}, defined as
        $\mathbf{mn}(\systemproxies) =
    \frac{1}{\systemweight}
    \sum_{\agent_i \in \systemproxies} {\weightof{\agent_i} \cdot \agent_i}$.
        This is a typical weighted average.
        \vicki{I thought l2 was sum of squares of differences???}
    }
    \item {
        \textit{Mid-range ($L_\infty$)}, defined as
        $\mathbf{mr}(\systemproxies) =
    \frac{
        \agent_{\text{lowest}} + \agent_{\text{highest}}
    }
    {2}
% TODO: I'm not so sure this equation is correct. Should it simply be the unweighted version.
%   Consider how mean and median are calculated. They make sense because each agent has a weight of 1. However, unweighted midrange ignores those weights and simple takes the larges and smallest votes. Should the "weighted" version not also ifnore those weights?$, meaning the weighted
        preference of the of the agent with the lowest preference plus the weighted
        preference of the agent with the highest preference, divided by the sum of
        their weights.
    }
    \item {
        \textit{Plurality}, defined as
        $\textbf{pl}(\systemproxies) =
    \argmax \weightof{\systemproxies}$, simply meaning the
        active voter with the highest weight.
        In the case of a tie, the output will be a simple average of the tied voters.
        This mechanism likely will not yield a lower error than the other mechanisms,
        but serves as a potential lower bound for mechanism accuracy.
    }
\end{enumerate}
Additionally, we will apply unweighted versions of these mechanisms against all active
voters (to simulate if proxy voting was not allowed), and all voters, inactive and
active (to simulate the best case where all agents vote).
These additional calculations serve as baselines to determine how well proxy voting
works in these scenarios.

More details on how each voting mechanism operates will be described in the full thesis.
% in~\autoref{subsec:voting-mechanisms}.  % TODO: Uncomment for thesis

There are also several ways an individual proxy can agree to cast its vote own behalf
of its constituents.
Each has different advantages and disadvantages for the proxy, its constituents, and
the system as a whole.
We will examine four of these different `coordination' mechanisms:
\begin{enumerate}
    \item {
        \textit{No preference change - Selfish}. The proxy allocates its weight to
        its own preference.
    }
    \item {
        \textit{No preference change - Cooperative}. The proxy allocates its weight
        to the mean of its and its constituents' preferences.
    }
    \item {
        \textit{Preference change - Selfish}. The proxy allocates its weight to its
        own preference, but its preference has changed.
        This change may be due to additional deliberation or some other cause.
    }
    \item {
        \textit{Preference change - Consequences}. The proxy allocates its weight to
        its own preference, but its preference has changed.
        However, due to this change the proxy is only allowed to apply its own
        weight, leaving its constituents unrepresented.
    }
\end{enumerate}
These mechanisms are used in an attempt to simulate real-world consequences of proxy
voting, as well as identify potential techniques to deter or mitigate a proxy's
ability to swing the system by aggregating a large amount of weight and abusing it.

The flexibility of the continuous model in both the discrete and continuous realms, its
ability to use different voting rules, and its easy interpretability, are the reasons
it is employed in this study.
This study will focus primarily on the continuous instead of the discrete output of
the model, since observing the continuous output allows for more granularity in the
differences between proxy and non-proxy voting, as well as in the differences between
voting rules.

\subsection{Assumptions}\label{subsec:assumptions}
In order to conduct this study, we make a number of assumptions.
First, we assume that each voter is able to choose their proxy, also known as a
delegate, individually for each topic.
This is not currently the case for the House of Representatives, which requires
a proxy to be chosen by letter and so would be difficult to change as different
topics are discussed~\cite{Congress.gov2020}.
However, since it would be fairly simple to implement a new process that allows
per-topic proxies, and since many complex topics can be reduced to a set of related
subtopics, we believe this assumption is reasonable.

Additionally, the resolution currently requires the delegate to vote exactly as the
specific instructions provided by the delegating voter tells them to
vote~\cite{CERP2020, Congress.gov2020}.
This essentially turns proxy into a relay for the delegating voter, which is not how
proxy voting is typically used and is not particularly interesting, since relaying a
vote essentially allows the delegator to vote as if they were present.
This means no information about the delegator's preference is lost.
Unfortunately, it also means they do not gain all the benefits of traditional proxy
voting since they still need to do all the work of casting their vote except for
being present.
The `relay' proxy voting also comes with the downside that delegators will be unable
to have their proxy vote differently if the delegator's preference were to change,
say through deliberation or as new information is provided, limiting the system's
adaptability.

Instead, we consider scenarios where actual proxy voting is used, where the proxy
receives the voting power of the delegating voter, increasing their weight, and
proceeds to votes according to their own preference.
Traditional proxy voting also allows the proxy to update their (and by extension, their
constituents) preferences as new information becomes available.
While traditional proxy voting gives substantial flexibility to the proxy to
operate on behalf of their constituents, this flexibility requires the delegating voter
to choose a proxy who they trust to vote as close to how they themselves would.
This is a process similar to selecting experts, as described by~\cite{Miller1969}
and~\cite{Mueller1972}.
By using traditional proxy voting instead of relay-style voting, we hope to exploit
the advantages of proxy voting that the relay-style does not provide.

Second, we assume that each voter has reasonable knowledge of potential proxies'
opinions, meaning they have a decent idea of the preferences of other proxies.
This will allow them to choose the proxy that has the opinion most similar to their own.
While in reality voters will likely not have perfect knowledge of others' opinions,
it is often not particularly difficult to gauge the opinion of others, especially
those with whom an individual often associates, and so we believe this assumption is
reasonable.

Third, we assume that abstention is not allowed.
In other words, all agents must vote either themselves or by proxy.
The primary reasoning behind this is to simplify the system being used and provide
the system with as much information as possible without needing to account for
extenuating circumstances such as the unexpected incapacitation of a voter.
Additionally, all individuals will have some form of opinion, even if that opinion is
completely neutral.
A neutral opinion can be represented as a preference close to the middle of the
preference space, which decreases the desire for abstention when voting is possible.
Nevertheless, as a baseline, we will explore occasions where those who are not
physically present are unable to vote.

Finally, we assume that there are no factors besides closeness in opinion that affect
the choice of proxy.
This differs from the actual system presently used by the House of Representatives,
which includes restrictions such as a proxy can only serve ten voters~\cite{CERP2020}.
However, we feel removing restrictions such as these leads to a more interesting
discussion, since it allows the use of different voting mechanisms and more extreme
cases.