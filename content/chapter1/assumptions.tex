\section{Assumptions}\label{sec:assumptions}
In order to conduct this study, we make a number of assumptions.
First, we assume that each voter is able to choose their proxy, also known as a
delegate, individually for each topic.
This is not currently the case for the House of Representatives, which requires
a proxy to be chosen by letter and so would be difficult to change as different
topics are discussed~\cite{Congress.gov2020}.
However, since it would be fairly simple to implement a new process that allows
per-topic proxies, and since many complex topics can be reduced to a set of related
subtopics, we believe this assumption is reasonable.

Additionally, the resolution currently requires the delegate to vote exactly as the
specific instructions provided by the delegating voter tells them to
vote~\cite{CERP2020, Congress.gov2020}.
This essentially turns proxy into a relay for the delegating voter, which is not how
proxy voting is typically used and is not particularly interesting, since relaying a
vote essentially allows the delegator to vote as if they were present.
This means no information about the delegator's preference is lost.
Unfortunately, it also means they do not gain all the benefits of traditional proxy
voting since they still need to do all the work of casting their vote except for
being present.
The `relay' proxy voting also comes with the downside that delegators will be unable
to have their proxy vote differently if the delegator's preference were to change,
say through deliberation or as new information is provided, limiting the system's
adaptability.

Instead, we consider scenarios where actual proxy voting is used, where the proxy
receives the voting power of the delegating voter, increasing their weight, and
proceeds to votes according to their own preference.
Traditional proxy voting also allows the proxy to update their (and by extension, their
constituents) preferences as new information becomes available.
While traditional proxy voting gives substantial flexibility to the proxy to
operate on behalf of their constituents, this flexibility requires the delegating voter
to choose a proxy who they trust to vote as close to how they themselves would.
This is a process similar to selecting experts, as described by~\cite{Miller1969}
and~\cite{Mueller1972}.
By using traditional proxy voting instead of relay-style voting, we hope to exploit
the advantages of proxy voting that the relay-style does not provide.

Second, we assume that each voter has reasonable knowledge of potential proxies'
opinions, meaning they have a decent idea of the preferences of other proxies.
This will allow them to choose the proxy that has the opinion most similar to their own.
While in reality voters will likely not have perfect knowledge of others' opinions,
it is often not particularly difficult to gauge the opinion of others, especially
those with whom an individual often associates, and so we believe this assumption is
reasonable.

Third, we assume that abstention is not allowed.
In other words, all agents must vote either themselves or by proxy.
The primary reasoning behind this is to simplify the system being used and provide
the system with as much information as possible without needing to account for
extenuating circumstances such as the unexpected incapacitation of a voter.
Additionally, all individuals will have some form of opinion, even if that opinion is
completely neutral.
A neutral opinion can be represented as a preference close to the middle of the
preference space, which decreases the desire for abstention when voting is possible.
Nevertheless, as a baseline, we will explore occasions where those who are not
physically present are unable to vote.

Finally, we assume that there are no factors besides closeness in opinion that affect
the choice of proxy.
This differs from the actual system presently used by the House of Representatives,
which includes restrictions such as a proxy can only serve ten voters~\cite{CERP2020}.
However, we feel removing restrictions such as these leads to a more interesting
discussion, since it allows the use of different voting mechanisms and more extreme
cases.
