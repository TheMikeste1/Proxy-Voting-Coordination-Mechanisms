\section{Assumptions}\label{sec:assumptions}
In order to conduct this study, we make a number of assumptions.
First, we assume agents vote on only one
topic at a time.
This allows agents to select a proxy that works best for the current topic instead of
selecting the best proxy for all topics.

Additionally, we only consider scenarios where unified-vote proxy voting is used, where
the proxy receives the voting power of the delegating voter, increasing their weight,
and makes one selection instead of relaying their constituents' vote.
This type of proxy voting also allows the proxy to update their (and by extension, their
constituents') preferences as new information becomes available.
While unified-vote proxy voting gives substantial flexibility to the proxy to operate
on behalf of their constituents, this flexibility requires the delegating voter to
choose a proxy who they trust to vote as close to how they themselves would.
This is a process similar to selecting experts, as described by~\cite{Miller1969}
and~\cite{Mueller1972}.
By using unified-vote proxy voting instead of relay-style voting, we hope to exploit
the advantages of proxy voting that the relay-style does not provide as described
in~\autoref{sec:background}.

We also assume that each voter has a reasonable knowledge of potential proxies'
opinions.
This will allow them to choose the proxy that has an opinion most similar to their own.
While in reality voters will likely not have perfect knowledge of others' opinions,
it is often not difficult to gauge the opinion of others, especially
those with whom an individual often associates, and so we believe this assumption is
reasonable.

Finally, we assume that there are no factors besides closeness in opinion that affect
the choice of proxy.
This differs from some systems, such as that presently used by the House of
Representatives which includes restrictions such as a proxy can only serve ten
voters~\cite{CERP2020}.
However, we feel removing restrictions such as these leads to a more interesting
discussion, since it allows the use of different voting mechanisms and more extreme
cases.
