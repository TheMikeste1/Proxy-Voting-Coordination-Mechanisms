%
%  This is an example of how a LaTeX thesis should be formatted.  This
%  document contains chapter 2 of the thesis.
%
%  Time-stamp: "[sample-chapter2.tex] last modified by Scott Budge (scott) on 2016-07-28 (Thursday, 28 July 2016) at 08:40:50 on goga.ece.usu.edu"
%
%  Info: $Id: sample-chapter2.tex 967 2016-07-28 15:33:29Z scott $   USU
%  Revision: $Rev: 967 $
% $LastChangedDate: 2016-07-28 09:33:29 -0600 (Thu, 28 Jul 2016) $
% $LastChangedBy: scott $
%

\chapter{CONCLUSIONS AND FURTHER WORK}\label{ch:conclusions-and-further-work}

\section{Summary}\label{sec:summary}


\section{Further Work and Improvements}\label{sec:further-work-and-improvements}
% FIXME: This might need to be moved to Chapter 4
% - Multiple levels of expertise. Expert voters could have varying levels of
% accuracy, allowing for more complex systems.
% - (If I decide to restrict experts and untrained to the same mins and maxes)
% Allow for untrained to have a lower restriction, making them less accurate.
% - Independently track time, monetary cost, etc. for system cost (described
% in the cost section of Criteria. Useful identifying multiple systems for
% different costs.
% Closer look into instances where proxy voting works better than weightlessly averaging
