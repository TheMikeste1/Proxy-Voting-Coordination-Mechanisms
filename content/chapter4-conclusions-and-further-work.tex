%
%  This document contains chapter 4 of the thesis.
%

\chapter{CONCLUSIONS AND FURTHER WORK}\label{ch:conclusions-and-further-work}

\section{Conclusions}\label{sec:conclusions}
Proxy vote systems as a method to increase system accuracy are situational tools.
Assuming the system is able to be as accurate or more accurate than the undesirable
measurement technique, the proxy voting system is best used when one of the
distributions of error is asymmetrical.
Unfortunately, this thus requires the expected distribution of error for the more
desirable measurement techniques to be known.
However, under the correct circumstances, the proxy vote system can be used to increase
the accuracy of a system.
Additionally, not all proxy vote systems are created equally and some combinations of
voting and weighting mechanisms can be more effective than others.
As such, the proxy vote system and the mechanisms used should be chosen after
understanding the problem for which the system is desired to solve.

Furthermore, the ratio between proxies and inactive agents should be chosen well.
As previously shown, all systems, including weightlessly averaging all agents, benefit
from an appropriate number of agents.
As discussed in~\fullref{sec:how-many-agents}, this sweet spot appears to be between
10 and 15 proxies, with slightly more or less inactive agents.
This may cause using such techniques to be too expensive if it is not possible to use
sufficiently cheap proxies and inactive agents.
However, if such agents are available, the proxy vote or weightless average approach
may be able to achieve better performance than alternative techniques that cost more
than the combined sum of the agents used.

Overall, proxy vote systems as a method to increase system accuracy appear to show
promise when used under the proper circumstances.
Further research may be required to identify its full capabilities and weaknesses
before its full potential can be exploited.

\section{Further Work and Improvements}\label{sec:further-work-and-improvements}
There are multiple ways in which the research performed in this study can be
furthered and improved.
Some obvious vectors for additional research include testing additional voting and
weighting mechanisms.
Furthermore, allowing for multiple levels of expertise for both inactive and proxy
agents could yield additional interesting results.

As previously mentioned, proxy vote systems appear to work best when one error
distribution is asymmetrical.
This idea can be further developed and refined to develop more specific rules as to
when proxy vote systems are superior to simply averaging all agents.

There may be potential application for proxy vote systems used in ensemble machine
learning methods where some form of voting already takes place, such as random forests.
Proxy votes may allow for new ensemble methods where some `agents' or trees are more
well-trained and serve as experts, while other agents remain cheaper to train.

Finally, this study assumes all agents, both expert and untrained, have the same
extents, or range of estimates.
In the real world, this is likely not the case--the range of error with less accurate
measurements could very possibly vary as much if not more than the distribution of
error.
An analysis into mixing different extents, in addition to the other parameters used
in this study, may yield additional use cases for proxy vote systems and is where the
author feels the most room for improvement lies.
