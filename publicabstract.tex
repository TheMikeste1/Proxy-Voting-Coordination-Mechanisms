%
%  Time-stamp: "[publicabstract.tex] last modified by Scott Budge (scott) on
%  2011-08-09 (Tuesday, 9 August 2011) at 09:17:43 on goga"
%
%  Info: $Id$   USU
%  Revision: $Rev$
% $LastChangedDate$
% $LastChangedBy$
%

\begin{publicabstract}
% A space is needed before the text starts so that the first paragraph
% is indented properly.

    Illness, injury, and other impediments are common occurrences of everyday life.
    Such impediments prevent or deter voters from participating in important parts
    of the voting process, especially deliberation, bargaining, and the voting
    itself.
    Without participation, the results of the vote may change.
    There is a need to provide a system with which voters are still able to
    participate in important voting processes to ensure their vote is represented.
    We explore `proxy voting,' a system in which voters are able to select another
    individual, or \textit{proxy}, to vote on their behalf.
    By choosing a good proxy, a voter can still have their vote represented and cause minimal change to the outcome.
    We additionally explore different ways a proxy can coordinate with their peers,
    as well as different ways to count votes, in order to make proxy voting as
    effective as possible.
    We show that proxy voting is effective in many scenarios and that it is
    consistently better than not allowing voters who are unable to fully participate to
    have their voices heard.

\end{publicabstract}


% Local Variables:
% TeX-master: "newhead"
% End:
