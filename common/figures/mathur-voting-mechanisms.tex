\begin{tabular}{| r | p{0.80\linewidth} |}
    \hline
    \thead[r]{Voting \\ Mechanism} & \thead[l]{Description}  \\
    \hhline{|=|=|}
    Borda & {
        Agents rank each candidate, 1 to $n$.
        The candidate ranked first receives $n - 1$ points, the candidate in second
        $n - 2$, etc.
        Whichever candidate receives the most points total wins.
    } \\
    \hline
    Copeland & {
        Agents rank candidates 1 up to $n$.
        Agents are able to leave some blank, all of which will be counted as last place.
        After all agents have ranked the candidates, they are compared pairwise.
        If a candidate is more often ranked better than another, it gets one point.
        If they are more often ranked worse, they get 0 points.
        When the number of ranked better vs ranked worse is equal, the candidate
        receives half a point.
        The candidate with the largest score wins.
    } \\
    \hline
    GT & {
        Agents rank each candidate, and a pairwise margin matrix is generated.
        In the matrix, position $(x, y)$ is the number of rankings that prefer $x$ over
        $y$, minus the number of rankings that prefer $y$ over $x$.
        A zero-sum game is defined, and an optimal mixed strategy is computed.
        \vicki{Explain the mixed strategy.}
         A zero-sum game is a situation in which for every point gained by one side,
        the other side loses an equal number of points.
        The winner is then chosen randomly using the optimal mixed strategy.
        See~\cite{Rivest2010} for more details.
    } \\
    \hline
    Minimax & {
        Agents rank every candidate from 1 to $n$.
        Candidates are compared pairwise, where candidate $x$ is compared to
        candidate $y$.
        The score of each comparison is the number of votes candidate $y$ receives
        more than candidate $x$.
        Then, the maximum score for each candidate $x$ (meaning, its largest
        pairwise defeat) is determined.
        The candidate with the lowest (or minimum) maximum score is selected as the 
        winner.
    } \\
    \hline
    Plurality & {
        Each agent selects their favorite candidate.
        The candidate with the most votes wins.
    } \\
    \hline
    Schulze & {
        Agents rank candidates, and are allowed to skip rankings, give the same rank
        twice, or not rank a candidate.
        Candidates with the same ranking are said to be equally preferred, and
        unranked candidates are ranked last.
        Then, a weighted directed graph connecting agents is generated.
        An edge going from candidate $A$ to candidate $B$ with a weight of 10 means
        10 more agents prefer $A$ over $B$ than $B$ over $A$.
        Using this graph, multiple paths are generated between candidates.
        The strength of the path is the sum of the weight of each edge in the path,
        and the path with the highest strength from $A$ is compared to the strongest
        path from $B$.
        The same occurs with paths from $A$ to $C$, $B$ to $C$, etc.
        The candidate that has stronger paths to each individual agent than all other
        agents wins.
    } \\
    \hline
\end{tabular}
