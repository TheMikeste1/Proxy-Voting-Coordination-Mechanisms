%
%

\begin{abstract}
% A space is needed before the text starts so that the first paragraph
% is indented properly. Max 350 words.

    Illness, injury, and other impediments are common occurrences in every day.
    Such impediments prevent or deter agents from participating in important parts
    of the voting process, in particular deliberation, bargaining, and the voting
    itself.
    Without participation, the results of the vote may change.
    There is a need to provide a mechanism by which agents are still able to
    participate in such processes to prevent changing the result of the vote.\vicki{to ensure their vote is represented}
    We examine single-vote/single-winner proxy voting in a one-dimension continuous
    preference space using a combination of $L_p$ aggregation methods.
    As part of this examination, we develop and examine `coordination mechanisms,' by
    which proxies and their constituents are able to compromise
    \vicki{combine?  may not be a compromise} on their preferences in
    order for constituents to still have their voices heard.
    In exchange, their proxy gains more weight and is able to have a stronger voice in
    deliberations.
    We employ a continuous preference space model and determine the result of a vote
    as a point in the model's space instead of the vote simply passing or failing.
    \vicki{Couldn't this also be an option for the design?}
    This increases granularity of the results and determine how much error is
    introduced due to proxy voting, as well as provide a more expressive way of
    casting a vote.
    % TODO: Put in a bit of what we discover


\end{abstract}


% Local Variables:
% TeX-master: "newhead"
% End:
