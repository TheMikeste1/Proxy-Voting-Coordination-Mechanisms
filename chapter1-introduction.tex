%
%  This document contains chapter 1 of the thesis.
%

\renewcommand{\chptindicator}{ch1}

\chapter{INTRODUCTION}\label{ch:introduction}
%%%%%%%% This line gets rid of page number on first page of text
\thispagestyle{empty}
%%%%%%%%%%%%%
From determining the best breakfast cereal~\cite{Curtis2021} to electing the next
president, voting has become an important aspect of modern society.
However, disease, injury, and other impediments can create difficulties for individuals
participating in such democratic processes, preventing them from expressing their
voice and participating in deliberation, as well as decreasing the overall quality of
the result.
Proxy voting is a method by which participants are able to have others vote on their
behalf.
We examine the ability of proxy voting to decrease the impact of such frustrations.
We additionally determine strategies by which proxies and their constituents can
cooperate in order to reduce the change their absence would otherwise cause.
These strategies include allowing proxies and constituents to aggregate their
preference into one result, as well as determining how best to aggregate all votes in
a unified-vote/single-winner/single-dimension continuous space model.

\section{Background}\label{sec:background}  % TODO: Maybe rename to "preliminaries?"

\subsection{What is proxy voting?}\label{subsec:what-is-proxy-voting?}
\textit{Proxy voting} is a group of methods by which individuals who are unable or
uninterested in voting in person can still have their voices heard through the use of
a \textit{proxy}, who is an individual authorized to act for another.
Agents are able to delegate another individual to be their proxy.
We call the delegating agent a \textit{delegator} or an \textit{inactive voter}, and
the group of agents that delegate to a proxy its \textit{constituents}.
Proxies are also known as \textit{delegates} or \textit{active voters}.

When voting, each individual starts with a certain number of votes that it can
allocate to any given option.
Typically, this number is one.
These votes are called a voter's \textit{weight} or \textit{voting power}.
For example, an individual with a weight of $n$ who votes for $x$ has the same effect
as if $n$ separate voters picked $x$.
When delegating a proxy, the delegator's weight is transferred to the proxy,
increasing its weight by the amount of power transferred to it.
The more weight a voter has, the more they can swing the vote in their favor.

Upon selecting a proxy, a delegator is no longer able to vote directly.
Instead, their delegate votes on their behalf in one way or another.
In contrast, \textit{direct voting} requires each agent to vote on their own, meaning
each agent must incur the costs of voting or not have their voice heard.
These costs may be tangible, such as needing to pay for gas, or intangible, such as
the time or effort required to vote in-person.
Such costs are common in voting, and are further discussed by~\cite{Gershtein2019}.
Error is introduced into the result of direct voting when an agent is unable or
unwilling to pay these costs, since information is lost when agents do not share
their preference via a vote.
In votes with discrete options, this error could be anything from a different result
(in the worst case) or a slightly different count of the votes (for example, 10
votes in favor instead of 11).
In these cases, such as in times of injury or illness, proxy voting provides an
excellent avenue through which the agent can still have its voice heard and reduce
the error in the system.

Proxy voting is beneficial for the delegates as well.
By working on behalf of their constituents, a proxy has a larger voice in discussions
and deliberations since they are representing more agents.
This allows them to have a larger influence and achieve a greater impact as topics
are debated.

Proxy voting systems are not, however, without flaws.
When agents do not participate in person, they are not able to participate in
deliberation.
Such discussions are essential to the voting process, since as agents confer, they
gain access to new information which may change their preference.
Inactive agents do not benefit from this deliberation;
only active agents do.
As such, proxies have to participate in deliberation and change their preference on
behalf of their constituents.
It is possible the proxy will change its preference in such a way that some of its
constituents would not.
If the proxy is only allowed to cast one vote on behalf of all its constituents, those
constituents will have their vote applied in a way they do not want.
In this `unified-vote' model, an agent must choose a proxy they are confident would
change their preference in the same way they would `if only [they] had the time and
knowledge to participate directly'~\cite{Miller1969}, or risk having their vote
misallocated.
When a vote is misallocated, error is again introduced into the system, and may
arguably be worse than if the inactive agent hadn't voted at all since it may change
the result of the vote from what it would with all agents participating.

As an alternative to only casting one vote on behalf of all constituents, the proxy
could be allowed to vote once per constituent, and can even be required to vote
precisely as each constituent requests.
This is the system the 116th United States Congress introduced in May 2020 in an
attempt to reduce the risk of COVID-19 in the House of
Representatives~\cite{CERP2020, Congress.gov2020}: the proxies effectively `relay' their
delegators' votes individually and for each inactive voter.
As such, information about the inactive voters' preferences is not lost.
This process is effective in that every voter will have their vote allocated exactly
as they want, but comes with the obvious downside that each proxy has to keep track
of how each individual constituent wants their vote relayed, increasing the
complexity and work required by the proxy.
Additionally, the inactive voters still need to be active in the deliberation process
in order to properly participate in the process.
This makes `relay-voting' only effective for occasions when agents have been able to
participate in the full process except for the voting itself and know exactly how
they want their vote allocated.

In this study, we will tackle some of the problems associated with unified-vote proxy
voting.
By determining methods to minimize the error produced by the model, the system will
yield the benefits of unified-vote deliberation while still producing a near-perfect
result.

%%%%%%%%% OLD %%%%%%%%%
% NOTE: Cooperation strategies
% There are many ways a proxy could cast their vote on behalf of their constituents.
% For example, the proxy could talk to its constituents to see what would make each of
% them the most happy and determine an option the would best represent the group as a
% whole.
% Alternatively, the proxy might decide to selfishly enhance its own voice by voting its
% own preference with all the weight allocated to it.
% \vicki{Calling it selfish seems too strong. The inactive voters are asking a favor.
% It seems weird to say, "Please represent me, but alter your vote to be more what I
% want."
% }
% Such an act might have consequences, such as the proxy no longer being allowed to
% represent its constituents and so losing its additional weight.
% We will call these different techniques `coordination mechanisms.'
% Naturally, requiring the proxy to coordinate with its constituents increases the work
% of the proxy, but can provide benefits for the systems in terms of accuracy, as well
% as in terms for the proxy by allowing them to aggregate more weight and so allow
% their voice to be stronger.
% We will explore several of these methods and their impacts in our analysis.

% NOTE: How voters choose proxies TODO: Move this to our model
% These questions on strategy in proxy voting are important, but in order to first
% determine if proxy voting would even be useful, this study employs a similar technique
% as used in~\cite{Cohensius2017}, which has the voters pass their voting power to the
% proxy that is closest in the preference space without any other strategic reasoning.

% NOTE: `Goodness' TODO: Maybe move this into background?
% The `goodness' of a system here is measured using two primary metrics: 1) the
% \textit{error} of the system, meaning how close the system is to the optimal
% solution, and 2) the total system \textit{welfare}.
% Welfare is a measure of how good, or bad, a situation or action is for an agent.
% For example, an agent not being required to vote in person increases that agent's
% welfare since it makes it easier for the agent to vote.
% Conversely, if the agent is required to incur all the costs of voting, such as when
% using direct voting, the agent's welfare is lower due to the increased work the agent
% needs to do.
% Total system welfare is the sum of the welfare of all agents in the system.
% An excellent system would have high accuracy and high welfare, while a poor system
% would have low accuracy and low welfare.
% In regard to proxy voting, the excellent system would maximize the use of proxies
% without degrading the accuracy of the system, while the poor would require all
% proxies to vote directly and still have high error.
% \vicki{Carried to extreme, however, everyone would vote remotely. This discounts the
% valuable discussion that happens in meetings.}

% NOTE: Cohensius & System welfare
% Proxy voting has been shown to increase system accuracy when compared to only
% allowing active/willing agents to vote~\cite{Cohensius2017}.
% This follows intuition: if a voter doesn't vote, the system loses information.
% By allowing them to still influence the voting game through proxy, some information
% is reintroduced into the system.
% Naturally, in terms of system accuracy the ideal situation is when all voters
% participate and so the system has the most information possible.
% However, system welfare can be enhanced on an individual level by allowing agents
% who want to be inactive through the use of a proxy to delegate their vote.
% When operating under the Congress resolution, system welfare can be enhanced through
% proxy voting by allowing ill members to quarantine and ensure the health of other
% agents.
% As will be discussed in the results section of the thesis,
%% \autoref{ch:results},  % TODO: Uncomment for thesis
% this additional system welfare comes at the cost of some system accuracy in terms of
% the actual preferences of the voters.
% This cost comes about because proxies will likely not vote exactly the same as their
% constituents due to having different preferences.
% \vicki{If this were the real problem, we could just make proxies vote exactly as
% their constituents desired. They could just vote for each one individually. {.2, .3,
% .33, .5}. To me the problem is that votes change with discussion, so the proxy is
% entrusted to make decisions as amendments and new issues are raised.
%%
% Before going into specific examples, give us the structure for how proxies operate.
%%
% Aggregation: mean, median, lp aggregation, plurality
%%
% Voting: continuous space, finite set of options
%     (Note, we are assuming even a finite set of options are ordered. This isn't
%     always the case, right? If my vote is between Sarah, Joe, and John, there may not
%     be a linear ordering that all agree to.
%%
% Proxy: averages votes of constituents. Votes their own preference.
% }
%
% This causes information about the delegators' preferences to be lost and the system's
% accuracy to suffer.
% The loss in accuracy may be acceptable, however, if the system's welfare is
% sufficiently
% improved.
% Perhaps more importantly, proxy voting can also increase system accuracy when a
% sufficient number of agents are unable to vote in person, since it allows the system
% to recover some lost information by allowing agents to delegate their vote to a proxy
% with a similar preference.

% TODO: Move to Future Work section once I have one
% In some cases, the proxy is also able to transfer their own vote, as well as the
% rest of
% their delegated weight, to yet another proxy.
% This is known as \textit{liquid democracy}, which has its own challenges and
% advantages.
% Ultimately, the use of liquid democracy is not considered in this study, but since
% proxy
% voting is a subset of the liquid democracy system, it would likely be possible to
% extend the results of this study to liquid democracy as well.  \vicki{This could be
% described in a future work section, but seems to have no purpose here.}

% NOTE: What are voting mechanisms?
% Votes are aggregated into an output using a \textit{voting mechanisms} or
% \textit{voting rules}.
% Two common voting mechanisms are the \textit{plurality} and \textit{mean} voting
% rules.
% The mean mechanism inherently works to a continuous space, since it can take all
% preferences, multiply them by their weights, and then average them.
% Plurality  \vicki{Define first, before adapting.}, on the other hand, needs to be
% adapted to operate in a continuous space.
% In our implementation, we will treat plurality as if the proxies themselves were
% candidates.
% This is similar to the framework in~\cite{Bulteau2021} treating each preference as a
% proposal.
% As such, the plurality mechanism will select the preference of the active voter with
% the most weight.
% \vicki{This doesn't seem right, as it gives the advantage to proxy voters. On a
% continuous space, plurality doesn't really make sense. This makes more sense to me.
% If there are three real options (-1, 0, 1), count the number of voters in equal
% ranges about these points. Select the option with the most votes.}

% NOTE: Example error introduced by proxy voting
% Proxy voting may seem like an extremely attractive option for the House of
% Representatives, and other organizations, to use.  \vicki{We aren't ready for this
% example as we don't know how the proxies deal with a variety of constituent votes.}
% However, it is not without its flaws.
% As a simple example, consider a vote taking over preference
% space $\systemspace = [-1, 1]$ with agent preferences $\truthof{\agent_1} = -1$,
% $\truthof{\agent_2} = 0.25$, $\truthof{\agent_3} = 0.5$, $\truthof{\agent_4} = 1$,
% $\truthof{\agent_5} = 0.15$.
% Under the mean mechanism, which aggregates opinions by simply taking their mean, if
% everyone were to vote we would get the actual preference of the
% system: $\systemtruth = 0.18$.
% However, if $\agent_2$, $\agent_3$, and $\agent_5$ were to become inactive and select
% $\agent_4$ as their proxy, the result would be $\systemtruth = 0.6$.
% This is visualized in \autoref{fig:voting-example}.
% This is an absolute error of 0.42, or 21\% of the entire preference space!
% Ideally the output of a system employing proxy voting would be much closer to the
% actual preference of the system, but since the center voters do not have a more
% central proxy, the system's output is skewed towards the most extreme voters.
%
% \begin{figure}[htbp]
%     \centering
%     % Built using:
% https://tex.stackexchange.com/a/148253/277236
% https://tex.stackexchange.com/a/380491/277236
\begin{tikzpicture}[scale=7.0]
    \draw(-1,0) -- (1,0) ; % Axis
    \foreach \x in {-1, 0, 1} % Numbers and lower lines
    \draw[shift={(\x,0)},color=black] (0pt,0pt) -- (0pt,-2pt) node[below]{$\x$};

    % Labeled points
    \tkzDefPoint(-1, 0){agent1}
    \tkzDefPoint(0.25, 0){agent2}
    \tkzDefPoint(0.5, 0){agent3}
    \tkzDefPoint(1, 0){agent4}
    \tkzDefPoint(0.15, 0){agent5}
    \tkzLabelPoint[above](agent1){$\truthof{\agent_1}$}
    \tkzLabelPoint[below](agent2){$\truthof{\agent_2}$}
    \tkzLabelPoint[above](agent3){$\truthof{\agent_3}$}
    \tkzLabelPoint[above](agent4){$\truthof{\agent_4}$}
    \tkzLabelPoint[above](agent5){$\truthof{\agent_5}$}

    \foreach \n in {agent1, agent2, agent3, agent4, agent5}
    \node at (\n)[circle,fill,inner sep=1.75pt]{};


    % Actual preference
    \draw[color=blue, line width=0.5mm, dotted]
    (0.18, 0pt) -- (0.18, -3pt);
    \node[color=blue] at (0.18,-4pt) {$\systemtruth_{actual}$};

    % Preference under proxy vote
    \draw[color=orange, line width=0.5mm, dotted]
    (0.6, 0pt) -- (0.6, -3pt);
    \node[color=orange] at (0.6,-4pt) {$\systemtruth_{proxy}$};
\end{tikzpicture}

%     \caption{
%         An example vote and its results.
%         $\textcolor{blue}{\systemtruth_{actual}}$ is the result when everyone votes,
%         and $\textcolor{orange}{\systemtruth_{proxy}}$ is when $\agent_2$, $\agent_3$,
%         and $\agent_5$ delegate their vote and make $\agent_4$ a super voter.
%     }
%     \label{fig:voting-example}
% \end{figure}

% NOTE: Super voters
% Previous research has also identified problems with proxy voting.
% \etal{Kling} and \etal{Gölz} all investigated a weakness in liquid democracy, a
% superset to proxy voting, dubbed `super voters'~\cite{Kling2015,Golz2021}, which are
% proxies that receive an extremely large amount of power, while others gain very
% little.
% While \etal{Kling} ultimately determined these proxies tend to use their power
% wisely, possibly to avoid estranging those voters who delegate their power to the
% proxy, there can be situations where super voters can be problematic with one-off
% issues.
% For example, in the previous situation $\agent_4$ could be considered a super voter.
% If they were to change their preference, say after bribery, threat, or even
% something benign such as changing their opinion after a debate, the system's output
% could change drastically.
% While there are methods to help mitigate the effect of super voters, which
% \etal{Gölz}~\cite{Golz2021} explore, the amount of error produced by a proxy vote
% system, including that of super voters, is precisely what this paper will explore.

\section{Previous Work}\label{sec:previous-work}
James Miller~\cite{Miller1969} imagined a governmental system utilizing proxy voting
in 1969 as a more direct form of a representative democracy.\footnote{
    That is to say, Miller envisioned a system where individuals could directly vote
    for an issue, or elect a proxy to vote for them.
    Naturally, any democracy that uses proxy voting is a representative democracy,
    since the proxy is representing the delegator.
    Nevertheless, it can be argued that Miller's proposal could provide a more direct
    democracy since a voter can directly vote for an issue is they so choose.
}
His work focuses on reworking the current House and Senate systems entirely by using a
more-directly involved populace, but his ideas can still be relevant under the current
system.
In particular, he introduces the idea we call \textit{expert proxies},
those being individuals who would `vote as [the delegator] would if only
[the delegator] had the time and knowledge to participate directly'~\cite{Miller1969}.
While for the general populace this could be a very valuable benefit of proxy voting,
it is not entirely desirable for the House of Representatives.
One of the reasons individuals are elected to the House of Representatives is to
research and create laws that are in the best interest of the people on behalf of
the people.
Though the past 25 Congresses have seen anywhere from 10 to over 25 thousand issues
over 2 years, only around 10\% are actually discussed~\cite{GovTrack2022}.
That would be approximately 1000 to 2500 issues per Congress, or about 500 to 1250 per
year.
While this is still a large number of issues, it is the job of a member of the House
of Representatives to learn about, research, and deliberate about each issue.
Additionally, Miller states `a representative should be an expert, or at least
competent, in each field [on which they are voting]'~\cite{Miller1969}.
This reinforces the idea that a member of the House of Representatives should be, as
their title would suggest, a representative of the people and have the responsibility
to become an expert in the issues they are voting on.
To remove this responsibility from the House of Representatives would be to remove
a large portion of their duties, and could easily result in the dictatorship of a few.
As such, we differ from Miller in the sense that all voters ought to be experts in
the field, and so we use proxy voting to allow them to be more efficient in their
duties and avoid spreading disease instead of reworking the system entirely.
Additionally, Miller did not consider using proxy voting for use by members of
Congress as it currently works, which we will explore in this paper.

\etal{Cohensius}~\cite{Cohensius2017} explore the use of proxy voting in a metric space
using three voting mechanisms: mean, median, and majority.
They discovered proxy voting using any of these mechanisms generally produces lower
error than direct voting with active voters alone.
This is not too surprising: reintroducing information lost through inactive voters by
using a proxy system ought to help the system.
Nevertheless, they were able to show that proxy voting is effective under a number of
symmetrical and asymmetrical preference distributions, while under both random and
strategic participation.
However, the majority of their research focuses on voting with infinite populations.
While this work would certainly be applicable to larger populations, since a
population of sufficient size will begin to behave like an infinite
population~($\lim_{x \rightarrow \infty} x = \infty$), we are more interested in the
effects of proxy voting on a relatively small, finite population of 435 members of the
House of Representatives.
As such, we will explore the effects of proxy voting on a finite population of this
size, as well as explore other possible voting mechanisms.

Anurita~Mathur~and~Arnab~Bhattacharyya~\cite{Mathur2017} looked at several voting
mechanisms applied on a single-winner election vote and determined a ranking for
these mechanisms.
They apply these mechanisms on a dataset while looking only at datapoints without a
Condorcet winner.
In their work, they say a mechanism `beats' another if it has a larger fraction of
the population prefer its output over the other's output.
They discover that the GT\footnote{
    Presumably meaning `Game Theory.'
}~method~\cite{Rivest2010} beats all others, the Schulze~method~\cite{Schulze2011}
and Minimax voting mechanisms always agree and beat all other mechanisms besides the
GT method, while Borda beats Copeland and Plurality, and Plurality comes in last.
This study will also look at voting mechanisms and attempt to determine which
mechanism is best suited for proxy voting.
Our work will differ significantly from theirs, however, as we will explore voting
mechanisms used in a continuous voting space instead of discrete-space, single-winner
elections.
Additionally, most of our mechanisms will be different due the mechanisms they used
not being beneficial on a continuous preference space or with a small number of
candidates, or not working well with proxy voting.

Jonas Degrave~\cite{Degrave2014} implemented a simple model to allow voters to
delegate to multiple proxies.
He treated proxy delegations as a digraph where nodes are voters and edges represent
delegating proxies.
He developed two algorithms to determine the weights of each proxy.
The first algorithm, which calls for simply dividing their weight equally among all
those to which the voter delegates, is precisely the model we will use.
This technique allows for a straightforward way to delegate voting power that would
not be confusing to voters.
We additionally augment this technique by looking at how many proxies a voter should be
allowed to delegate.
As with Degrave's approach, a delegator's voting power will be divided equally amongst
all its proxies.
Intuitively, if a voter were to delegate all other voters as proxies the result would
be the same as if the voter had simply not voted.
However, if the voter were to only delegate a single proxy, the result might not be
as ideal to the delegator or system as if they had been able to delegate to two proxies
that would produce a better result.
As such, we ask, if voters need to delegate more than one proxy with equal weighting and
will always select those closest to them, how many proxies should be allotted before
more proxies cease to be useful?

James Fearon~\cite{Fearon1998} described how cooperation often occurs in two phases:
bargaining and enforcement.
The bargaining phase involves negotiation and determining the terms of an arrangement,
while the enforcement phase involves ensuring that the terms of the arrangement are met.
Generally votes take place at the end of the bargaining phase, and the result is the
agreement to be used in the enforcement phase.
We plan to expand voting to feed back into the bargaining phase by allowing voters to
express their dissatisfaction with either side of an issue by voting towards the
center of the interval.
In the case of binary issues, such as yea/nay issues, if enough agents vote towards the
center the bargaining phase can be restarted to allow more negotiation before
performing a second vote, thereby facilitating deliberation.
For more continuous issues, such as how much money to spend on defense, the
continuous model allows agents to vote for moderate spending by voting towards the
center, instead of high or low spending by voting at one of the extremes.

\section{Preliminary Setup}\label{sec:preliminary-setup}

\subsection{The Model}\label{subsec:the-model}
An important part of any study is the model it uses to represent the system being
studied.
We employ a model described by \etal{Cohensius} in their 2017
article~\cite{Cohensius2017}.
This model places voters' preferences in a single-dimension continuous
metric~space~\systemspace, such as in~\autoref{fig:system-metric-space}.
In this model, two points that are close together in the metric space represent
similar preferences, while two points that are far apart represent very different
preferences.
As a point moves further away from the agent's preference, the agent likes it less.
This model works best when an upper and lower bound is provided, such as only
allowing agents to vote in the interval $[-1, 1]$, to prevent agents from voting
extremely far in either direction and so potentially biasing the vote.
Not all methods of voting are susceptible to such attacks, but it must be a
consideration for those that are.

The difference between multiple equidistant points in the model may not be equivalent
to the agents.
For example, an agent who must spend \$30 may prefer spending more money rather than
less.
As such, even though \$29 and \$31 are equidistant to \$30, an agent may vastly
prefer \$31 over \$29.
However, for our model, we will assume agents only care about the distance from their
preference, rather than if it is greater or less than some amount.

\begin{figure}[htbp]
    \centering
    % Built using:
% https://tex.stackexchange.com/a/148253/277236
% https://tex.stackexchange.com/a/380491/277236
\begin{tikzpicture}[scale=7.0]
    \draw(-1,0) -- (1,0) ; % Axis
    \foreach \x in {-1, 0, 1} % Numbers and lower lines
    \draw[shift={(\x,0)},color=black] (0pt,2pt) -- (0pt,0pt);
    \foreach \x in {-1, 0, 1} % Numbers and lower lines
    \draw[shift={(\x,0)},color=black] (0pt,0pt) -- (0pt,-2pt) node[below]{$\x$};

    % Labeled points
    \tkzDefPoint((-4/7), 0){agentA}
    \tkzDefPoint((3/4) , 0){agentB}
    \tkzDefPoint((1/12), 0){agentC}
    \tkzLabelPoint[above](agentA){$\truthof{a}$}
    \tkzLabelPoint[above](agentB){$\truthof{b}$}
    \tkzLabelPoint[above](agentC){$\truthof{c}$}

    \foreach \n in {agentA, agentB, agentC}
    \node at (\n)[circle,fill,inner sep=1.75pt]{};
\end{tikzpicture}
    \caption{
        Example of a 1D continuous preference metric space, where \truthof{x} represents
        the preference of agent $x$.
        The x-axis represents some preference space.
        An agent can have a preference anywhere within this space.
        One way to interpret the model is to have the leftmost point be the most
        against some idea and the rightmost point is the most in favor of the same idea.
        Importantly, points towards the center of the space are the most ambivalent,
        neutral, or central on the idea.
        Alternatively, points towards the center may also prefer some type of
        compromise or alternative solution instead.
    }
    \label{fig:system-metric-space}
\end{figure}

Such a model is extremely flexible and can be interpreted in different ways.
When applied to binary for-or-against voting problems, options can be placed at either
extreme of the interval.
As an example, shareholders at some company are voting on new data collection policies.
Using the interval $[-1, 1]$, we can place `no collection at all' at -1, and `heavy
collection' at 1.
Agents vote according to their preference: -1 for no collection, 1 for heavy collection.
So far, everything works the same as a normal vote.
However, due to the continuous nature of the voting space, the agents can vote
anywhere in the given interval.
Therefore, agents who do not care one way or the other can vote at 0 instead of being
forced to choose an option.
Additionally, those that are only slightly in favor (meaning they would prefer
collection but do not really care that much) can choose some value between 0 and 1.
Once the votes are aggregated, the result can be rounded to whichever option is
closest.
The continuous space allows agents to better express themselves according to what
their actual preference is, instead of being forcibly binned into one value or the
other.

Alternatively, if the majority of the votes are about the center, it may be a sign
neither option is satisfactory.
Using the data collection example, we can reinterpret 0 to mean the agents do not
mind collecting users' data, but also do not personally care either way.
These agents would vote around 0 to avoid imposing.
By making the agents aware of how 0 will be interpreted, they can express their
dissatisfaction by voting at or around 0.
Normally, voting goes through a bargaining and an enforcement
phase~\cite{Fearon1998}, but if a sufficient number of agents vote close to 0, the
group can reopen discussion about the topic and re-bargain before conducting a new vote.
This enhances the cooperation aspect of voting, and creates a fundamental change:
voting is no longer the end of the negotiation, but rather part of it.

Finally, the continuous space model allows for another, very powerful interpretation:
interpreting the votes and result as continuous values.
For example, imagine Congress is voting on how much to allocate to the defense budget.
In this scenario, Congress can decide to allocate anywhere between \$0 and \$1,000.\footnote{
    Naturally, these values are not realistic and are meant to be representative.
}
A voter can place their vote anywhere between those values, and the result of the
vote can be how much Congress allocates.
Say there are three voters, each with their own preferences, and all voters are
active (no voter delegates their vote).
Agent $a$ prefers \$750, $b$ prefers \$600, and $c$ prefers \$250.
One way we could aggregate these preferences is by finding their mean.
By averaging these preferences, the system would output
$\frac{\$(750 + 600 + 250)}{3 \text{ voters}} = \frac{\$1600}{3 \text{ voters}} =
\$533.33$,
which would be the amount allocated to the defense budget.
Being able to vote on continuous problems and yield a continuous output, as well as
working with binary issues, makes the model extremely flexible and allows it to
tackle any number of problems.

We are also able to easily calculate error using a continuous space.
The error can simply be the distance from the result of the system when all agents
are active, and the result under proxy voting.

The flexibility of the continuous model in both the discrete and continuous realms, its
ability to use different voting rules, easy interpretability, and easy error
calculation are the reasons it is employed in this study.
We will focus primarily on the continuous instead of the discrete output of the
model, since observing the continuous output allows for more granularity in the
differences between proxy and non-proxy voting, as well as in the differences between
voting rules.

\subsection{Voting Mechanisms}\label{subsec:voting-mechanisms}
This study also makes use of \textit{voting mechanisms} or \textit{voting rules},
which are functions that map a set of preferences in~\systemspace\ to an outcome that
also exists in~\systemspace.
For these, we take inspiration from \etal{Bulteau}'s~\cite{Bulteau2021} work in
aggregating one-dimensional single-winner elections by using their $L_p$ aggregation
methods, as well as mixing in plurality.
$L_p$ aggregation methods work by minimizing the sum of distances to the power of $p$
($d^{\,p}$, where $d$ is a distance) between a possible solution and the voters'
preferences.
Naturally, since~\cite{Bulteau2021} did not use weighted proxies, these methods need
to be adjusted to allow for weight.
Below, \agentweight\ represents the weight of an agent \agent, and \systemproxies\ is
the set of active voters ordered by preference.
Additionally, \agenttruth\ is the preference of an agent, and \system\ is the set of
all agents.
With these notations, the mechanisms we use are
\begin{enumerate}
    \item {
        \textit{Median ($L_1$)}, defined as
        $\text{let } i =
    \min i \text{ s.t. } \sum{j = 1}{i}
        \weightof{{\systemproxies}_j} > \frac{\systemweight}{2}
;
\textbf{md}(\systemproxies) = \truthof{\agent_i}
$.
         % 05/13/2023: \vicki{
         %  In your original definition, you need to control j (starting at 1) so we know it is a subscript in Sp.
% This notation doesn't say what you want.  Nothing says that j begins at 1 in Sp.
%
% Better as:
% let i = min i such that (sum(j=1,i) w(Spj) >w(S)/2)
% md(Sp) = T(ai)
%
%
%          }
        % This definition is saying the median is the preference of the first agent
        % whose additional weight makes the sum of weights more than half the total
        % weight of the system.
        %
        % `j` refers to `agent j,` as in the agent from the set of proxies that is
        % not agent i. I could start at agent 1 and sum the weights to agent i.
        % Instead I said we sum the other agents (agent j) from the set of proxies
        % up to but not including agent i (which is why it goes to agent (i - 1)).
        %
        % I deliberately add agent i outside the sum to make it extremely apparent
        % their weight is the weight that achieves the median.
        Essentially, the median is the active agent whose additional weight makes the
        sum of weights become equal to or greater than half the total weight of the
        system.
        The sum will occur in the same order as the ordering of voters in
        \systemproxies.
        There is an edge case where the sum of weight perfectly reaches half the
        total weight.
        When this occurs, the preference that reaches half is called the lower
        weighted median, and the preference that starts from half is called the higher
        weighted median.
        In this case both the lower and higher weighted medians,
        % 05/05/2023: \vicki{Awkward, lower weighted median isn't really a thing.}
        % I'm not sure what else to call it. A quick internet search indicates these
        % are known terms for the lower and higher bounds of a weighted median.
        % I've added a sentence clarifying what these terms mean.
        as well as any value in between, can claim to be the median.
        This is similar to when there are an even number of preferences in a normal
        median.
        In this case, we take the average between the preference of the agent with
        the highest preference in the lower median half and the agent with the lowest
        preference in the higher half of the weighted median.
    }
    \item {
        \textit{Mean ($L_2$)}, defined as
        $\mathbf{mn}(\systemproxies) =
    \frac{1}{\systemweight}
    \sum_{\agent_i \in \systemproxies} {\weightof{\agent_i} \cdot \truthof{\agent_i}}$.
         % \vicki{
         %  I'm confused.  Even if Sp is only the active agents, isn't the weight of Sp the same as the weight of S?
         %   If so, why use S at all in this equation?
         % }
        % S and Sp are different, S being the set of all agents and Sp being the set
        % of active agents.
        % I realized I forgot to define S, so I've added that above.
        %
        % I feel it's important to make this distinction because we want to divide by
        % the weight of all agents, not just the weight of the active agents.
        This is a typical weighted average.
    }
    \item {
        \textit{Mid-range ($L_\infty$)}, defined as
        $\mathbf{mr}(\systemproxies) =
    \frac{
        \truthof{{\systemproxies}_1} + \truthof{{\systemproxies}_n}
    }
    {2}
$, where $n$ is the number
        of proxies
        This equation essentially means the preference of the active agent with the
        lowest preference plus the preference of the active agent with the
        highest preference, divided by two.
        %
        % 05/13/2023 \vicki{Since Sp is ordered, don't you just want T(Sp1) and T(Spn)?
        % You say SP is the ordered set of active voters.  You need to say it is ordered by preference (not weight).
        % I wonder if  you just need better notation.  Instead of Sp, call it W (for weighted proxies, ordered by preference) and then elements can be w1, w2, ... wn
        % }
        % I'll change Sp to W, but w is already used to mean the weight function.
    }
\end{enumerate}
In addition to applying these voting mechanisms on the set of active agents, we
% 05/05/2023: \vicki{Not sure what is meant by additionally.  Were you meaning that the
% voting mechanism was used for some other reason?}
% Yes, the mechanisms as described above are applied to the weighted set of active
% voters. Here I'm explaining  my baseline, those being using these mechanisms on all
% voter, active and inactive, as well as applying them to only the active voters
% without weights. I've rephrased things to try to make it more apparent.
apply these mechanisms to active voters without weights (to simulate if proxy voting was
not allowed) and all voters, both active and inactive, (to simulate the best case
where all agents vote).
These additional calculations serve as baselines to determine how well proxy voting
works in these scenarios.

\subsection{Coordination Mechanisms}
\label{subsec:coordination-mechanisms}
There are also several ways an individual proxy can agree to cast its vote on behalf
of its constituents.
Each has different advantages and disadvantages for the proxy, its constituents, and
the system as a whole.
We will examine five different `coordination' mechanisms that `groups' (meaning a
proxy and its constituents) can use:
% 05/05/2023: \vicki{
%   Consider removing the "no preference change" to the first four, including
%   preference change for the last four.
%   The reader isn't ready for the preference change idea, as it hasn't been introduced,
%   so it is just confusing.
% }
\begin{enumerate}
    \item {
        \textit{Expert}.
        The group applies its total weight to the proxy's preference.
    }
    % \item {
    %     \textit{Active Only}.  \vicki{The previous sentence indicates these govern how groups coordinate.  This doesn't really meet that criteria.
 % The name also seems odd. }
 %        Only active agents vote with no additional weight.
 %    }
    % This is a baseline which I discussed previously, so I've removed it
    \item {
        \textit{Cooperative Mean}.
        The group allocates its weight to the mean of the proxy's and its
        constituents' preferences.
    }
    \item {
        \textit{Cooperative Median}.
        The group allocates its weight to the median of the proxy's and its
        constituents' preferences.
    }
    % \item {
    %     \textit{Preference change - Expert}.
    %     The group applies its total weight to the proxy's preference but allows the proxy to change its vote.
    %     This is useful for cases when a proxy is deemed by its constituents as an
    %     `expert', as discussed by James Miller~\cite{Miller1969}.
    %     However, active agents' preferences have changed as they have participated in
    %     deliberation and discussion with the other active voters.
    % }
    % \item {
    %     \textit{Preference change - Expert with Consequences}.
    %     The proxy is an expert and the group allocates its weight to the proxy's
    %     preference, but active agents', including the proxy's, preferences have changed.
    %     However, due to this change the proxy is only allowed vote for itself,
    %     meaning it can only use its own weight.
    %     This causes its constituents to be unrepresented.
    %     This represents situations where a proxy is considered an expert, but the
    %     constituents impose a condition that the proxy can only represent them if
    %     they vote a certain way.
    %
    %     Note this is analogous to when agents change their preference and only active
    %     agents are allowed to vote.
    %     % 05/05/2023: \vicki{Not if the proxy doesn't change their vote.  I think you
    %     % have to handle this case because otherwise, it makes no sense to have a new name.
    %     % }
    %     % Naturally this doesn't apply to when agents change their vote. That's why I
    %     % specify they've changed their preference. Nevertheless, I've extracted
    %     % these and explain the coordination mechanisms are also applied to when the
    %     % agents change their votes.
    %
    % }
    % \item {
    %     \textit{Preference change - Cooperative Mean}.
    %     The group allocates its weight to the mean of the proxy's and its
    %     constituents' preferences.
    %     However, active agents' preference has changed as it has participated in
    %     deliberation and discussion with the other active voters.
    %
    % }
    % \item {
    %     \textit{Preference change - Cooperative Median}.
    %     The group allocates its weight to the median of the proxy's and its
    %     constituents' preferences.
    %     However, active agents' preferences have changed due to deliberation.
    % }
\end{enumerate}
These mechanisms are used in an attempt to simulate real-world consequences of proxy
voting, as well as identify potential techniques to deter or mitigate a proxy's
ability to swing the system by aggregating a large amount of weight and abusing it.

These mechanisms will additionally be applied after active agents change their
preferences in order to determine how well they represent the agents after
deliberation has occurred.

\section{Assumptions}\label{sec:assumptions}
In order to conduct this study, we make a number of assumptions.
First, we assume that voting issues are one off, meaning agents vote on only one
topic at a time.
This allows agents to select a proxy that works best for the current topic instead of
selecting the best proxy for all topics.

Additionally, we only consider scenarios where single-vite proxy voting is used, where
the proxy receives the voting power of the delegating voter, increasing their weight,
and proceeds to allocate all their weight towards one vote instead of relaying their
constituents vote.
This type of proxy voting also allows the proxy to update their (and by extension, their
constituents) preferences as new information becomes available.
While single-vote proxy voting gives substantial flexibility to the proxy to operate
on behalf of their constituents, this flexibility requires the delegating voter to
choose a proxy who they trust to vote as close to how they themselves would.
This is a process similar to selecting experts, as described by~\cite{Miller1969}
and~\cite{Mueller1972}.
By using single-vote proxy voting instead of relay-style voting, we hope to exploit
the advantages of proxy voting that the relay-style does not provide.

We also assume that each voter has reasonable knowledge of potential proxies'
opinions, meaning they have a decent idea of the preferences of other proxies.
This will allow them to choose the proxy that has the opinion most similar to their own.
While in reality voters will likely not have perfect knowledge of others' opinions,
it is often not particularly difficult to gauge the opinion of others, especially
those with whom an individual often associates, and so we believe this assumption is
reasonable.

Next, we assume that abstention is not allowed.
In other words, all agents must vote either themselves or by proxy.
The primary reasoning behind this is to simplify the system being used and provide
the system with as much information as possible without needing to account for
extenuating circumstances such as the unexpected incapacitation of a voter.
Nevertheless, as baselines , we will explore occasions where those who are not
physically present are unable to vote as well as the result when all agents are able
to vote.

Finally, we assume that there are no factors besides closeness in opinion that affect
the choice of proxy.
This differs from some systems, such as that presently used by the House of
Representatives which includes restrictions such as a proxy can only serve ten
voters~\cite{CERP2020}.
However, we feel removing restrictions such as these leads to a more interesting
discussion, since it allows the use of different voting mechanisms and more extreme
cases.

\section{Contribution}\label{sec:contribution}
We explore proxy voting in a unified-vote/single-winner single-dimension
continuous space model.
We employ three well-known $L_p$ mechanisms, specifically
\begin{enumerate}
    \item {
        Median ($L_1$)
    }
    \item {
        Mean ($L_2$)
    }
    \item {
        Mid-range ($L_\infty$)
    }
\end{enumerate}
Each of these mechanisms will additionally be applied to direct voting with all
agents and direct voting with only those agents that are present.
This is done to show what the output of the system would be with all information
(direct voting with all agents), as well as the output with minimum information
(direct voting with only those agents that are present).
The error will be the distance between the result and direct voting with all agents.

We additionally apply what we've dubbed `coordination mechanisms,' which are
techniques by which proxies and constituents work together to determine how their
weight will be allocated, meaning how the proxy will vote on their behalf.
The mechanisms we explore are described
in~\autoref{subsec:coordination-mechanisms}.
These mechanisms are used in an attempt to simulate real-world consequences of proxy
voting, as well as identify potential techniques to deter or mitigate a proxy's
ability to swing the system by aggregating a large amount of weight and abusing it by
misallocating it towards a vote the constituents do not prefer.
% 05/05/2023: \vicki{
%   Not clear what the abuse refers to.
%   They got the weight fairly.
%   Is the abuse in threatening to change the preference for all?
%   Not all aggregation methods would permit abuse???
% }
% I've added something to specify "abuse" means "misallocating."
% Some voting mechanisms (which I assume is what is meant by "aggregation methods")
% may not permit abuse, but inactive voters don't participate in the voting
% mechanisms; they vote through the proxy.
% This means the coordination mechanism needs to discourage the proxy from abusing
% (misallocating) its constituents votes.
%
% If "aggregation methods" was supposed to mean "coordination mechanisms," the
% methods that don't permit abuse only apply to when the proxy and its constituents
% are choosing their group preference, but wouldn't help when the proxy actually goes
% to cast the vote. To combat misallocation abuse, we introduce the "with
% consequences" coordination mechanisms.

Whereas voting on a continuous interval is uncommon and finding a real-world dataset
using preferences on an interval currently does not seem possible, these investigations
are performed using preferences generated from several statistical distributions.
We use various distributions to allow the gathered data to represent
different distributions of voters in the real world.
These distributions include well-known statistical distributions, such as the uniform
distribution, the Gaussian distribution, as well as a few beta distributions.
The distributions used and their notations are listed
in~\autoref{tab:distributions-used}.

\begin{table}[!htbp]
    % increase table row spacing, adjust to taste
    \renewcommand{\arraystretch}{1.3}

    \caption{
        The distributions to be used to generate preferences.
        Note how each distribution represents a population type.
        These types are representative, and any distribution could potentially
        represent a different population type that shares the same shape as the
        distribution.
        Additionally, any skewed distributions can be inverted to create a
        distribution that is skewed in the other direction (e.g. a distribution
        skewed in favor can be inverted to create a flipped distribution skewed
        against).
    }
    \label{tab:distributions-used}

    \centering
    \begin{tabular}{|r|l|c|l|}
    \hline
    \thead{Distribution} & \thead{Notation} & \thead{Symmetrical?} & \thead{Population Type}
    \\
    \hhline{|=|=|=|=|}
    Uniform & \uniform{-1}{1} & \ding{51} & Evenly spread
    \\
    \hline
    Normal/Gaussian & \gaussian{0}{\sfrac{1}{3}} & \ding{51} & Mostly
    centrist/indifferent
    \\
    \hline
    Beta(0.3, 0.3) & \betadistribution{0.3}{0.3} & \ding{51} & At either extreme
    \\
    \hline
    Beta(50, 50) & \betadistribution{50}{50} & \ding{51} & Strongly
    centrist/indifferent
    \\
    \hline
    Beta(4, 1) & \betadistribution{4}{1} & \ding{55} & Skewed in favor
    \\
    \hline
\end{tabular}
\end{table}

Initial experiments show larger populations behave similarly to those with fewer.
As such, each experiment will have a population of 24.
We chose this number because prior experimentation shows it to be a good balance
between having few enough agents that an agent going inactive can change the
results, while not so few that an agent going inactive is catastrophic.
Since having no delegates or all delegates would not be interesting, we will
experiment with 1 to 22 inactive agents and examine how error correlates with
the number of delegators.
We chose 1 inactive agent as the minimum because the case with 0 agents is not
interesting due to there not being any proxies.
% 05/05/2023: % \vicki{I thought you did consider 0?}
% No, I never have. As we've discussed, having 0 inactive agents is the same as not
% using proxy voting and is not interesting. This plan was specified in the proposal.
% Instead, I was asked to extend my graphs to include 0 (which I did by actually
% running the simulations with 0 inactive agents).
Similarly, we chose 22 as the maximum because the case with 24 agents means there are
no active agents to vote, and the case with 23 simply devolves into the coordination
mechanism replacing the voting mechanism.

For each experiment, agents will randomly select some preference using the current
distribution.
Afterward, one agent
% \vicki{sounds like various numbers of inactive voters are used.
% Can you state that initially?}
% MDH: This is stated in the previous paragraph. I've reworded it slightly to make it
% more clear.
will become inactive and choose the closest active agent to be their proxy.
Following this, the proxies and their constituents will apply a coordination
mechanism, and each voting mechanism will be applied.
The next coordination mechanism will be used, then each voting mechanism will be
applied again.
Once all coordination mechanisms have been used, the number of inactive agents will
increase until 22 agents are inactive.
% 05/05/2023: \vicki{Earlier you said 22 is the maximum}
% Whoops, that was a leftover from when I was running up to 23. I've fixed it to be
% correct.
Finally, agents will select a new preference from the next distribution and the
process will begin again.
This will continue until all possible combinations have been used.
% 05/05/2023 \vicki{
%   For each experiment, are preferences and inactive voters re-selected?
%   The way you discuss it, it sounds like the steps are additive.
%   If they aren't there is no need to describe it as a progression.
%   You should also state that each experiment was run 1024 times.
% }
% The preferences are reselected whenever a new preference distribution is used, as
% stated two sentences prior. Since we only want to change one variable at a time
% (similar to not using different RNG seeds), we only change the preferences once all
% combinations have be used. This way, each coordination mechanism and voting
% mechanism is applied to the same preferences with the same proxies and inactive
% voters.
%
% I've specified the number of runs below.
Each combination is run 1024 times to ensure accuracy in the analysis.

We will show that proxy voting with the right combination of mechanisms generally
yields considerably lower error than active-only voting.
We also show proxy voting is beneficial even when agents' preferences change.
However, proxy voting appears to be least effective on highly-polarized topics.


\input{common/references}
